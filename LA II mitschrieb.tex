\PassOptionsToPackage{dvipsnames}{xcolor}
\documentclass[titlepage,12pt,a4paper,ngerman]{report}
\usepackage[utf8]{inputenc}
\usepackage[T1]{fontenc}
\usepackage[german]{babel}
\usepackage{graphicx}
\usepackage{wrapfig}
\usepackage{amsmath}
\usepackage{amsfonts}
\usepackage{amssymb}
\usepackage{tikz}
\usepackage{tikz-cd}
\usepackage{nicefrac}
\usepackage{mathtools}
\usepackage{enumerate}
\usepackage{cancel}
\usepackage{hyperref}
\usepackage{cleveref}

\newenvironment{bew}[1]{\subsection{Bew: #1}}{\hfill$\square$}

\def\checkmark{\tikz\fill[scale=0.4](0,.35) -- (.25,0) -- (1,.7) -- (.25,.15) -- cycle;} 

\newcommand{\Bew}[2]{\begin{bew}{#1}#2\end{bew}}
\newcommand{\verteq}{\rotatebox{90}{$\,=$}}
\newcommand{\equalto}[2]{\underset{\scriptstyle\overset{\mkern4mu\verteq}{#2}}{#1}}
\newcommand{\equaltoup}[2]{\overset{\scriptstyle\underset{\mkern4mu\verteq}{#2}}{#1}}
\newcommand{\custo}[3]{\underset{\scriptstyle\overset{\mkern4mu\rotatebox{-90}{$\,#1$}}{#3}}{#2}}
\newcommand{\custoup}[3]{\overset{\scriptstyle\overset{\mkern4mu\rotatebox{-90}{$\,#1$}}{#3}}{#2}}
\newcommand{\tx}[1]{\textrm{#1}}
\newcommand{\ov}[1]{\overline{#1}}
\newcommand{\ub}[1]{\underbrace{#1}}
\newcommand{\ob}[1]{\overbrace{#1}}
\newcommand{\im}{\tx{im}}
\newcommand{\spa}{\tx{span}}
\newcommand{\adj}{\tx{adj}}
\newcommand{\grad}{\tx{grad}}
\newcommand{\ord}{\tx{ord}}
\newcommand{\basis}[3]{\{#1_{#2}, \dots, #1_{#3}\}}
\newcommand{\const}{\tx{const.}}
\newcommand{\summ}[2]{\sum_{#1}^{#2}}
\newcommand{\intt}[2]{\int_{#1}^{#2}}
\newcommand{\casess}[4]{\left\{ \begin{array}{ll} {#1} & {#2} \\ {#3} & {#4} \end{array} \right.}
\newcommand{\dmat}[3]{\begin{pmatrix} #1_{#2}&&\\ &\ddots& \\ && #1_{#3} \end{pmatrix}}
\newcommand{\ska}[2]{\langle #1 , #2 \rangle}


\newcommand{\enph}{F: V \to V \textrm{ Endomorphismus}}

\newcommand{\hfw}{\color{RubineRed}\tx{ $\star$hier fehlt was$\star$ } \color{black}}

\hbadness=99999

\begin{document}
	
\renewcommand{\thechapter}{\Roman{chapter}}

\title{
	\Huge Lineare Algebra II \\[1em]
	\Large Vorlesung von Prof. Dr. Amador Martin - Pizarro im Sommersemester 2018}
\author{Markus Österle\\ Andréz Gockel}
\date{17.04.2018}
\maketitle
\tableofcontents

\chapter{Wiederholung}
\subsection{Def: Ringe} Ein (kommutativer) Ring (mit Einselement) ist eine Menge zusammen mit zwei binären Operationen + und *, sodass:
\begin{itemize}
	\item $(R,+,0_{R})$ ist eine abelsche Gruppe
	\item $(R,*,1_{R})$  ist eine kommutative Halbgruppe
	\item $(R,+,0_{R})$ die distributiven Gesetze:\\
	$x(y+z) = xy+xz $\\
	$(x+y)z=xz+yz $
	gelten
\end{itemize}

\subsection{Def: Integritätsbereich} Ein Integritätsbereich ist ein Ring ohne Nullteiler $$\forall x,y \in R: \quad (x\cdot y = 0 \Rightarrow x=0\textrm{ oder } y=0)$$ 
\subsection{Def: Körper} Ein Körper K ist ein Ring derart, dass:
\begin{itemize}
	\item $1_{K} = 0_{K}$
	\item $\forall x \in K \setminus \{0\}$ $\exists x^{-1} x\cdot x^{-1} = x^{-1} \cdot x = 1_{K}$
\end{itemize}
\subsection{Bew:} Körper sind Integritätsbereiche
\subsection{Bem: Ring Homomorphismus}  Sei R ein nichttrivialer Ring $(0_{R} \neq 1_{R})$,
$$\varphi: \mathbb{Z} \to R , n \mapsto \left\{\begin{array}{lll} & \ 1_{R}+\dots + 1_{R} & n\ge 0\\ - & ( 1_{R}+\dots + 1_{R}) & n < 0 \end{array}\right.$$\\
$\varphi$ ist ein Ring Homomorphismus:
$$\tx{ker}(\varphi) = \{n\in Z | \varphi(n) = 0\}$$\\
2 Möglichkeiten
\begin{itemize}
	\item[a)] $\ker(\varphi) = {0}$
	R hat Charakteristik 0
	\item[b)] $\ker(\varphi) \neq {0}$
\end{itemize}

$\rightarrow$ es existiert ein kleinstes positives Element $p>0$ in  $\ker(\phi)$%, R hat Charakteristik p
\subsection{Bew:} 
R Integritätsbereich $\Rightarrow$ p eine Primzahl z.B. 
\begin{equation*} \frac{\mathbb{Z}}{p\mathbb{Z}} = \{\bar{0}, \bar{1},\dots,\overline{n-1}\} \end{equation*}
hat \underline{Charakteristik n}.\\
Insbesondere enthält jeder Körper der Charakteristik p eine „Kopie“ von $\frac{\mathbb{Z}}{p\mathbb{Z}}$\\
K Charakteristik p $\Rightarrow$ $\frac{\mathbb{Z}}{p\mathbb{Z}} \buildrel \tx{injektiv} \over \rightarrow K$\\
$\frac{\mathbb{Z}}{n\mathbb{Z}}$ ist ein Körper:
$a \in \frac{\mathbb{Z}}{p\mathbb{Z}}\setminus \{0\}$\\
$\Rightarrow$ a und p sind teilerfremd\\
$$ 1 = a\cdot b+p\cdot m \Rightarrow 1 = \bar{a} \cdot  \bar{b}$$
\subsection{Def: Polynomring} Sei K ein Körper. Der Polynomring K[T] in einer Variable T über K ist die Menge formeller Summen der Form $$ g,f := \sum^{n}_{i=0} a_{i} \cdot T^{i}, \qquad a_i\in K $$
$\grad(f) := \textrm{max}\{i\vert a_{i} \neq 0 \}$ \\
$ \grad(0) := -1$
\subsection{Bem:} K[T] ist ein Integritätsbereich 
$$(\sum^{n}_{i=0} a_{i}\cdot T^i)+(\sum^{m}_{j=0} b_{j}\cdot T^j) = \sum^{\textrm{max}\{n,m\}}_{k=0} (a_{k}+b_{k})\cdot T^k$$
$$(\sum_{i=0}^{n}a_{i}\cdot T^{i})\cdot (\sum_{j=0}^{m}b_{j}\cdot T^{j}) = \sum_{k=0}^{n\cdot m}c_{k}\cdot T^{k}$$
$$ c_{k}=\sum_{i+j=k}a_{i}\cdot b_{j}$$

\noindent $f,g$ beide $\neq 0$
$$\Rightarrow \grad(f\cdot g) = \grad(f) + \grad(g)$$
$f\cdot g \neq 0$
$$\grad (f + g) \le \textrm{max}\{\grad(f), \grad(g)\}$$
\subsection{Satz:(Division mit Rest)}
Gegeben $f,g \in K[T]$\\
$\grad(g)> 0$ \\\\
Dann existieren eindeutige $q,r\in K[T]$\\
sodass $f=g\cdot q+r$\\
wobei $\grad(r) < \grad(g)$ eindeutig\\
$f = g\cdot q+r = g\cdot q' + r'$\\
$g\cdot(q-q') = r' - r$\\
$g\neq g'$
\begin{gather*}
\grad(g\cdot(q-q')) = \grad( r' - r) = \textrm{max}\{ \grad(r'),\grad(r) \} < \grad(g) \\ \grad(g\cdot(q-q')) \buildrel q\neq q'\over = \grad(g) + \grad(q-q') 
\end{gather*}
$\Rightarrow$ Widerspruch (Wid)!\\
$\Rightarrow$ $q = q' \Rightarrow r = r'$\\\\
\subsubsection{Existenz Beweis: Induktion auf $\grad(f)$}
I.A.: $\grad(f)= 0 \rightarrow f = g\cdot0 + f$\\
,,n+1`` $\grad(f)= n+1$
$$\grad(f) < \grad(g) = m$$
$$\rightarrow f = g\cdot 0 + f$$

OBdA $n+1 = \grad(f)\ge \grad(g) = m > 0$\\
$f = a_{n+1} \cdot  T^ {n+1} + \tilde{f}$\\
$\grad(\tilde{f}) \le n$\\
$ a_{n+1} \neq 0$\\
Sei $$f' := f - b_{m}^{-1}\cdot  a_{n+1} \cdot  T^{n+1-m} \cdot  g$$
$\Rightarrow \grad(f') \le n$
$$ g= \sum_{i=0}^{m} b_{i} \cdot  T^i$$

$\buildrel \tx{I.A.}\over \Rightarrow f' = g \cdot  q' + r'$
$$ f - b_{m}^{-1} a_{n+1} T^{n+1-m} \cdot g$$
$\grad(r') < \grad(g)$
$$\rightarrow f = g(\underbrace{b_{m}^{-1} a_{m} T^{n+1-m} + q'}_{q}) + r' ?= \grad(r') < \grad(g)$$
$( r' = r)$

\subsection{Def: Teiler} 
$f,g \in K[T]$\\
$ \grad (g)> 0$
$$g \textrm{ teilt } f \Leftrightarrow f = g\cdot q$$
$(g|f)$ $(r=0)$\\
%$(T-1|f)$
\subsection{Def: Nullstelle} $f \in K[T]$ besitzt eine Nullstelle $\lambda \in K$\\
gdw (genau dann wenn) $(T-\lambda) | f \Leftrightarrow f(\lambda) = 0$ \\
$f = (T-\lambda) q + r$\\
\subsection{Bem: Anzahl Nullstellen} $$f\in K[T] , \qquad f\neq 0 , \grad(f) = n$$\\
Dann besitzt f höchstens n viele Nullstellen in K.\\
\subsection{Bew:} $n= 0,  f = a_{0} \neq 0$\\
$n> 0 $ Falls f keine Nullstellen in K besitzt $\Rightarrow$ Ok !\\
Sonst, bei $\lambda\in K $ eine Nullstelle von $f$. \\
$$f = (T-\lambda) \cdot g$$\\
$ \grad (g) = n-1< n$\\
$\buildrel \tx{I.A.}\over \rightarrow$ besitzt g höchstens $n-1$ viele Nullstellen.\\
Jede Nullstelle von f ist $\lambda$ oder eine Nullstelle von g $\Rightarrow$ f hat höchstens n viele \underline{Nullstellen}.\\
\subsection{Def: Vielfachheit der Nullstellen} $f\in K[T], f \neq 0 , \lambda \in K$ Nullstellen von f\\
$$\rightarrow f = (T-\lambda)^{K_{\lambda }} \cdot g, \quad (g(\lambda) \neq 0)$$\\
($K_{\lambda}$ ist die Vielfachheit der Nullstelle $\lambda$ in f)\\
\subsection{Def: Algebraische Abgeschlossenheit} Ein Körper heißt algebraisch abgeschlossen falls jedes Polynom über K positiven Grades eine Nullstelle in K besitzt.
\subsection{Frage:} Ist $\mathbb{R}$ algebraisch abgeschlossen ?\\ 
Nein: $T^2+1$
\subsection{Bem:} $\mathbb{C}$ ist algebraisch abgeschlossen 
\subsection{Bem: Unendlichkeit} Jeder alg. abg. Körper muss unendlich sein!
\subsection{Warum?:}
(Beweis läuft wie unendlichkeit der Primzahlen)\\
$K = \{\lambda_1,\dots,\lambda_n\}$\\
$f=(T-\lambda_1)\cdot \dots \cdot (T-\lambda_n) +1$
\subsection{Bem: Algebraische Abgeschlossenheit} K ist genau dann alg. abg. wenn jedes Polynom f positiven Grades in lineare Faktoren zerfällt:\\
$$f=(T-\lambda_1)\cdot \dots \cdot (T-\lambda_n) \in K$$
\subsection{Bew:}
,,$\Leftarrow$`` Trivial\\
,,$\Rightarrow$``$\grad(f) = n>0$\\
$\rightarrow f = (T-\lambda_n)\cdot g$ \\ $(\grad (g) \le n - 1 < n)$\\
$\buildrel \tx{I.A.} \over \rightarrow g=c(T-\lambda_2)\cdot \dots \cdot (T-\lambda_n)$\\
$\Rightarrow f=c(T-\lambda_2)\cdot \dots \cdot (T-\lambda_n)$ \checkmark %häckchen of approval\\


%2. vorlesung

\subsection{Def: Vektorraum}
Vektorraum V über K ist eine ablesche Gruppe $(V,+,0_V)$ zusammen mit einer verknüpgung \\
$$ K \times V \mapsto V$$
$$(\lambda, v) \mapsto \lambda\cdot v$$
sodass:
\begin{itemize}
	\item[1.)] $\lambda(v+w) = \lambda v +\lambda w$
	\item[2.)] $\lambda(\mu \cdot v) = \lambda\mu) \cdot v$
	\item[3.)] $(\lambda + \mu) v = \lambda v + \mu v$
	\item[4.)] $ 1_K \cdot v = v \qquad\forall v \in V$
\end{itemize}

Ein Untervektorraum $U\subset V$ ist eine Untergruppe, welche unter Skalarmultiplikation, abg. ist.
\subsection{Bem: Untervektorräume} $\{U_i\}_{i\in I}$ Unterräume von V\\
$\rightarrow \bigcap_{i\in I} U_i$ ist auch ein Unterraum.\\
Insbesondere gegeben $M \subset V $ existiert :
$$\spa(M) = \langle M \rangle =\textrm{ der kleinste Untervektorram von V, welcher M enthält}$$\\
$$ \spa(M) = \{\sum^n_{i\in I} \lambda_i m_i, n\in M ,\lambda\in K, n\in \mathbb{N}\}$$
M ist also ein erzeugenden System für $\spa(M)$\\
$\{U_i\}_{i\in I}$ Unterräume von V\\
$$\rightarrow \sum_{i\in I} U_i = \spa (\bigcup_{i\in I} U_I)$$
$$M_i \subset M_2 \Rightarrow \spa(M_1) \subset \spa(M_2)$$
\subsection{Def: Lineare Unabhängigkeit}
$ V : VR/K$\\
$$v_1,\dots, v_n \in V\textrm{ lin. unabh. falls }\forall\lambda_1 , \dots ,\forall\lambda_n \in K: $$\\
$$\sum \lambda_i v_i= 0 \Rightarrow\lambda_1 =  \dots = \lambda_n = 0$$
$M \subset V$ ist lin. unabh. falls jede endliche Teilmenge von M lin. unabh. oder äquivalent dazu sind. Falls kein Element $m$ aus $M$ sich schreiben lässt als linear kombination von $M\setminus \{m \}$.
$\lambda_i \neq 0$\\
$$ \lambda_1 m_1 + \lambda_2m_2 + \dots + \lambda_n m_n = 0 \Rightarrow m = \sum - (\frac{\lambda_i}{\lambda_1}) m_i$$
\subsection{Def: Basis = min. Erz. System} Eine Basis $\mathcal{B}$ von V ist ein lin. unabh. erzeugendensystem von V, äquivalent dazu, wenn jedes Element von V sich \underline{eindeutig schreiben lässt} als lin. kombi. von Elementen aus $\mathcal{B}$. Aquivalent dazu: $\mathcal{B}$ ist min. Erzeugenden System , max. lin. unabh.\\
\subsection{Satz: Basisergänzungssatz}
$M\subset V$ lin. unabh. $\Rightarrow\exists \mathcal{B} \subset V $ Basis welche M enthält\\
Insbesondere hat jeder VR eine Basis ,,Je zwei Basen sind eine Bijekktion``\\
V ist endlichdimensional, falls V eine endliche Basis besitzt. Sonst ist $V$ unenlichdimensional.\\
$$\dim(V) = | \mathcal{B} | $$ ($\mathcal{B}$ eine Basis)\\

\subsection{Basisauswahlsatz}
$M\subset V$ erzeugendensystem $\rightarrow \exists \mathcal{B} \subset M $ Basis von V
\subsection{Bem: Dimensions Addition} 
$U\buildrel UR \over \subset V$, 
$\dim(V) < \infty \Rightarrow \dim(U) < \infty$
dim ist modular:
$$\dim(U_1 + U_2) + \dim(U_1 \cap U_2) = \dim(U_1) + \dim(U_2)$$

\subsection{Def: Direkte Summe}
$$V=U_1 \oplus U_2 \qquad \Leftrightarrow \qquad V= U_1 + U_2\textrm{ und }  U_1 \cap U_2 = \{0\}$$
$\oplus = direkte Summe$\\
$$ V = \oplus_{i \in I} U_{i} \qquad \Leftrightarrow \qquad V = \sum_{i\in I } U_i \textrm{ und } \forall i \in I $$
Die Familie $$\{U_i\}_{i \in I} \rightarrow U_i \cap \Big(\sum_{\substack{j\in J\\ j\neq i}} U_j \Big)=\{0\}$$
ist \underline{konversal}.

\subsection{Bsp:} $K^2$ ist ein $K - VR.$\\
$$\begin{pmatrix}1 \\ 0 \end{pmatrix} , \begin{pmatrix}0 \\ 1 \end{pmatrix} = e_1, e_2$$\\
$U$ = Span$(e_1) \rightarrow K^2 = U \oplus$ Span$(e_2) = U \oplus$ Span$(e_1, e_2)$

\subsection{Def: Lineare Abbildungen}
$ F: V\rightarrow W $ ist \underline{linear}, falls $$D(\lambda v+\mu u ) = \lambda F(v) + \mu F(u)$$\\
(F UR von V)
$$ Ker(F) =\{ v \in V \vert  F(v) = 0 \}$$
(F UR von W)
$$ Im(F) = \{ w \in W \vert \exists v \in V, F(v) = w\}$$
\subsection{Bem: Dimensionssatz}
Falls $\mathcal{B}$ eine Basis von v ist $\Rightarrow F(\mathcal{B})$ ist ein erzeugendensystem von $Im(F)$\\ 
$$\textrm{F injektiv} \Leftrightarrow Ker(F) = \{0\}$$\\
V endlich $$\rightarrow dim(V) = dim(Ker(F)) + dim(Im(F))$$

$$ \nicefrac{v}{Ker(F)} \cong Im(F)$$
\subsection{Bem: Isomorphie}
V, W endlich $\{r_1,\dots ,r_n\}$ eine Basis von V
$$ V \cong K^n$$
$$v_i \mapsto e_i$$
$$F: V (dim = n ) \to W (dim = m)$$

$$\{v_1,\dots ,v_n\} \qquad \{w_1,\dots ,w_m\}$$
$$\begin{array}{ccc}
V & \buildrel F \over \rightarrow &  W \\
R & & R \\
K^n & & K^m \\
\begin{pmatrix}
\lambda_1\\
\vdots\\
\lambda_n
\end{pmatrix} & \rightarrow & A \cdot \begin{pmatrix}
\lambda_1 \\
\vdots \\
\lambda_m
\end{pmatrix}
\end{array}$$

Wie bekommt man die Matirx A ?\\
$$ F(v_j) = \sum_{i = a}^m a_{ij} w_i$$\\
$$F(v_1) \dots F(v_n)$$ 
\nopagebreak
$$\begin{pmatrix}
1_{11} & \cdots & a_{1n} \\
\vdots & & \vdots\\
a_{m1} & \cdots & a_{mn}\\
\end{pmatrix} $$

\begin{flushright}
	$m\times n$ Matrix
\end{flushright}


\subsection{Def: Rang}
$$Rg(A) = dim (\textrm{Span(Spaltenvektorraum)}) = dim (\textrm{Span(Zeilenvektorraum)})$$
$F: V \rightarrow W$ linear $$ Rg(F) = Rg(A) = dim (Im(F)) $$

\subsection{Satz: Basismatrix}
V, W endlichdim.\\
Es existieren Basen $\{v_1,\dots, v_n\} \textrm{von} V, \hspace{.5cm} \{w_1,\dots w_m\} \textrm{von} W$ , sodass die Darstellungsmatrix von F die From :

$$\left(\begin{array}{ccccc}
1 & \cdots & \cdots & 0 \\
\vdots & 1 & & \vdots \\
\vdots & & \ddots & \vdots \\
0 & \cdots & \cdots & 1 \\
\end{array} \right)$$

\noindent hat.

\subsection{Bew:}
Sei $U = Ker(F)$ und wähle $\{v_{r+1},\dots, v_n\}$ eine Basis von U. \\
Sei U' ein Komplement von U in $V \rightarrow V = U\oplus U'$\\
Sei $\{v_1,\dots, v_r\}$ eine Basis von U'\\
$\mathcal{B} = \{v_1,\dots , v_n\}$ ist eine Basis von V\\
$Im(F)$ hat $\{F(v_1),\dots, F(v_n)\}$ als Basis\\
$$\sum^{r}_{i = 1} \lambda_i F(vi) = 0$$\\
$$F(\sum^{r}_{i = 1} \lambda_i vi) \Rightarrow \sum^{r}_{i = 1} \lambda_i vi \in U \cap U' = {0}$$\\
$$\Rightarrow \sum^{r}_{i = 1} \lambda_i v_i = 0\qquad \Rightarrow \underline{\lambda_i = \dots = \lambda_r = 0}$$\\
$ \text{Ergänze } \{F(v_1), \dots , F(v_r)\}$ zu einer Basis $ \mathcal{B}' = {w_1, \dots , w_m} \text{Basis von } W$\

$$F(v_1) \dots F(v_r), F(v_{r+1}) \dots F(v_n)$$ 
\nopagebreak
$$\begin{pmatrix}
1   &\cdots  & 0  &   0 & \cdots & 0 \\
\vdots   &\ddots & \vdots  & \vdots && \vdots\\
0 & \cdots  &1 &  0 & \dots & 0 \\
0 & \dots & 0 & 0 & \dots & 0 \\
\vdots  & & \vdots & \vdots & & \vdots\\
0&\cdots &0 & 0&\cdots & 0\\
\end{pmatrix} $$

\begin{flushright}
	$\square$
\end{flushright}

\subsection{Def: Invertierbarkeit}
$A \in M_{n\times n} (K) $ ist \underline{invertierbar}, falls es eine Matrix $B\in M_{n\times n} (K)$ gibt, sodass: $$B\cdot A = A\cdot B = E_n$$\\
$$GL(n,K)= GL_n(K) = \{A\in M_{n\times n} (K) \textrm{invertierbar}\}$$
$GL_n(K)$ ist eine Gruppe\\
$$A\in GL_n(K) \Leftrightarrow rg(A) = n$$
also \underline{invertierbar $\Leftrightarrow$ regulär}\\

\subsection{Bem: Eindeutige Lösung}
Wenn A regulär ist, dann besitzt ein Gleichungssystem

$$A \cdot \begin{pmatrix}
\lambda_1\\
\vdots\\
\lambda_n
\end{pmatrix} 
= \begin{pmatrix}
b_1\\
\vdots\\
b_n\\
\end{pmatrix}$$

eine eindeutige Lösung:

$$A^{-1} \begin{pmatrix}
b_1\\
\vdots\\
b_n\\
\end{pmatrix}$$

\subsection{Bem: Zeilenoperationen}
A ist regulär gdw. A sich durch elementare zeilenoperationen in $E_n$ überführen lässt.\\
$E_{ij} \leftarrow$ Die Matrix die 1 an der Stelle $(i,j)$ hat und 0 sonst.\\
Multiplizieren der i-ten Zeile mit $\lambda$ \\
Addieren $\lambda$ mal j-te Zeile zur i-ten Zeile = $E_n + \lambda E_{ij}$ \\
Vertauschen der i-ten Zeile mit der j-ten Zeile: $E_n - E_{ii} -E_{jj} + E_{ji} + E_{ij}$

$$\begin{array}{cccc}
\begin{pmatrix}
A & \vline & E_n 
\end{pmatrix} &&  \\\\
\rotatebox[origin=c]{180}{$\Lsh$} Zeilenoperationen \rightarrow & &\begin{pmatrix}
E_n & \vline & A^{-1}
\end{pmatrix} \\
\end{array}$$

$$\underbrace{B_n \cdot \dots \cdot B_2 \cdot B_1 \cdot A}_{ A^{-1}} = E_n$$

\noindent \underline{Übergangsmatrizen:}
$dim(V) = n$\\
$\{v_1, \dots , v_n\}$\\
$\{v_j', \dots , v_n\}$\\
$$v'_i = \sum S_{ij} v_{ji}$$
$$S = \begin{pmatrix}
S_{11} & S_{1j} \\
S_{i1} & S_{ij} \\
\end{pmatrix}$$

%vorlesung 3

Vektorraum V/K
$\rightarrow \mathcal{B}$ Basis\\
V ist endlich,falls es eine endliche Basis besitzt.
$$dim(V) = \vert \mathcal{B} \vert$$
$$F: V\to W$$
F: lineare Abbildung
$$\mathcal{B} = \{v_1, \dots , v_n\}$$ Basis von V 
$$\mathcal{B}' = \{w_1, \dots , w_n\}$$ Basis von W 
$$ F(v_1), \dots , F(v_n)$$ 
$$ F(v_j) = \sum _{ij} w_i$$
sind lin. unabh. von der Basis von W
F hat Darstellungsmatrix:

$$\begin{pmatrix}
a_{11} & \dots & a_{1n} \\
\vdots & & \vdots \\
a_{m1} & \dots & a_{mn}
\end{pmatrix}$$

$$V \mapsto V$$
$$v_i \mapsto v_i'$$
$$\mathcal{B} = \{v_1, \dots , v_n\} \qquad \mathcal{B}' = \{w_1,\dots , w_n\} $$

$$S=(s_{ij}) \rightarrow v_j'= \sum s_{ij} v_i \in K$$
Transformationsmatrix von $\mathcal{B'}$ nach $\mathcal{B}$

$$\begin{pmatrix}
s_{11} & \dots & s_{1n} \\
\vdots & & \vdots \\
s_{n1} & \dots & s_{nn}
\end{pmatrix} \begin{pmatrix}
1\\
0\\
\vdots \\
0\\
\end{pmatrix}$$

$S^{-1}$ ist die Transformationsmatrix von $\mathcal{B}$ nach $\mathcal{B'}$

$$V\buildrel F \over \to W$$
\begin{center}
	\begin{tabular}{ccc}
		$\{v_1,\dots,v_n\}$ & $\buildrel A \over \rightarrow$ & $\{w_1,\dots , w_n\}$ \\
		$\uparrow S$ &  & $\uparrow T$ \\
		$\{v_1',\dots, v_n'\}$ & $\buildrel D \over \rightarrow$ & $\{w_1',\dots, w_n'\}$
	\end{tabular}
\end{center}

Wie sieht die Darstellungsmatrix von F bzgl. diesen Matritzen aus?\\
Darstellungsmatrix von F bzgl. $\{v_1',\dots v_n'\} ,\{w_1',\dots ,w_n'\}$ ist $T^{-1} \cdot A \cdot S$.
\subsection{Def: Äquivalenz und Ähnlichkeit} Zwei (mxn)-Matritzen A, A' sind \underline{äquivalent}, falsls es regulare Matritzen $T\in GL_m(K),\quad S \in GL_n(K)$ gibt, sodass: $$A' = T^{-1} \cdot A \cdot S$$\\
A, A' $\in M_{n\times n}(K)$ sind \underline{ähnlich}, falls es $S \in GL_n(K)$ gibt, sodass: $$A' = S^{-1} \cdot A \cdot S$$

\subsection{Bem: Ähnlichkeit}
Ähnlichkeit (Än) auf $M_{n\times n}(K)\Rightarrow$ Äquivalenz (Äq) auf $M_{m\times n}(K)$

\subsection{Def: Determinante} Die Determinante $$det(K^n) \mapsto K$$ multilineare alternierende Abbildung derart, dass $det(e_1,\dots e_n) = 1$.
$$A\in M_{n\times n}(K)$$

$$A  = \Bigg( a_1 \bigg\vert a_2\bigg\vert \dots \bigg\vert a_n\Bigg) $$
$$ det(A) = det(a_1,\dots , a_n)$$
$$A=(a_{ij})$$
$$det(A) = \sum_{\pi \in \underbrace{S_n}_{B_{ij}(\{1,\dots ,n\})}} sign(\pi) \cdot \prod _{i = 1}^n a_{\pi(i)i}$$
$sign(\pi)$ (-1) Anzahl von Fehlständen $\{(ij)\vert i<j \pi(i) >\pi(j)\}$, \\
(-1) Anzahl von Faktoren von $\pi$ als Produkt von Transpositionen\\

%käastchen mit tickz um 22 n-1 1 und n n-1 mit verbindungslinien

$$\begin{pmatrix}
a_{11} & a_{12} & \dots & a_{1n} \\
a_{21} &  &  &  \\
\vdots &  &  & \vdots \\
&  &  &  \\
a_{n1} & \dots &  & a_{nn} 
\end{pmatrix}$$

derline{Eigenschaften}\\
\begin{itemize}
	\item[1)] det($A \cdot B) = \textrm{det}(A) \cdot \textrm{det}(B)$
	\item[2)] $A \textrm{ invertierbar geanu dann wenn det}(A) \neq 0$
	\item[3)] det$(A^{-1}) = \textrm{det}(A)^{-1}$
	\item[4)] det$\equalto{(A^T)}{(a_{ji})} = \textrm{det} (A)$\\$E_n + (-E_n) \textrm{ ist nicht invertierbar.}$
\end{itemize}
\underline{Laplascher Enwicklungssatz}\\
Sei $j_0$ ein Spaltenindex, det$(A) = \sum_{i=1}^n (-1)^{i+j_0} a_{ij_0} \textrm{det}(A_{j_0})$\\

$$\leftrightarrow
\left(\begin{matrix}
a_{11} \\
\vdots \\
a_{n1} \\
\end{matrix}
\right| \left.
\begin{matrix}
&  \dots & a_{1n} \\
&  & \vdots \\
&  \dots & a_{nn}
\end{matrix} \right)$$

Cramersche Regel
$$A \begin{pmatrix}
x_1\\
\vdots\\
x_n
\end{pmatrix}
= \begin{pmatrix}
b_1\\
\vdots\\
b_n
\end{pmatrix}
= b
$$

Falls A regulär ist, dann gibt es eine einzige Lösung zum System:
$$ x_j = \frac{det(a_i,\dots,a_{j-1},b,a_{j+1},\dots , a_n)}{det(A)}$$
\subsection{Def: Darstellungsmatrix}
$det(F) = det(a)$, Darstellungsmatrix von F bzgl. der Basis $\{v_1,\dots , v_n\}$
\subsection{Def: Adjunkte} Sei A $(= a_{ij})$ eine n$\times$n-Matrix\\
definiere die Adjunkte von A,  
$$\textrm{adj}(A) = (\gamma_{ij})$$ wobei $$\gamma_{ij} = (-1)^{i\neq j} \det(A_{ij})$$
\subsection{Bem: Determinante und Adjunkte}
Sei $c_j$ die $j$-te Zeile von $\textrm{adj}(A)$.\\
Sei $a_i$ die $i$-te Spalte von A 
\begin{align*}\overbrace{(\gamma_{j1}, \dots , \gamma_{jn})}^{c_i} \cdot \underbrace{\begin{pmatrix} c_{1i}\\ \vdots\\ a_{ni} \end{pmatrix}}_{a_i}  = \sum_{k=1}^n \gamma_{jk} a_{ki} &= \sum_{k=1}^n (-1)^{j+k} a_{ki} \det(A_{jk}) \\ &= \det(a_1, \dots , a_{j-1}, a_i , a_{j+1}, \dots , a_n) \end{align*}
(Laplacesche Entwicklung) $ = \left\{\begin{array}{ll} det(A) & j=i \\ 0 & j\neq i \end{array}\right.$\\\\
A regulär:
$$adj(A) \cdot A = det(A) \cdot E_n$$
$$\Rightarrow \frac{adj(A)}{det(A)} \cdot A = E_n = A^{-1} \cdot A$$
$$\Rightarrow \frac{adj(A)}{det(A)} = A^{-1} \quad \Rightarrow \quad A \cdot adj(A) = det(A) \cdot E_n$$

\chapter{Lineare Algebra II}
\subsection{Def: Diagonaliserbarkeit}
V Vektorraum\\
$\{U_i\}_{i=1}^k$ Unterräume von V
$$ V = \bigoplus_{i=1}^{k} U_{i} \Leftrightarrow \left\{ \begin{array}{ll}
V = \sum_{i=1}^n U_i & 1\le i \le k \\
& \\
U_i \cap \sum\limits_{\substack{j=1\\ j\neq i}}^k U_j = \{0\} & 
\end{array} \right.$$
\\
Äquivalent dazu, wenn jeder Vektor $v\in V$ sich eindeutig schreiben  lässt als Linearkombination von den Vektoren 

$$\bigcup_{i=1}^k \mathcal{B}_j$$
$\mathcal{B}$ ist Basis von $v_j$

\subsection{Def: Eigenvektor} Ein Endomorphismus 
$$F: V\to V$$ besitzt einen Eigenvektor, falls es $v\in V\setminus\{0\}$ derart gibt, dass $F(v)= \lambda \cdot v $ für ein $ \lambda\in K$\\
Falls $F(v) = \lambda \cdot v$ \\
$\lambda$ ist eindeutig bestimmt von F und v $\Rightarrow \lambda$ ist Eigenwert von F
$$F(v) = \mu \cdot v = \lambda \cdot v \Rightarrow (\lambda\cdot \mu) \cdot v = 0$$
\subsection{Def: Eigenraum} $\lambda \in K\quad F: V\to V \textrm{ Endomorphismus}$
$$V(\lambda) = \{ v\in V\vert F(v) = \lambda \cdot v\}$$
$V(\lambda)$ ist Eigenraum zum Wert $\lambda$\\
und ist ein Unterraum
\subsection{Bem: Eigenwert}$\lambda$ ist ein Eigenwert von F gdw. $\dim(V(\lambda)) \ge 1$.
\subsection{Bem: Eigenwerte} $\lambda_1,\dots , \lambda_k$ verschiedene Eigenwerte von F 
$$\rightarrow V(\lambda_{i}) \cap  \sum_{\substack{j = 1\\j\neq i}}^k V(\lambda_j) = \{0\}$$
\subsection{Def: Diagonalisierbarkeit} V endlich VR / K
$ F: V \to V$ Endomorphismus ( bzw. eine Matrix $A = K^n \to K^n$) ist diagonalisierbar, falls 
$$V = \bigoplus_{i=1}^{k} V(\lambda_i) $$
$\lambda_1, \dots , \lambda_k $ verschiedene Eigenwerte von F\\
Äquivalent dazu, wenn V eine Basis von Eigenvektoren von F besitzt. Äquivalent dazu, wenn F bzgl. einer Basis von V die  \\
Darstellingsmatrix:

$$\begin{pmatrix}
\lambda_1 & & 0 \\
& \ddots & \\
0 & & \lambda_n
\end{pmatrix}$$
hat.

\underline{Für Matrizen:}\\
$A$ ist diagonlaisierbar genau dann wenn es eine reguläre Matrix $S$ gibt, sodass $$S^{-1}AS=
\begin{pmatrix}
\lambda_1 & & 0 \\
& \ddots & \\
0 & & \lambda_n
\end{pmatrix}$$
\subsection{Satz: Zu Eigenwerten}$ A \in \mathcal{M}_{n \times n}(K)\\
\lambda \in K\\
\lambda \textrm{ ist ein Eigenwert von } A \textrm{ genau dann wenn } \\
\lambda E_n - A \textrm{ night regulär ist} \Leftrightarrow \textrm{det}(\lambda \cdot E_n - A) = 0 $

\subsection{Def: Charakteristisches Polynom} Das charakteristische Polynom einer Matrix
$$ A\in \mathcal{M}_{n\times n}(K)$$ ist 
$$\chi_A (T) = det(T\cdot E_n -A)$$
\subsection{Bem: Eigenwerte als Nullstellen} $\lambda$ ist Eigenwert von $A \Leftrightarrow \chi _A (\lambda) = 0 $
\subsection{Bsp:} 
$$A = \begin{pmatrix}
0 & 1\\
-1 & 0\\
\end{pmatrix} \in M_{2\times 2} (\mathbb{R})$$
$$\chi_A (T) = T^2 + 1 = \det(\begin{pmatrix}
T & -1\\
1 & T\\
\end{pmatrix} ((= T \cdot E_2 - A))$$

\subsection{Bem: Spur} $A$ und $A'$ ähnlich $A' = S^{-1} A S $
$$\Rightarrow \chi_A (T) = \chi_{A'} (T)$$
Insbesondere, können wir über das charakteristische Polynom eines Endomorphismus
$$\chi_F(T) \qquad F: V\to V$$
sagen:
$$(a_{ij}) = A \in M_{n\times n} (K) $$
$$\chi_A(T) = T^n + b_{n-1} T^{n-1}+ \dots + b_0$$
wobei $$b_0 = (-1) ^n \det(A)$$
$$b_{n-1} = - Tr(A) = -\sum_{i=1}^{n} a_{i(i oder j)}$$
Tr = Trace = Spur von A

\subsection{Kor: Anzahl der Eigenwerte} dim(V) = n\\
Ein Endomorphismus
$$F: V\to V $$ kann höchstens n viele Eigenwerte besitzen.

\subsection{Kor: Diagonalisierbarkeit} $$F= V \to V $$
mit Eigenwerten $\lambda_1, \dots , \lambda_n \neq 0$ ist diagonalisierbar gdw.
$$n = \sum _{i=1}^k d_i$$
$$ d_i = \dim(V(\lambda_i)) = \{v\in V \vert F(v) = \lambda \cdot v\}$$
$d_i =$ geometrische Vielfachheit von $\lambda_i$

\subsection{Bew: }$\Rightarrow$ \\
F ist diag. gdw. V eine Basis aus Eigenvektoren besitzt, welche aus 
$$n = |B| = \bigcup_{i=1}^{n} |\mathcal{B}_i| \qquad \mathcal{B}_i = \textrm{Basis von} V(\lambda)$$
$\vert \mathcal{B}_i \vert = d_i = \dim(V(\lambda_i))$\\\\
$\Leftarrow$\\
$n = \sum d_i \Rightarrow \dim(\sum_{i=1}^{k} V(\lambda_i)) = n \Rightarrow V = \sum_{i=1}^k V(\lambda_i)$ \\
Da die Eigenräume transversal sind, und ein Vektorraum nur einen UVR der dimension $\dim(V)$ hat, sich selbst.
\begin{flushright}
	$\square$
\end{flushright}


%Vorlesung 4

\subsection{Bem: Eigenwerte und Diagonalisierbarkeit} $F: V \rightarrow V$ Endomorphismus $0 \neq v \in V$ ist ein Eigenvektor für $F$ 
$$F(v) = \lambda v , \lambda \in K$$
$$V(\lambda)=\{v \in V | F(v) = \lambda v \}$$ 
$\lambda$ ist Eigenwert $\Leftrightarrow \dim(V(\lambda)?=1)$\\
$V(\lambda) = \ker(F - \lambda Id_v)$ F ist diagonalisierbar wenn $V$ eine Basis von Eigenvektoren besitzt.\\
$$S^{-1}AS =\begin{pmatrix}
* & & 0 \\
& \ddots & \\
0 & & *
\end{pmatrix}$$
$\chi_A(T) =\det(T \cdot E_n - A)$\\
$A \in \mathcal{M}_{m \times n}(k) = T^n + b_{n-1} T^{n-1} + \dots + b_0$ normiert\\
$\lambda \in K$ ist $\Leftrightarrow \chi_A (\lambda) = 0$ Eigenwert von $A$\\
\subsection{Kor: Geometrische Vielfachheit}
$F: V \rightarrow V$ ist diagonalisierbar $\Leftrightarrow$ geometrische vielfache von Eigenwert $n = \dim(V) = \sum_{i = 1}^k \dim(V(\lambda_i)) = \lambda_1 \dots \lambda_k$ sind die Eigenwerte von $F$\\

\subsection{Bsp:}
$$A= \begin{pmatrix}
0  & 1 \\
0  & 0 \\
\end{pmatrix} 
\in M_{2\times 2} (K) $$

$$\chi_A(T) = \det \begin{pmatrix}
T & -1 \\
0 & T \\
\end{pmatrix}
= T^2$$

0 ist der einzige Eigenwert von A,\\
A ist diagonalisierbar gdw. \\
$$ 2 = \dim ( \ker (A) )$$
$$\begin{pmatrix}
0 & 1\\
0 & 0
\end{pmatrix} \begin{pmatrix}
x\\y
\end{pmatrix} = \begin{pmatrix}
0\\0
\end{pmatrix}$$
$$\Leftrightarrow y = 0 \quad 2 \neq 1 \Rightarrow \tx{A ist nicht diagonalisierbar}$$
$$\Rightarrow \dim(\ker(A)) = 1 $$

\subsection{Def: Algebraische Vielfachheit}
$\dim(V)< \infty \quad F: V\to V$ Endomorphismus\\
$\lambda \in K$ Eigenwert\\
$$\Rightarrow \chi_{F}(\lambda) = 0$$
algebraische Vielfachheit von $\lambda \quad K = \textrm{ord}_\lambda(F)$
$$\chi_F(T) = (T-\lambda)\cdot G(T) \qquad G(T) = 0$$
\subsection{Bem:}
$$\textrm{ord}_\lambda(F) \ge \dim V(\lambda)$$
\subsection{Bew:}
Sei $ \{v_1,\dots,v_k\}$ Basis von $V(\lambda)$ und erweitern sie zu einer Basis \\
$\{v_1,\dots,v_k,v_{k+1, \dots , v_n}\}$ von $V$.\\
Die Darstellungsmatrix von F bzgl. $\mathcal{B}$ 

$$F(v_1) \dots F(v_k), F(v_{k+1}) \dots F(v_n)$$ 
\nopagebreak
$$\begin{pmatrix}
\lambda   &  0 & 0 & \\
0 & \lambda &  0 & & & C_2\\
0 & 0 &  \lambda & & \\
& & &  \ddots &  \\
& 0 & & & & C_1 & \\
& & & & & \\
\end{pmatrix} $$

$ C_1 (\tx{vllt.} C_2) \in Mat_{n - k \times k} (K)$

$$\chi_{F(\tx{vllt} \mid U)}(T) = \det(TE_n-M) = (T-\lambda)^k\cdot \underbrace{\det(TE_{n-k} \cdot C_1)}_{H(\lambda)}$$
$$\Rightarrow k = \textrm{ord}_\lambda(F)$$
(weil es sein könnte, dass $H(\lambda) = 0$)
\begin{flushright}
	$\square$
\end{flushright}

\subsection{Lemma:}
Sei V endlichdim.
$F: V\to V$ Endomorphismus\\
$U \subset V$ F-invarianter Unterraum ( $F(U) \subset U$)\\
$$\tilde{F}: V/U \to V/U \quad \textrm{lineare Abbildung}$$
$$ \bar{v} \mapsto \ov{F(v)}$$

$\tilde{F}$ ist wohldefiniert und ferner
$$\chi_F(T) = \chi_{F \mid U} (T) \cdot \chi_{\tilde{F}} (T)$$
\subsection{Bew:}
$\tilde{F}$ ist wohldefiniert:
$$\bar{v}_1 = \bar{v} \buildrel \tx{zu zeigen} \over \Rightarrow \overline{F(v)} = \tilde{F}(v_1) = \tilde{F}(v) = \overline{F(v)}$$

$$\Rightarrow v_1 = v + \underbrace{(v_1-v)}_{\in U}$$
$$ F(v_1) = F(v) + \underbrace{F(v_1 - v)}_{\in U}$$
$$\Rightarrow \overline{F(v_1)} = \overline{F(v)}$$
Restklassen sind linear und $\tilde{F}$ ist linear $\Rightarrow\; \tilde{F}$ ist linear
Sei $\{u_1,\dots,u_k\}$ eine Basis von U und erweitere sie zu einer Basis $\{u_1,\dots,u_k, v_{k+1}, \dots, v_n\}$ von V.
\subsection{Bem:}
$\{\overline{v_{k+1}},\dots,\overline{v_n}\}$ ist eine Basis von $V/U$ 
\subsection{Bem: Darstellungsmatrizen}
Einfach Darstellungsmatrix von F bzgl. $\mathcal{B}$.

\pagebreak
$$F(u_1) \dots F(u_k), F(v_{k+1}) \dots F(v_n)$$ 
\nopagebreak

$$\begin{matrix}
u & \rightarrow\\
\vdots \\
u_k & \rightarrow\\
v_{k+1} & \rightarrow\\
\vdots \\
v_n & \rightarrow
\end{matrix}
\begin{pmatrix}
& & & \\
& A & & & & C_2\\
& & & & \\
0 & \dots & 0 & &  \\
\vdots & & \vdots & & & C_1 & \\
0 & \dots & 0 & & & \\
\end{pmatrix} $$

\begin{align*}
\chi_F(T) &= \det(T\cdot E_n - H) \\
&= \det\bigg( T \begin{pmatrix}
E_k & 0 \\
0 & E_{n-k}
\end{pmatrix} - \begin{pmatrix}
A & C_2 \\
0 & C_1 
\end{pmatrix} \bigg)\\
&= \det\begin{pmatrix}
TE_k-A & -C_2 \\
0 & TE_{n-k} - C_1
\end{pmatrix} \\
&= \det(TE_k-A) \cdot \det(TE_{n-k} - C_1)
\end{align*}

A ist die Darstellungsmatrix von $ F \mid U $ bzgl. $\{u_1, \dots, u_k\}$ 
$$ \Rightarrow \det (TE_k -A) = \chi_{F 1 \mid U}(T)$$
$C_j$ ist die Darstellungsmatrix von $\tilde{F}$ bzgl. $\{\overline{v_{k+1}}, \dots, \overline{v_n}\}$\\
$$ \Rightarrow \det (TE_{n-k}-A) = \chi_{\tilde{F}}(T)$$
\subsection{Satz: Diagonalisierbarkeit}
$\dim(V) < \infty \quad K$ Körper \\
$ F: V\to V$ Endomorphismus\\
F ist diagonalisierbar gdw. $\chi_{F}(T)$ in linearfaktoren zerfällt
$$\chi_{F}(T) = (T-\lambda_1)^k \dots (T-\lambda_r)^{k_r}$$
$\lambda_1, \dots, \lambda_r = $ Nullstellen (verschwinden)
und für jeden Faktor $T-\lambda_i$ gilt:
$$\textrm{ord}_{\lambda_i}(F) = \dim(V(\lambda_i))$$

\subsection{Bew:}
$\Rightarrow$ Sei $\mathcal{B}= \{v_1,\dots,v_n\}$ eine Basis von Eigenvektoren.\\
Seien $\lambda_1,\dots,\lambda_r$ die verschwindenden Eigenwerte.
$$v_1,\dots,v_{\alpha_1} \in V(\lambda_2)$$
$$v_{\alpha + 1}, \dots , v_{\alpha, \alpha_2} \in V(\lambda_2)$$

$v_{d_1}+ \dots + v_{d_{r-1}}, \dots , \ub{v_{d_1}+ \dots + d_r}_{n}$
$d_i = \dim(V(\lambda_i))$
Die Darstellungsmatrix von F bzgl. $\mathcal{B}$
$$F(v_1) \dots F(v_d), F(v_{d+1}) \dots F(v_n)$$ 
$$\begin{pmatrix}
\lambda_1\\
& \ddots\\
& & \lambda_1\\
& & & \lambda_2\\
& & & &\ddots\\
& & & & & \lambda_2\\
& & & & & &\ddots\\
& & & & & & & \lambda_r\\
& & & & & & & & \ddots \\
& & & & & & & & & \lambda_r\\
\end{pmatrix}\begin{matrix}
\\
\left\}d_1 \right.\\
\\
\\
\left\}d_2 \right.\\
\\
\\
\\
\left\}d_r \right.\\
\\
\end{matrix}$$

Wobei $d_i$ viele $ \lambda_i $ auf der Diagonalen liegen

$$\chi_F(T) = \det(T E_n - A) = (T-\lambda_1)^{d_1} \ub{\dots (T-\lambda_r)^{d_r}}_{a(T)}$$
$$a_q(\lambda_1) \neq 0 \quad (\lambda\neq \lambda_i , i \neq 1)$$
$ \Leftarrow$
$$\chi_F(T) = (T-\lambda_1)^{d_1} \dots (T-\lambda_r)^{d_r}$$
$(\tx{Grad}(\chi_F) = n)$
$$d_i = \dim(V(\lambda_i)$$
F ist diagonalisierbar $\Leftrightarrow n = \dim(V) = \sum d_i$


\subsection{Def: Diag. und ähnlichkeit} Eine Matrix $A \in M_{n\times n} (K)$ ist \underline{trigonalisierbar}, wenn sie \underline{ähnlich zu einer} \underline{oberen Dreiecksmatrix} ist 

$$\begin{pmatrix}
a_{11} & \dots & a_{1n} \\
& \ddots & \vdots\\
0 & & a_{nn}
\end{pmatrix}$$


% Vorlesung 5 oder so

endlichdim VR / K
$F: V \to V $ Endomorphismus

$ A $ Dartellungsmatrix bzw. $B$
$$ \chi_F (T) = \det (T\cdot E_n - A)$$
normiert
$$\chi_F(\lambda) = 0 \leftrightarrow \lambda \textrm{ Eigenwerte von F } \Leftrightarrow \exists v\in V\setminus \{0\} F(v) = \lambda \cdot v$$
$U \subset V$ Untervektorraum F-invariant ($F(U) \subset U$)
$$\chi_F = \chi_{F \mid U} \cdot \chi_{\tilde{F}}$$
$$ \tilde{F}: V/U \to V/U$$
$$\bar{v} \mapsto \ov{F(v)}$$
F ist dia/tri ? gonalisierbar $\Leftrightarrow \ \chi_A$ in linearfaktoren zerfällt und $\forall \lambda_1, \dots \lambda_k$ Eigenwerte
$$\dim(V(\chi)) = \textrm{ord}_{\lambda_i}(\chi_F)$$
\subsection{Def:}
$F: V\to V $\\
F trigonalisierbar ist, falls es eine Darstellungsmatrix von F gibt, welche in oberer Dreiecksform ist.

$$\begin{pmatrix}
a_{11} & \dots & a_{1n} \\
& \ddots & \vdots\\
0 & & a_{nn}
\end{pmatrix}$$

\subsection{Satz:}
$F: V\to V $
ist trigonalisierbar, gdw $\chi_F$ in Linearfaktoren zerfällt 
$$\chi_F(T) = (T-\lambda_1) \dots (T-\lambda_n)$$
(eventuell mit wiederholungen)
\subsection{Kor:}
Jeder Endomorphismus eines endlichdim VR über einem alg. abg. Körper (z.B.$\mathbb{C}$) ist trigonalisierbar.
\subsection{Bew: (Satz)}
$\Rightarrow$ \\
F ist Darstellungsmatrix 

$$\begin{pmatrix}
a_{11} & & & *\\
& a_{22} & & \\
& & \ddots & \\
0 & & & a_{nn}
\end{pmatrix}$$

$$\chi_F(T) = \det(T\cdot E_n - A) = \prod^n_{i=1} (T-a_{ii})$$
$\Leftarrow$\\ Induktion über $n = \dim(V)$\\
$n=1 \rightarrow$ Jede $1\times 1$ Matrix ist in oberer Dreiecksform $\rightarrow a_n$ !\\
$n\ge 2$
$$\chi_F(T) = (T-\lambda_1) \dots (T-\lambda_n)$$
$\lambda_1$ ist ein Eigenwert 
$$\rightarrow \exists v_1 \in V\setminus \{0\} \quad F(v_1) = \lambda_1 \cdot v_1$$
$$U = \textrm{span}(v_1) \subset V$$
ist F-invariant
$$\equalto{\chi_F(T)}{(T-\lambda_1)\prod^n_{i=2}(T-\lambda_i)} = \equalto{\chi_{F \mid U}}{(T-\lambda_1)} \cdot \chi_{\tilde{F}}$$
$K[T]$ ist ein Integritäsbereich
$$\Rightarrow \chi_{\tilde{F}}(T) = \prod^n_{i=2}(T-\lambda_i)$$
$$\dim(V/U) < \dim(V)$$
$\Downarrow$ I.A.
Es gibt eine Basis $\bar{v_2}, \dots , \bar{v_n}$ von$ V/U$ derart, dass $\tilde{F}$ bzgl. dieser Basis Darstellungsmatrix

$$\begin{pmatrix}
\lambda_2 & & * \\
& \ddots & \\
0 & & \lambda_1
\end{pmatrix} = (\mu_{ij})$$
\subsection{Beh:} Seien $v_i \in V \quad \overline{v_i} = \overline{v_i}, 2 \le i \le n$\\
$\{v_1,v_2,\dots,v_n\}$ it eine Basis von V

\subsection{Bew: } (Übungsaufgabe)
\subsection{Frage:}
Wie sieht die Darstellungsmatrix von $F$ bezüglich $\{v_1, \dots, v_n  \}$ aus?
$$\tilde{F}(\bar{v_j})=\overline{F(v_j)}$$
$$\Rightarrow \sum_{2\leq i\leq j}\mu_{ij}\bar{v_i} = \overline{\sum_{2\leq i\leq j}\mu_{ij}v_i}$$
$$\Rightarrow \mu_{ij} \in K | _{F(v_j)} = \mu_{2j}v_1 + \sum_{2\leq i \leq j}\mu_{ij}v_j = \sum_{1 \leq i \leq j}\mu_{ij}v_j$$
$$F(v), F(v_2) \dots F(v_n) $$
$$\begin{pmatrix}
\lambda_{1} & \lambda_{12} & \cdots &  *\\ 
& \lambda_2 &  &  \vdots\\ 
0 &  & \ddots &  \\ 
0 & 0 &  & \lambda_n
\end{pmatrix} $$

\subsection{Lemma:} V endlichdim / K \\
$v\in V \setminus \{0\} \quad \exists r \in \mathbb{N}$

$$\equalto{F^r(v)}{ \underbrace{F \circ F \circ \dots \circ F}_{r \textrm{Mal}}} = \sum ^{r-1}_{i=0} \underbrace{a_i}_{\substack{\textrm{eindeutig} \\ \textrm{bestimmt}}} F^i(v)$$

Insb. ist
$$U = \textrm{Span}(v, F(v), \dots , F^{n-1}(v))$$
ist F-invariant, hat Basis $\{v_1, \dots, F^{n-1}(v)\}$\\
$F \mid U$ hat Darstellungmatrix

$$\begin{pmatrix}
0 & 0 & 0 & \dots & a_0 \\
1 & 0 & 0 & \dots & a_1 \\
& 1 & 0 & \dots & a_2 \\
&   &   \ddots & \ddots & \vdots \\
0  &   &   &   & a_{r-1}
\end{pmatrix} $$

$$\chi_{F \mid U}(T) = T^r - a_{r-1} T^{r-1} \dots - a_{0}$$
\subsection{Bew:}
$ n = \dim(V)$
$v, F(v), \dots , F^r(v)$ lin. unabh.\\
Sei $r>0$ kleineste rat. Zahl, sodass:
$v,F(v), \dots , F^r(v)$ lin. abh.

$v, F(v) , \dots , f^{r-1}(v) $\\
lin. unabh. (aus der Minimalität von r)\\
$\Downarrow$ Austauschprinzip:
$$F^r(v) = \sum^{r-1}_{i=0} \ub{a_i}_{\substack{\textrm{eindeutig} \\ \textrm{bestimmt}}} F^i(v)$$
$$ U = \tx{Span} (v, F(v), \dots , F^{r-1}(v)) \quad \tx{ ist F-invariant}$$
$$v\in U \quad v = \sum^{r-1}_{i=0} \mu_i F^i(v)$$
\begin{align*}
F(v) & = \sum^{r-1}_{i=0} \mu_i F^{i+1}(v) \\
& = \ub{\sum^{r-2}_{i=0} \mu_i F^{i+1}(v)}_{\in U} + \ub{\mu_{r-1} \equaltoup{\ob{F^r(v)}}{\sum^{r-1}_{i=0} a_i F^i(v)}}_{\in U }
\end{align*}

$\{v, F(v), \dots , F^{r-1}(v)\}$ ist eine Basis von U.

$$ F(v) , \dots , \ob{F(F^{r-1}(v))}^{F^r(v)}$$
$$A = \begin{pmatrix}
0 & 0 & 0 & \dots & a_0 \\
1 & 0 & 0 &  & a_1 \\
& 1 & 0 & & \vdots \\
&   & \ddots & \ddots & \\
0 &   &   &   & a_{r-1}
\end{pmatrix} $$

\begin{align*}
\chi_{F\mid U } (T) &= \det (T\cdot E_n -A)\\
&= \det\begin{pmatrix}
T & 0 & & \dots  & a_0 \\
-1 & T & & 0\\
& -1 & T  & & \vdots \\
& & & \ddots \\
0 & & & & T- a_{r-1}
\end{pmatrix} \\
\end{align*}
(Laplacescher Entwicklungssatz nach der r-ten Spalte)\\

\begin{align*}&= (-1)^{r+1} (-a_0) \det \begin{pmatrix}
-1 & T \\
& -1 & T \\
& & & \ddots \\
& & & & (-1) ^T
\end{pmatrix} + \\
& \quad + (-1)^{r+2} (-a_1) \det \begin{pmatrix}
-1 & T \\
& -1 & T \\
& & & \ddots \\
& & & & (-1) ^T
\end{pmatrix} + \dots + \\
& \quad + (-1)^{2r} (T-a_{r-1}) \det \begin{pmatrix}
-1 & T \\
& -1 & T \\
& & & \ddots \\
& & & & T
\end{pmatrix}\\
&= (-1)^r a_0 (-1)^{r-1} a_1 + (-1)^{r+1} a_1 T (-1)^{r-1} + \dots + (-1)^{2r} (T-a_{r-1}) T^{r-1}\\
& = -a_0 - a_1 T - \dots - a_{r-1} T^{r-1} + T^r 
\end{align*}

\noindent
\subsection*{\underline{Notation}}
$$\equalto{P(T)}{\sum^m_{i=0} a_i T^i} \in K[T]$$
$\enph$\\
$P(F): V\to V $\\
$ v\mapsto \sum^m_{i=0} a_i F^i(v)$

Mit dieser Notation haben wir, das im vorherigen Lemma
$$\chi_{F \mid U} (v) = F^r(v) - \sum^{r-1}_{i=0} a_i F^i(v) = 0$$
\subsection{Satz: (Calay - Hamilton)}
$\enph $ endlichdim.
$\chi_F(F)$ ist der 0 Endomorphismus auf V
\subsection{Bew:}
Zu Zeigen:
$\forall v\in V : \chi_F(F)(v) = 0$\\
\underline{$v=0$} $\rightarrow$ ok\\
sonst $v\neq 0 \rightarrow \exists r \in \mathbb{N}$
$$ U = \textrm{Span}(v,F(v), \dots, F^{r-1}(v))$$
ist $F$ in $V$
U F-invariant
$$\Rightarrow \chi_F = \chi_{F \mid U} \cdot \chi_{\tilde{F}} = \chi_{\tilde{F}} \cdot \chi_{F \mid U}$$
$\tilde{F}: V/U \to V/U$\\
$\bar{w} \mapsto \bar{F(w)}$\\
\underline{Aufgabe}\\
$R(T) = P \cdot C$ allg. Polynom
$\Rightarrow R(F) = P \cdot (G(F))$ als Endomorphismen
$$\chi_F(F)(v) = \equalto{\underbrace{\chi_{\tilde{F}} \cdot (\equalto{\underbrace{\chi_{F \mid U}(v)}}{0})}}{0} $$

\subsection{Kor:}
$A \in \mathcal{M}_{n\times n}(K)$
$$\chi_A(T) = T^n + \sum^{n-1}_{i=0} b_i T^i$$
$$\Rightarrow A^n + \sum^{n-1}_{i=0} b_i \cdot A^i = 0$$
da $n\times n$ Matrix
\subsection{Satz:}
$\enph \quad V$ endlichdim. /K\\
Dann existiert genau ein normierter Polynom kleinsten Grades $m_F$ derart, dass $\forall P\in K[T]$\\
$$m_F | _P \Leftrightarrow P(T) = 0  \quad \textrm{als Endomorphismus von F}$$
Insb. gilt $m_F(F) = 0$\\
Das Polynom $m_F (= \mu _F)$ heißt das minimalpolynom von F.\\																										

%Vorlesung 6




\subsection{Satz:}
$\enph$ V eindlich dim. und trigonalisierbar
falls es eine Basis $\mathcal{B}$ von V gibt, sodass F Darstellungsmatrix der Form
$$\begin{pmatrix}
* && * \\
& \ddots \\
0 && *
\end{pmatrix}$$
Falls $\chi_F(T)$ in Linearfaktoren zerfällt dann ist F trigonalisierbar (Insb. falls K alg. abg. ist z.B. $ \mathbb{C}$)
$$F^0 = Id_{iv}$$
$$ F^i = \underbrace{F \circ \dots \circ F}_{\tx{i mal}}$$
$P\in k[T] \quad P = \sum_{i=0}^mT^i$
$$P(F): \sum a_i F^i: V\to V, v \mapsto \sum a_i F^i(v)$$
\underline{Aufgabe:}\\
komposition des Endomorphismus ... Produkt im Polynom
$$P(F) \circ f(F) = (P \cdot G)(F)$$

\subsection{Satz: (Caley-Hamilton)}
$$\chi_F(F) = 0_{iv} $$
als Endomorphismus
\subsection{Kor:} $A\in \mathcal{M}_{n\times x} (K)$
$$\chi_A(T) = T^n + \sum_{i=0}^{n-1} b_i T^i$$
$$\Rightarrow A^n + b_{n-1} A^{n-1} + \dots  + b_0 En = 0$$
\subsection{Satz:}
$ V \tx{endlichdim} / K \enph$
Dann existiert genau ein normiertes Polynom $m_F(T)$ derart, dass $\forall P\in K[T]$:
$$ m_F|_P \Leftrightarrow P(F) = 0$$
\subsection{Def:}
Das Polynom $m_F(T)$ heißt das Minimalpolynom von F. Insb. $m_F(F) = 0$
\subsection{Bew:} Sei $$\mathcal{F} = \{ P\in K[T] \tx{ normiert } |P(F) = 0 \tx{ als Endomorphismus}\}$$
Caley.Hamilton
$$\chi_F(T) \in \mathcal{F} \neq \varnothing$$
Sei $m_F(T) \in \mathcal{F}$ Polynom kleinsten Grades.\\
Zu Zeigen: $\forall P\in K[T]$
$$m_F|_P \Leftrightarrow P(F) = 0$$
$\Rightarrow$\\
$$m_F|_P \Leftrightarrow \exists G \in K[T] \quad P = G \cdot m_F$$
$$ P(F) = G(F) \circ \ub{m_F(F)}_{= 0 (m_F(F) \in \mathcal{F}} = 0$$
$\Leftarrow$\\
Sei $P \in K[T] | P(F) = 0$\\
Division mit Rest $\rightarrow \exists G \buildrel r \over \in K[T]$
$ P = G \cdot m_F + r \quad \grad(r) < \grad(m_F)$
$$0=P(F) = \equalto{\ub{G(F) \circ \equalto{\ub{m_F(F)}}{_{0}}}}{_{ 0}} + r(F)$$
$\Rightarrow r(F) = 0$ als Endomorphismus\\
$ \Rightarrow r = 0$ ( Sonst $\frac{1}{a_{(\grad(r))}} \cdot r(T) \in \mathcal{F})$
\underline{Eindeutigkeit}\\
Angenommen $m_F'$ ist normiert. wäre auch so:
$$m_F'|_P \Leftrightarrow P(F) = 0 \forall P \in K[T]$$
$$\Rightarrow m_F|_{m_F'} \quad \& \quad m_F'|_{m_F}$$ 

$\left. \begin{array}{ll}
m_F = & Q \cdot m_F' \\
m_F' = & H \cdot m_F
\end{array} \right\} \rightarrow Q,H \tx{ sind beide \underline{normiert}} $\\[10pt]
Zu zeigen: \underline{$Q=H=1$}
$$m_F = Q \cdot m_F' = Q \cdot H \cdot m_F$$
K[T] \underline{Integritätsbereich}
$$ \Rightarrow 1 = Q \cdot H$$
$\grad(G\cdot H) = \grad (1) = 0$\\
$\grad(G) + \grad(H) \Rightarrow G,H \in K $ und normiert\\
(als Polynom) $\Rightarrow Q = H = 1$

\subsubsection{Bsp:}
$$A = \begin{pmatrix}
0 & 1 \\
0 & 0
\end{pmatrix}$$
$m_A? $

$$\chi_A(T) = T^2 = \det\begin{pmatrix}
T & -1 \\
0 & T
\end{pmatrix}$$
$$m_A|_{T^2} \Rightarrow m_A = \left\{ \begin{array}{ll}
T \\ \\ T^2
\end{array} \right.$$
$$m_A = 0 \rightarrow m_A(T) \neq T (A\neq 0)$$
$$A^2 = \begin{pmatrix}
0 & 0\\
0 & 0
\end{pmatrix}$$
$\rightarrow m_A(T) = T^2$
\subsection{Lemma:}
Gegeben $\enph$ V endlichdim / K , dann haben $\chi_F$ und $m_F$ dieselben Nullstellenn in K.
\subsection{Bew:}
$$m_F|_{\chi_F} \Rightarrow \chi_F = G \cdot m_F$$
$\forall \lambda \in K$ , falls $m_F(\lambda) = 0$
$$ \Rightarrow \chi_F(\lambda) = 0$$

Sei $\lambda \in K | \chi_F(\lambda) = 0$\\
$\Rightarrow \lambda$ ist ein Eigenwert von F \\
$\Rightarrow \exists v \in V\setminus \{0\} | F(v) = \lambda\cdot v$\\
Sei $m_F(T) = T^d + \sum_{i=0}^{d-1} c_i T^i$\\
\begin{align*}
0 &= m_F(F) (v)\\
&= (F^d + \sum c_i F^i)(v)\\
&= F^d(v) + \sum c_i F^i(v) \\
&= \lambda^d v + \sum c_i \lambda^i \cdot v\\
&= \equalto{\ub{(\lambda^d + \sum c_i \lambda^i)}}{_{m_F(\lambda)}} \cdot v
\end{align*}
$v\neq 0 \quad m_F(\lambda) = 0$

\subsection{Satz:}
$\enph$ V endlichdim. / K ist diagonalisierbar gdw $m_F$ in lauter verschiedene Linearfaktoren zerfällt.

$$m_F(T) = \prod_{i=1}^{k} (T-\lambda_i) : \lambda_i \neq \lambda_j \tx{ für } i\neq j $$
Die $\lambda_i$ sind die Eigenwerte von F
\subsection{Bew:}
$ \boxed{\Rightarrow} $\\
Sei F diagonalisierbar. Dann gilt 
$$\chi_F(T) = \prod_{i=1}^{k} (T-\lambda_i)^{d_i}$$
wo die $ \lambda_i $ Eigenwerte von F sind
$$ \tilde{F} = \begin{pmatrix}
\lambda_1 \\
& \ddots\\
& & \lambda_i \\
& & & \ddots\\
& & & & \lambda_k
\end{pmatrix} \qquad \chi_F(T) = \det(T\cdot E_n - \tilde{F})$$

Außerdem ist $ V = \bigoplus\limits_{i=1}^{k} V\lambda_i $\\
Sei $ v\in V $beliebig.  Dann gilt $ V = \sum_{i=1}^{k} V_i $, wobei $ v_i \in V(\lambda_i) $\\
Setze 
$$ p(T) = \prod_{i=1}^{k}(T-\lambda_i) \quad (\buildrel ? \over = m_F(T)) $$
\begin{align*}
\Rightarrow p(F) v &= \prod_{i=1}^{k} (F-\lambda_i) v \\
&= \prod_{i=1}^{k}(F-\lambda_i) \bigg (\sum_{j=1}^{k} v_j \bigg )\\
&= \sum_{j=1}^{k}\bigg( \prod_{i=1}^{k} (F-\lambda_i) v_j \bigg)\\
&= \sum_{j=1}^{k} (F-\lambda_1 \cdot Id) \circ \dots \circ (F-\lambda_i \cdot Id) \circ \dots\\ &\quad \quad \quad \dots \circ (F-\lambda_{i+1} \cdot Id) \circ \dots \circ (F-\lambda_j \cdot Id)(v_j)\\
&= 0
\end{align*}
da $ v_j \in V(\lambda_i) $.\\
v war beliebig $ \Rightarrow P(F) = 0 \in \tx{end}(V) $\\
also $  m_F|_p \Rightarrow P(T) = Q(T) \cdot m_F(T)$\\
Aber $ m_F(T) = \prod_{i=1}^{k}(T-\lambda_i)^{s_i} \quad s_i \ge 1 $\\
$$ P(T) = \prod_{i=1}^{k}(T-\lambda_i) = Q(T) \prod_{i=1}^{k} (T-\lambda_i)^{s_i}$$
$\Rightarrow s_i = 1 \tx{ für } i = 1, \dots , k$\\
sonst wäre Grad (rechte Seite) > Grad(linke Seite)\\
$$ \Rightarrow m_F(T) = \prod_{i=1}^{k} (T-\lambda_i)$$
Dies war zu zeigen ($ \Rightarrow $).\\\\
$ \boxed{\Leftarrow} $\\
Sei $  m_F(T) = (T-\lambda_1)\circ \dots \circ(T-\lambda_k) \qquad (\lambda_i \neq \lambda_j \tx{ für } i \neq j ) $\\
zu zeigen ist F ist diagonalisierbar ($ \lambda_1 , \dots , \lambda_k $ sind die Eigenwerte von F)\\
Es genügt zu zeigen,dass $ V = \bigoplus\limits_{i} V(\lambda_i) $\\
Beweis erfolgt durch Induktion über $ \dim(V) $\\
\underline{Sonderfälle:}
\begin{itemize}
	\item[(1)] Sei $ \dim(V) = 1 \Rightarrow m_F(T) = (T-\lambda_1) \Rightarrow \lambda_1 $ ist Eigenwert von F\\
	$$\exists v \in V: \ F v = \lambda_1 v \Rightarrow V = \spa\{v\} = V(\lambda_1)$$
	$ (v\neq 0) $
	\item[(2)] Sei nun $ \dim(V) = \ub{n}_{\in \mathbb{N}} \ge 2 $ und $  k = 1 \Rightarrow m_F(T) = (T-\lambda_1)$\\
	Nach definition von $ m_F $ gilt: $ m_F(F) \Rightarrow F-\lambda_1 = 0 \in \tx{end}(V) $\\
	für jeden Vektor $ v\in V $ gilt: $ (E-\lambda_1) v = 0 \Rightarrow F v = \lambda_1 v $\\
	$\Rightarrow V = V(\lambda_1)$	
\end{itemize}
Induktionsannahme (für den allgemeinen Fall): \\
Unser Satz gilt für $ \dim(V) < n (n\in \mathbb{N}) \tx{ fest} $\\
Also sei $ \dim(V) = n \ge 2 \ \&\  k \ge 2 $ \\
$(*)$ \underline{Beh:} \\	
$$ V = \ub{\ker(F-\lambda_i \cdot I_d)}_{\substack{V(\lambda_i)  \\ (\dim(V(\lambda_i)) \ge 1 !)}} \oplus \ub{\im(F-\lambda_i \cdot I_d)}_{\substack{\tx{zu zeigen:} \\ (= V(\lambda_2) \oplus \dots \oplus V(\lambda_k))}} $$
\underline{Bew:} Sei $ R(T) := \prod_{i=2}^{k}(T-\lambda_i) \quad \bigg[ \Rightarrow m_f(T) = R(T) (T-\lambda_1)\bigg] $\\
Division mit Rest: $ \exists Q(T) \& r\in K $, sodass:
$R(T) = Q(T) (T-\lambda1) + r $\\
$0 \neq R(\lambda_1) = 0 + r \Rightarrow r \neq 0 \in K$\\
Sei $ v\in V $ beliebig $ R(F)^v = Q(F) (F-\lambda_1) v + r \cdot v $\\
$$\Rightarrow v = \ub{\frac{1}{r} R(F) r \neq (F-\lambda_1) }_{\in \ \ker(F-\lambda_1)} \circ \ub{(-\frac{1}{r} Q(F) r)}_{\in \ \im(F-\Lambda_1)} $$
$$(F-\lambda_1) R(F)v = \ub{m_F(F)}_{0}v = 0 \quad \Rightarrow V = \ker(F-\lambda_1) + \im(F-\lambda_1)$$
Wir wollen zeigen, dass die Summe direkt ist $ \Leftrightarrow  \ker(F-\lambda_1) \cap \im(F-\lambda_1) = \{0\}$\\
Sei $ v \in \ker(F-\lambda_1) \cap \im(F-\lambda_1) \quad \Rightarrow v = (F-\lambda_1)w \quad w\in V $\\
$R(F)v = Q(F) (F-\lambda\cdot I_d) v + r\cdot v $\\
$$\ub{R(F)\circ (F-\lambda_1)}_{m_F(F) = 0} w = \ub{Q(F)(F-\lambda_1 \cdot I_d)}_{= 0} v + r \cdot v \quad \Rightarrow r \cdot v = 0 \Rightarrow \equalto{v}{0} \in V$$
$\Rightarrow V = \ub{\ker(F-\lambda_1 I_d)}_{V(\lambda_1)} \oplus \im(F-\lambda_1 I_d ) $
\begin{flushright}
	$ \square (*)$
\end{flushright}
Beweis durch Induktion $ \cdot $ Fortsetzung $ \cdot \ \dim(V) = n \ge 2 , k \ge 2$\\
Setzte $ W = \im(F-\lambda_1) $ (W ist UVR von V mit $ \dim(W) \subset \dim(V) $)\\
$(**)$ \underline{Beh:} W ist $ F $-invariant\\
\underline{Bew:} Sei $ v\in W $ beliebig $ \Rightarrow v = (F-\lambda_1) w \quad w \in W$\\
$\Rightarrow F(v) = F \circ (F-\lambda_1) w = (F-\lambda_1) \circ \ub{F(w)}_{\in V} \in W$
\begin{flushright}
	$ \square (**)$
\end{flushright}
$(***)$ \underline{Beh:} (setzte $ F' = F|_W \in \tx{end}(w) $)\\
$$m_F(T) = \prod_{i=2}^{k} (T-\lambda_i) \ (:= R(T))$$
\underline{Bew:} Sei $v\in W$ beliebig $ \Rightarrow  v= (F-\lambda_1)w \quad w \in V$\\
$R(F')(v) = R(F) \circ \ub{(F-\lambda_1)}_{m_F(F) = 0}w = 0 $\\
$\Rightarrow R(F') = 0 \in \tx{end} (W)$\\$ m_{F'}(T) |_{R(T)} \Rightarrow R(T) = H(T) \cdot m_{F'}(T)$\\
Es gilt auch 
$$ 0 = \ub{ m_{F'}(F')}_{=0} \circ (F-\lambda_1))\ub{(V)}_{v \in V \tx{ beliebig}}$$
$$m_{F'}(F') \circ (F-\lambda_1) = 0 \in \tx{end}(V)$$
$$m_F(T) |_{m_{F'}(T) \cdot (T-\lambda_1)}$$
$$ \Rightarrow m_{F'} (T) \cdot (T-\lambda_1) = Q(T) \cdot m_F(T) = Q(T) \cdot R(T) \cdot (T-\lambda_1)$$
$$ \Rightarrow m_{F'}(T) = Q(T) \cdot R(T) = Q(T) \cdot H(T) \cdot m_{F'}(T) \Rightarrow Q(T)  = H(T) = 1$$
$\Rightarrow R(T) = m_{F'}(T)$
$\Rightarrow$ Induktionsannahme: $ W = \bigoplus\limits_{i=1}^{k} W(\lambda_i)  $ Eigenraum von $ F' $ zu $  \lambda_i $ \\
$\Rightarrow V = V(\lambda_1) \bigoplus\limits_{i=2}^{k} W(\lambda_i)$ zu zeigen: $ W(\lambda_i)  = V(\lambda:i) $\\
dann gilt: $ V = \bigoplus\limits_{i=1}^{k} V(\lambda_i) $\\
Es gilt: $  W(\lambda_i) = \{w\in W: F'(w) = \lambda_i w\}  \subseteq \{v \in V : F(v) = \lambda_i \cdot v\} := V(\lambda_i)$ 
Sei $  v\in \ub{V(\lambda_i)}_{(i>1)} \Rightarrow F(v) = \lambda_i v$ Setze: $ w = \frac{1}{(\lambda_i - \lambda_1)} \cdot v \in V(\lambda_i) $\\
$ \Rightarrow F(w) - \lambda_i w = v \Rightarrow v \in \im(F-\lambda_i) = W $\\
$ \Rightarrow v \in \bigoplus\limits _{j=2}^{k} W(\lambda_j)  \Rightarrow v \in W(\lambda_{ i oder 1 })$ Warum?\\
$ v = \sum \alpha_j w_j $ für $ w_j \in W(\lambda_i) \quad \alpha_j \in K \quad (\subseteq V(\lambda_j))$ Also nur $ \alpha_i \neq 0 $ \\
Aber $ + V(\lambda_j) = \oplus V(\lambda_j) $ oder $ v \in V(\lambda_i)) \Rightarrow v = \alpha_i w_i \in W(\lambda_i)$	


\begin{flushright}
	$ \square $
\end{flushright}

\subsection{Bem:} $A\in \mathcal{M}_{n \times n} (\mathbb{R})$
$ A^2 = E_n$
$$\Rightarrow A \tx{ ist diagonalisierbar}$$
da $A^2 = E_n \Rightarrow (T^2-1) (A) = 0$\\
$m_F | _{T^2-1}$
$$ m_F = \left\{ \begin{array}{ll}
T^2-1 & = (T-1)(T+1) \\
T-1 \\
T+1
\end{array} \right.$$

\chapter{Die Jordansche Normalenform}
\subsection{Lemma:}
$\enph \quad U \subset V$ Untervektorraum:\\
Dann ist U F-invariant gdw
$ U(F-\lambda Id)$- invariant ist für $ \forall \lambda \in K$
\subsection{Bsp:}
$\enph \ V$ endlichdim. derart, dass 
$F^m = 0$ als Endomorphismus) und $m>0$ minimal (,, F ist nilpotent``)\\
$ F^{m-1} \neq 0 \rightarrow \exists v \in V | F^{n-1}(v) \neq 0$\
$\{v,F(v),\dots,F^{n-1}(v)\}$ ist linear unabh.\\
Ergänze das Systme zu einer Basis $\mathcal{B}$ von V: $\{v, \dots , F^{n-1}(v), \dots , v_n\}$
$$F^{n-1}(v) F^{n-2}(v) \dots, v, v_{mn} , \dots , v_1$$
$$\begin{pmatrix}
0 & 1 & 0 & \\
0 & 0 & 1 & & & * & & \\
0 & 0 & 0 & \\
&&& \ddots\\ 
\\
& 0 & & & & *
\end{pmatrix}$$

\subsection{Def: Hauptraum}
Sei $\lambda \in K$ ein Eigenwert von $\enph \ V$ endlichdim.
$$ V(\lambda) = \ker ( F-\lambda )$$
Eigenraum von F bzgl. $\lambda$
$$ \ker(F-\lambda) \subset \ker(F-\lambda)^2 \subset \ker(F-\lambda)^3 \subset \dots$$
$$V_\lambda = \bigcup_{n\in \mathbb{N}} \ker(F-\lambda)^n$$
ist ein Hauptraum von F bzgl. $\lambda$
\subsection{Bem:}
$V_\lambda$ ist ein Unterraum und $V_\lambda$ ist F-invariant.\\
\underline{Warum ?} Weil $V_\lambda$ $(F-\lambda)$-invariant ist.
\subsection{Bem:} Falls $(\ker(F-\lambda))= V(\lambda) = 0$
$$\Rightarrow \bigcup_{n\in \mathbb{N}} \ker(F-\lambda)^n = 0$$
\subsection{Bew:}
Sei $v \in \bigcup_{n \in \mathbb{N}}(F-\lambda)^n$
$$\Rightarrow \exists n \in \mathbb{N} | (F-\lambda)^n(v) = 0$$
$n=0$\\
$\rightarrow \ub{Id(v)}_{=V} = 0$\\
sonst: $$(F-\lambda)((F-\lambda)^{n-1}(v)) = 0$$
$$\Rightarrow (F-\lambda)^{n-1}(v) \in \ker(F-\lambda) = 0$$
$$\Rightarrow (F-\lambda)^{n-1}(v) = 0$$
\subsection{Lemma:}
Seien $\lambda_1,\dots, \lambda_k$ verschiedene Elemente aus K
$$\Rightarrow V_{\lambda_i} \cap \sum^k_{\substack{ j=1 \\ j \neq i}} V_{\lambda_j} = \{0\}$$
\subsection{Bew:}
$V_{\lambda_j}$ ist $(F-\lambda_j)$.invariant\\
$\Rightarrow V_{\lambda_j}$ ist F-invariant\\
$\Rightarrow V_{\lambda_j}$ ist $(F-\lambda_i)$-invariant\\
Wir wollen zuerst zeigen, dass $F-\lambda_i | V_{\lambda_j}$ ein Automorphismus ist ($i \neq j$)\\
$\Rightarrow$ Es genügt zu zeigen, dass $F-\lambda_i$ injektiv auf $V_{\lambda_j}$ ist.\\
Sei $w\in V_{\lambda_i} \setminus \{0\}$\\
$$\Rightarrow \exists m \in \mathbb{N} \tx{ kleinstes } (F-\lambda_i)^m(w) = 0$$
$$\buildrel ( m>0 (w\neq0) \over \Rightarrow (F-\lambda_j)^{m-j} (w) \neq 0$$
$$\Rightarrow (F-\lambda_j)^{m-1} (\ub{(\lambda_i - \lambda_j)}_{\neq 0} (w)) \neq 0$$
\begin{align*}
\tx{So} 0 &\neq (F-\lambda_j)^{n-1} ([F(w) - \lambda_j w] - (F(w) - \lambda_i w ))\\
&= \ub{(F-\lambda_j)^m(w)}_{=0} + (F-\lambda_j)^{m-1}(F-\lambda_i)(w)\\
&\Rightarrow (F-\lambda_i)(w) \neq 0 \tx{ OK!}
\end{align*}
Insb. ist jede Potenz $(F-\lambda_i)^k$ ein Automorphismus von $V_{\lambda_i}$
$$V_{\lambda_i} \bigcap \sum^k_{\substack{j=1 \\ j\neq i}} V_{\lambda_j} = \{0\}$$
$$v\in V_{\lambda_i} \bigcap \sum^k_{j\neq i} V_{\lambda_j}$$
$$ v = \sum_{j\neq i} \ub{v_j}_{\in V_{\lambda_j}}$$
$ \Rightarrow \exists \  m_j \tx{ kleinstes } $
$$(F-\lambda_j)^{m_j}(v_i) = 0$$
$$ M = (F-\lambda_i)^{m_1} \circ \dots \circ (F-\lambda_{i-1})^{m_{i-1}} \circ (F-\lambda_{i+1})^{m_{i+1}} \circ \dots \circ (F-\lambda_K)^{m_k}$$
ist ein Automorphismus von $ V_{\lambda_i}$\\\\


% neue Vorlesung

Wiederholung:\\\\
$\enph \quad \dim_k V = n \qquad \lambda \in K$
$$V_\lambda = \bigcup \equalto{\ker(F-\lambda)^n}{(F-\lambda)\circ \dots \circ (F-\lambda)}$$
$V(\lambda) = \ker (F-\lambda) = \{0\}  \Leftrightarrow \chi_F(\lambda) \neq 0 \Rightarrow V_\lambda = \{0\}$\\
$\lambda_1,\dots,\lambda_k \tx{ verschiedene Eigenwerte}$\\
$\Rightarrow V_{\lambda_i}  \bigcap \sum^k_{\substack{i=1 \\ i \neq j}} V_{\lambda_i} = \{0\}$
\subsection{Bem:}
$V_\lambda $ ist $(F-\lambda)$-invariant\\
$\Rightarrow V_\lambda$ ist F-invariant
\subsection{Lemma:}
Sei $\lambda \in K $\\
$\dim V_\amalg = \ord_\lambda (\chi_F)$\\
Ferner hat $ F \upharpoonright V_\lambda $ Matrixdarstellung der Form
$\begin{pmatrix}
\lambda & & * \\
& \ddots \\
0 & & \lambda
\end{pmatrix}$

\subsection{Bew}
Sei $K = \tx{ord}_\lambda ( \chi_F(T))
\Rightarrow \chi_F (T) = (T - \lambda)^K \cdot Q(T)$ wobei: $Q(\lambda) \neq 0$ 
\subsection{Beh}
Es gibt einen $F$-inv. Unterraum $U \subset V$ der dimension $K$, so dass $F \upharpoonright U$ matrix darstellung $$\begin{pmatrix}
\lambda & & \star \\ & \ddots & \\ 0 & & \lambda
\end{pmatrix}$$

Falls $ n=0  \rightarrow $ Ok!\\
OBdA ist $ k\ge 1 $ Wir beweisen die Behauptung mit Induktion auf $ n = \dim V $\\
n=1 $\Rightarrow$\\
$\exists v \in V\setminus \{0\} \quad F(v) = \lambda \cdot v$\\
$\Rightarrow V = \equalto{\underbrace{\spa (v)}}{U_0}$\\
$n\ge 2$\\
$\lambda$ ist ein Eigenwert\\
$\Rightarrow \exists v \in V \setminus \{0\} \quad F(v) = \lambda\cdot v$\\
$U = \spa (v) \rightarrow \dim 1 \quad F\tx{-invariant}$
$\Rightarrow\tilde{F}: V/U_0 \rightarrow V/U_0 ; \bar{\omega} \mapsto \ov{F(\omega)}$
$$\equalto{\chi_F(T)}{(T-\lambda)(T-\lambda^{k-1}) \cdot Q(T)} = \equalto{\chi_{F \upharpoonright U}(T)}{(T-\lambda)} \cdot \chi_{\tilde{F}}(T)$$
da K[T] Integritätsbereich:
$$\Rightarrow \chi_{\tilde{F}} (T) = (T-\lambda)^{k-1} \cdot Q(T)$$
$$ Q(\lambda)\neq 0$$

Nach I.A. existiert einen $\tilde{F}$- Invarianten Unterraum $\tilde{U} \subset V / U$ der Dimension $n-1$, so dass die Matrixdarstellung von $\tilde{F}$ der Form 
$$\begin{pmatrix}
\lambda & & \star \\ & \ddots & \\ 0 & & \lambda
\end{pmatrix}$$
besitzt.
Sei $\bar{v}_2, \dots, \bar{v}_K$ eine Basis von $\bar{U}$ so dass die Darstellungsmatrix von $\tilde{F}$ bzg $\{ \bar{v}_2, \dots, \bar{v}_K \}$ 
$$A =\begin{pmatrix}
\lambda & & \star \\ & \ddots & \\ 0 & & \lambda
\end{pmatrix} \tx{ist.}$$
Die Vektoren $\underbrace{\{ v, v_2 \dots , v_K \}}_{\tx{Vektoren aus \underline{V}}}$
sind linear unabhängig.\\
$U = \tx{span} (v, v_2 \dots , v_K)$ hat Dimension K.\\

$U$ ist $F$-Invariant 
\begin{align*}
F( \sum_{i=1}^{K} \mu_i v_i) &= \sum \mu_i F(v_i) \\
&= \underbrace{\mu_1 \lambda v_1 + \sum^k_{i=2} \mu_i F(v_i) }_{\in U}
\end{align*}
da: $ F(v_i) \buildrel \tx{nach} U_0 \over = \sum_{j\le i} a_j = v_j$ 
$F \upharpoonright U$
$$F(v_1),F(v_2), \dots F(v_k)$$
$$\begin{pmatrix}
\lambda & & & *\\
& \lambda \\
& & \lambda \\
& & & \lambda
\end{pmatrix}$$

Zu Zeigen: $U = V_\lambda$\\
$U \subset V_\lambda: $\\
$$\chi_{F \upharpoonright U} = (T-\lambda)^{k}$$
$$\Downarrow \tx{Caley Hamilton}$$
$$(F-\lambda)^{k}= 0 \quad \tx{auf } U$$
$\Rightarrow U \subseteq \ub{\ker(F-\lambda)^k}_{\subset V_\lambda} $\\
Falls $ U \subsetneq V_\lambda $\\
$\Rightarrow \dim V_\lambda/ U \ge 1$
$$\equalto{\chi_F(T)}{(T-\lambda)\cdot Q(T)} = \equalto{\chi_{F \upharpoonright U}(T)}{(T-\lambda)^k} \cdot \chi_{\tilde{F}} (T)$$
$\tilde{F}: V/U \to V/U$
$$\Rightarrow \chi_{\tilde{F}(T)} = Q(T) $$
aber $ Q(\lambda) \neq 0$ 
$\Rightarrow \lambda$ ist kein Eigenwert von $ \tilde{F} $!!\\
$\Rightarrow$ Der Hauptraum für $ \lambda $ von $\tilde{F}: V/U \to V/U$ ist \underline{trivial}\\
Sei $\omega \in V_\lambda$\\
$\Rightarrow \exists s \in \mathbb{N}$\\
$(F-\lambda)^s(\omega) = 0 \Rightarrow (F -\lambda)^s(\ov{\omega}) = 0$\\
$\Rightarrow \ov{\omega} = 0 \Rightarrow \omega \in U$
\subsection{Def:}
$\enph$ heißt nilpotent falls es eine feste Zahl m existiert, so dass $\underbrace{F\circ \dots \circ F}_{m} = F^m = 0$ auf V ist.
\subsection{Lemma:}
Sei $\enph \quad \dim V = n$ Folgende Aussagen sin äquivalent:
\begin{itemize}
	\item[1)] F ist nilpotent
	\item[2)] $\forall v \in V \exists m_v \in \mathbb{N}: F^{m_v}(v) = 0$
	\item[3)] Es existiert eine Basis von V, so dass F Darstellungsmatrix $$\begin{pmatrix}
	0 & & *\\
	& \ddots \\
	0 & & 0
	\end{pmatrix}$$ hat
	\item[4)] $\chi_F(T) = T^n$
\end{itemize}

\Bew{}{$\boxed{1 \Rightarrow 2}$ Trivial\\
	$\boxed{2 \Rightarrow 3}$ Induktion auf $n = \dim V$
	$n = 1: Sei v \in V\setminus \{0\} \\
	\exists m_v \in \mathbb{N} \quad \tx{kleinstes} \\
	F^{m_v}(v) = 0 \\
	\Rightarrow m_v \neq 0 \\
	\Rightarrow F^{m_ - 1}(v) \neq 0 \\
	\Rightarrow V= \tx{span}(F^{m_ - 1}(v))$
	\noindent Ferner $ F(F^{m_v-1} (v)) = 0$\\
	\indent $\rightarrow$ Darstellungsmatrix bzgl. $\{F^{m_v-1}(v)\}$ ist $(0)$\\
	$n\ge 2$\\
	Sei $v_i \in V\setminus\{0\}$, so dass $F(v_i) = 0$\\
	$\Rightarrow U = \spa (v_i)$ ist $F$-invariant\\
	$ \Rightarrow \tilde{F}: \ub{V/U}_{\equalto{\dim < n}{n-1}} \to V/U $\\
	$\Rightarrow$ I.A. es existiert eine Basis $\{\ov{v_2}, \dots , \ov{v_n}\}$ von $V/U$, so dass $\tilde{F}$ Darstellungsmatrix:$$\begin{pmatrix}
	0 & & *\\
	& \ddots \\
	0 & & 0\\
	\end{pmatrix}$$ hat.
	Die Familie $ \{v_1,v_2,\dots , v_n\} $ ist eine Basis von V und hat Darstellungsmatrix 
	$$ \begin{pmatrix}
	0 & & & & * \\
	& 0 \\
	& & 0 \\
	& & & \ddots \\
	& & & & 0
	\end{pmatrix} $$
	
	$\boxed{3 \Rightarrow 4}$ \\
	$$\chi_F(T) = \det \bigg( T \cdot E_n - \begin{pmatrix}
	0 & & *\\
	& \ddots \\
	& & 0
	\end{pmatrix} \bigg) = \det\begin{pmatrix}
	T & & * \\
	& \ddots \\
	& & T
	\end{pmatrix} = T^n$$
	
	$\boxed{4 \Rightarrow 1}$  Caley-Hamilton\\
	$$F^n = \chi_F(F) = 0$$
}
\subsection{Satz: Jorda-Charelley Zerlegung}
$\enph \ V$ endlichdim.\\
Falls $ \chi_F(T) $ in linearfaktoren zerfällt dann ist $ V = \bigoplus\limits_{i=1}^{k} V_{\lambda_i} \quad \lambda_1,\dots, \lambda_k$ die verschiedenen Eigenwerte und F lässt sich als Blockmatrix darstellen
$$\begin{pmatrix}
A_1 \\
& A_2 \\
& & A_3 \\
& & & \ddots \\
& & & & A_k
\end{pmatrix} \qquad A_i = \begin{pmatrix}
\lambda_i & & *\\
& \ddots \\
0 & & \lambda_i
\end{pmatrix}$$

insb. ist $ F = \ub{G}_{\tx{diagonalisierbar}}+\ub{H}_{\tx{nilpotent}} $\\
$G \circ H = H \circ G $\\
Sei $ G: V\to V \quad G _{V_{\lambda_i}} =$ Multiplikation mit $ \lambda_i $. E hat diagonale Matrix bzgl. der Basis $ \mathcal{B} $ 
$$\begin{pmatrix}
\lambda_1 \\
& \lambda_2\\
& & \ddots \\
& & & \lambda_n
\end{pmatrix}$$

setzt $ H = F - G $ die Darstellungsmatrix von $ H $ bzgl. $ \mathcal{B} $ ist:
$$\begin{pmatrix}
0 & & *\\
& \ddots \\
& & 0\\
& & & 0 & & *\\
& & & & \ddots\\
& & & & & 0\\
& & & & & & \ddots\\
& & & & & & & 0 & & *\\
& & & & & & & & \ddots\\
& & & & & & & & &0
\end{pmatrix} \begin{matrix}
\\
\bigg\} V_{\lambda_1}\\
\\
\bigg\} V_{\lambda_2}\\
\\
\\
\\
\\
\\
\\
\end{matrix} \Rightarrow H \tx{ ist nilpotent}$$

\subsection{Bew:}
$\lambda_1,\dots , \lambda_k$ verschiedene Eigenwerte
$$\chi_F(T) = \prod_{i=1}^{k}(T-\lambda)^{d_i}$$
$\dim V = \grad \chi_F(T) = n\sum_{i=1}^{k} \equalto{d_i}{\dim V_{\lambda_i}}$\\\\
$$\dim(\sum_{i=1}^{k} V_{\lambda_i}) \buildrel V_{\lambda_i} \bigcap \sum_{\substack{j=1 \\ j \neq i}}^{k} V_{\lambda_j} = \{0\}\over = \sum d_i = n = \dim V$$
$\Rightarrow V = \bigoplus^{k}_{i=1} V_{\lambda_i}$\\
V besitzt eine Basis $ \mathcal{B} $, welche aus der Vereinigung der Basen jedes $V_{\lambda_i}$ besteht.

$F$ wird durch $\equalto{F \upharpoonright V_\lambda\ ,}{\equalto{\begin{pmatrix}
		\lambda_1 & & \star \\ & \ddots & \\ 0 & & \lambda_1
		\end{pmatrix}}{A_1}} \dots \ ,\equalto{F \upharpoonright V_{\lambda_K}}{\equalto{\begin{pmatrix}
		\lambda_K & & \star \\ & \ddots & \\ 0 & & \lambda_K
		\end{pmatrix}}{A_K}}$ bestimmt \\

$$\begin{pmatrix}
A_1 &&&\\ &A_2&&\\ &&\ddots &\\ &&&A_K
\end{pmatrix}$$


% markus blatt 2???

zu Zeigen: $G\circ H = H\circ G$ Beachte, dass jedes $V_{\lambda_i}$ $G$-Invariant.\\
$\Rightarrow V_{\lambda_i}$ ist $H$-Invariant.\\
Es genügt zu zeigen, dass $G\circ H = H\circ G$ auf $\underline{\underline{V_{\lambda_i}}}$\\ 
$w \in V_{\lambda_i}$\\
$H\circ G(w) = H(\lambda\cdot w)$\\

$G\circ H (w) \buildrel{W(w) \in V_{\lambda_i}} \over{=} \lambda_1 H(w)$

% neue vorlesung
%beweis von satz 0.49 

%neue vorlesung wieder mit pizarro

\subsection{Def:}
$ \enph $ V endlichdim.\\
Eine Basis $ \{v_1,\dots,v_n\} $ ist F-adaptiert, falls:\\
$$F(v_1) = 0$$
$$2\le j \le n \quad F(v_j) = \casess{0}{}{v_{j-1}}{}$$
\subsection{Bem:}
Falls V eine F-adaptierte Basis besitzt, dann ist $ \enph $ nilpotent.
\subsection{Bew:}
Sei $ \{v_1,\dots ,v_n\} $ F-adaptiert\\
Es genügt zu zeigen, dass $ F^n = 0 $ (als Endomorphismus)
$$F^n(\equalto{v}{\sum_{j=1}^{n} \lambda_j v_j}) = \sum_{j=1}^{n} \lambda_j \ub{F^n(v_j)}_{=0} = 0$$
\underline{Notation}
$$ \mathcal{N}_m = \begin{pmatrix}
0 & 1 \\
& 0 \\
& & \ddots \\
& & & 0 & 1\\
& & & & 0 
\end{pmatrix} \in \mathcal{M}_{m\times m} (\mathbb{K})$$
$\mathcal{N} _j = (0) $
$$ \mathcal{N}^m_m = \begin{pmatrix}
0 & & 0 \\
& \ddots \\
0 & & 0
\end{pmatrix}$$
\subsection{Def:}
$ \enph $ nilpotent. Es gibt $ m \le n $ kleinste Zahl, sodass 
$$ \ker(F^m) \subsetneq \dots \ker (F^m) \subsetneq \ker(F^{m+1}) \subsetneq \dots \subsetneq V$$
m heißt der Index von F
$$ V = \ker(F^m) \bigoplus F^m(V)$$
\subsection{Satz:}
Sei $ \enph \dim(V) = n $ und $ \mathcal{B} $ eine F-adaptierte Basis von F. Dann hat F Matrixdarstellung der Form
$$\begin{pmatrix}
\mathcal{N}_{k_1}\\
& \ddots \\
& & \mathcal{N}_{k_r}
\end{pmatrix}$$
$$\sum_{j=1}^{r} k_j = n \quad \tx{ Index} (F) = \tx{max} \{k_j\}_{1 \le j \le r}$$
\subsection{Bew:}
Sei $  \{v_1,\dots,v_n\} $ F-adaptiert\\
Sei $ k \le i_1 < \dots < i_r < n $ eine Aufzählung der Menge $ \{j | F(v_{j+1})=0\} $
\nopagebreak
$$v_1,v_2,\dots,v_{i_j},v_{i_{j-1}}, \dots\quad $$
\nopagebreak
$$ \begin{array}{rrrrrrrrrrrrrrrr}
v_1 & v_2 & \dots & v_{i_1} & v_{i_{j-1}} & & & & & & & & & & &
\end{array} $$
$$\begin{pmatrix}
0 & 1\\
& 0 \\
& & \ddots\\
& & & 0 & 1\\
& & & & 0\\
& & & & & 0 & 1\\
& & & & & & 0 \\
& & & & & & & \ddots\\
& & & & & & & & 0 & 1\\
& & & & & & & & & 0\\
\end{pmatrix} \begin{array}{l}
\\
\\
\bigg\} i_1\\
\\
\\
\\
\bigg\} i_2-i_1\\
\\
\\
\end{array}$$

% nacher matrix foto: done 

\subsection{Satz:}
Sei $ F: V \to V  $ nilpotent mit Index k. Gegeben einen Unterraum $ U \subset V $ derart, dass $ U \cap \ker(F^{k-1}) = \{0\} $. Dann lässt sich jede Basis von $ U $ zu einer $ F $-adaptierten Basis von $ V $ ergänzen.

\subsection{Kor:}
Jeder nilpotente Endomorphismus besitzt eine $ F $-adaptierte \underline{Basis}
$$ \ker F \subsetneq \ker F^2 \subset \dots \ker F^{k-1} \subsetneq \ker F^k = \dots = V$$
$$ B_1 \quad \subset \quad  B_2 \ \ \subset \ \  \dots B_{k-1} \quad \subset \quad B_k \qquad \qquad \quad $$
$$\bigg(F(U_1) \dots F(U_2) \rightarrow B_{k-1} \qquad U_1, \dots U_r \in B_k \setminus \ker(F^{k-1}) \bigg)$$
$ \ker F^{K-1} $ hat ein Komplement $ \equalto{U}{\substack{U_1 \\ \vdots \\ U_j}}  $ in $ \ker F^k $

\subsection{Beh:}
$ \{F(U_1), \dots , F(U_r) \} $ sind lin. unabh. \\
$ \rightarrow $ sie bilden die Basis von $ F(w) $

\subsection{Bew:}
$$ \sum_{i=1}^{r} \lambda F(U_i) = 0 = F\bigg(\sum_{i=1}^{r} \lambda_i U_i\bigg) $$
$$ \sum_{i=1}^{r} \lambda_i U_i \in \ker (F)$$
$$ \qquad \Downarrow W \cap V = \{0\}$$
$$ \sum \lambda_i U_i = 0 \qquad $$
$$ \qquad \qquad \qquad \qquad \quad  \Downarrow \{ U_1, \dots , U_i\} \tx{ sind lin. unabh. }$$
$$\underline{ \lambda_1 = \dots = \lambda_i = 0 } $$
Aus unserer Induktionsannahme
$$ F(w) \tx{ und } F \upharpoonright V': V' \to V'$$
$ \Rightarrow $ Es gibt eine adaptive Basis $ \{v_1',\dots , v_m'\} $ von $ V'$,\\ welche $ \{F(U_1), \dots , F(U_r) \} $ ergänzt.\\
$ F(U_j) = v' _{i_j} $ wobei $ i_1 < \dots < i_r $ (sonst ordne $ v'_j $ 's um !)
\begin{align*}
v_1 &= v_1' \\
v_2 &= v_2' \\
&\vdots \\
v_{i_1} &= v_{i_1}' \\
v_{i_1 +1} &= U_1 & \rightarrow \tx{ Basis } \{v_1,\dots,v_n\} \tx{ von } V (W\cap V' = {\alpha_i})\\
v_{i_1+2} &= v_{i_1+1}' \\
&\vdots \\
v_{i_2} &= v_{i_2 -1}' \\
v_{i_2+1} &= v_{i_2}' \\
v_{i_2+2} &= U_2 \\
\end{align*}

\subsection{Folgerung}
Jeder nilpotente Endomorphismus lässt sich bezüglich  einer geeigneten Basis durch eine Blockmatrix darstellen:
$$\left (\begin{array}{*{20}c}
0 & 1\\
& 0 \\
& & \ddots\\
& & & 0 & 1\\
& & & & 0\\
& & & & & 0 & 1\\
& & & & & & 0 \\
& & & & & & & \ddots\\
& & & & & & & & 0 & 1\\
& & & & & & & & & 0\\
& & & & & & & & & & & \ddots \\
& & & & & & & & & & & & 0 & 1\\ 
& & & & & & & & & & & & & 0 \\ 
& & & & & & & & & & & & & & \ddots\\ 
& & & & & & & & & & & & & & & 0 & 1\\ 
& & & & & & & & & & & & & & & & 0\\
\end{array} \right)$$

\subsection{Kor: Jordansche Normalform}
Jeder Endomorphismus eines endl. dimensionalen VR, dessen charakteristisches Polynom in Linearfaktoren zerfällt, lässt sich bezüglich einer geeigneten Basis durch eine Blockmatrix folgender Form darstellen:
$$\left (\begin{array}{*{20}c}
\lambda_1 & 1\\
& \lambda_1 \\
& & \ddots\\
& & & \lambda_1 & 1\\
& & & & \lambda_1\\
& & & & & \lambda_2 & 1\\
& & & & & & \lambda_2 \\
& & & & & & & \ddots\\
& & & & & & & & \lambda_2 & 1\\
& & & & & & & & & \lambda_2\\
& & & & & & & & & & & \ddots \\
& & & & & & & & & & & & \lambda_3 & 1\\ 
& & & & & & & & & & & & & \lambda_3 \\ 
& & & & & & & & & & & & & & \ddots\\ 
& & & & & & & & & & & & & & & \lambda_3 & 1\\ 
& & & & & & & & & & & & & & & & \lambda_3\\
\end{array} \right)$$

\subsection{Bew:}
Induktion auf n\\
$ \boxed{k=1} $\\
$ F = 0 $ als Endomorphismus $ \rightarrow $ Jede Basis ist F-adaptiert\\
$ \boxed{k\ge 2} $\\
Sei $ V' = \ker(F^{k-1}) \subsetneq V $\\
Sei $ \{u_1,\dots , u_j\} $ eine Basis von U.
$U \cap V' = \{0\} \quad U + V' = U \oplus V'$
hat eine Basis: $ \{u_1,\dots,u_j,\ob{\tilde{v}_1, \dots, \tilde{v}_n }^{\tx{Basis von } V'}\} $\\
Ergänze diese Basis zu einer Basis von V $ \{u_1,\dots,u_j,u_{j+1},\dots,u_r,\tilde{v}_1 , \dots , \tilde{v}_n\} $\\
$U \subset \mathcal{W} = \spa(u_1,\dots,u_j)$\\
$ W \cap V' = \{0\} $\\
$ F(W) \subset \ker(F^{k-1}) = V' $ (Weil $ F^k = 0 $), $V'$ ist F-invariant und\\
$ F \upharpoonright V' = V' \to V' $ hat Index \underline{\underline{$ k-1 $}}.
\subsection{Beh:}
$ \{0\} = F(\mathcal{W}) \cap \ker(F^{k-2}) $\\
Sei $ u \in F(\mathcal{W}) \cap \ker(F^{k-2}) $\\
$ \Rightarrow F^{k-2}(u) = 0 $\\
es existiert ein $ w \in \mathcal{W} | u = F(w) $\\
$ 0 = F^{k-2}(u) = F^{k-1} (w)  \rightarrow w \in \ker(F^{k-1} = V')$\\
$ \Rightarrow w = 0 \Rightarrow U = F(w) = 0 $\\
$ W \cap V' = \{0\} $\\

% erik 2 

Es genügt zu zeigen, dass die Basis $ \{v_1,\dots,v_n\} $ \underline{F-adaptiert} ist.\\
z.B. $$F(v_{j+1})  = V'_{i_1} = F(u_1)$$
$ \Rightarrow u_1 - \ub{v_i +1}_{\in V'} \in \ker(F) \subset V' $\\
$ \Rightarrow u_1 \in V' \cap W = \{0\} $ \underline{Widerspruch !}\\\\

$ \enph $ nilpotent index K.

% blockblatt markus 1

\subsection*{Beispiel:}
$$A = \begin{pmatrix}
1 &1&0&1\\0&2&0&0\\-1&1&2&1\\-1&1&0&3
\end{pmatrix} \in \mathcal{M}_{n\times n} (\mathbb{K})$$

\begin{align*} \chi_A(T) &= \det\begin{pmatrix}
T-1 & -1 & 0 & -1\\ 0 & T-2 & 0 & 0 \\ 1 & -1 & T-2 & -1 \\ 1 & -1 & 0 & T-3
\end{pmatrix}\\[8pt]
&= (T-2) \det \begin{pmatrix}
T-1 & 0 & -1 \\ 1 & T-2 & -1 \\ 1 & 0 & T-3
\end{pmatrix}\end{align*}

$$\chi_A (T) = (T-2)^4 \rightarrow 2 \tx{ ist der einzige Eigenwert.}$$
$A-2 E_4 = \begin{pmatrix}
-1&1&0&1\\0&0&0&0&\\-1&1&0&1\\-1&1&0&1
\end{pmatrix}$ ist \underline{\underline{Nilpotent}}\\
$ \ker(A - 2 E_4) = \{(x_1,x_2,x_3,x_4) | -x_1 +x_2 + x_4 = 0\} $

% vorlesung 

\chapter{Dualität}

\subsection{Def:}
Sei $V$ ein $K$-VR Der Dualraum $V^*$ ist die kollektion aller linearen Abbildungen $V \to K$.
\subsection*{Bem:} $V^*$ ist ein $K$-vektorraum. $F + G: V\to K, \ v\mapsto F(v) + G(v)$ ist linear.\\
$\lambda \in K, \ \lambda \cdot F = V \to K, \ v \mapsto \lambda \cdot F(v)$ ist linear.


\subsection{Def:} Sei V endlichdim und wählte eine Basis $ \mathcal{B} = \{ b_1,\dots,b_n\} $ von  V.\\
Die duale Basis $ \mathcal{B}^* = \{b_1^*,\dots,b_n^*\} $ ist eine Kollektion linearer Abbildungen derart, dass $ b^*_i (b_j) = \delta_{ij} = \casess{1}{i=j}{0}{\tx{sonst}} $\\
Insbesondere:
$$b^*_i(v) = b_i^* (\sum_{j=1}^{n} \lambda_j \cdot \backepsilon_j) = \lambda_i$$
\subsection*{Bem:} Falls V endlichdim. ist, dann ist $ \mathcal{B}^*  $ eine Basis von $ V^* $ und sonst $ V\simeq V^* $

\Bew{}{
	$$b_i^* \dots b_n^* \quad \tx{sind linear unabhängig.}$$
	warum? $$\sum \lambda_i b_i^* = 0 \quad \tx{(als lin. Abbildung)}$$
	$$\sum \lambda_i \equalto{b_i^*}{\lambda_j \tx{ für } 1 \leq j \leq n}  (b_j) =0$$ 
	$\lambda_i \dots = \lambda_n = 0 \rightarrow$ OK!
	$$\tx{span}(b_i^* \dots b_n^*) = V^*$$
	Sei $F: V \to K$ beliebig 
	$$F(b_i) = \lambda_i \in K \quad 1 \leq i \leq n$$
	$$(F- \sum \lambda_i b_i^*) ( b_j) = 0 = \equalto{F(b_j)}{\lambda_j} - \underbrace{\sum \lambda_i b_i^* (b_j)}_{\lambda_j} \quad 1 \leq j \leq n$$
	
	Insb. $ V\to V^* $ ist ein Isomorphismus $  b_i \mapsto b_i^* $\\
	\underline{Achtung:} Der Isomorphismus $ V\simeq V^* $ hängt von der Basis $ \{b_1,\dots,b_n\} $ ab! NICHT KANONISCH}
\subsection{Lemma:} 
kanonischer Monomorphismus $ V \buildrel \varphi \over \hookrightarrow (V^*)^* , v \mapsto \varphi_v:\quad V^* \to K , F \mapsto F(v)$ 
\subsection{Bew:} $ \varphi $ ist wohldefiniert 
$$ \varphi_v(F+G) = (F+G) (v) = F(v)+ G(v) \quad \tx{Ok!}$$\\
Zu zeigen: $ \varphi $ ist injektiv (Übungsaufgabe !)
\subsection{Korollar:} Wenn V endlichdim. ist, $ V \simeq V^* $ \underline{kanonisch}

\Bew{}{ 
	$$\dim V = \dim V^* = \dim (V^*)^* \Rightarrow \varphi: V \to (V^*)^* \tx{ ist surjektiv} $$
	$\Rightarrow $ ein Isomorphismus}
\subsection{Lemma:}
Sei $V$ endlich und wähle Basen $\mathcal B = \{b_1 \dots b_i \}$ und $\mathcal B' = \{b_1' \dots b_i' \}$ von $V$.

Seien $ \mathcal{B}^* $ und $ (\mathcal{B}')^* $ die entsprechenden dualen Basen in $ V^* $. Wenn A die Transformationsmatrix von $ \mathcal{B} $ nach $ \mathcal{B}' $ ist, dann ist die Transformationsmatrix $ \mathcal{B^*} $ nach $ (\mathcal{B}')^* \ \ (A^*)^{-1}$
\subsection{Bew:}
$$\begin{pmatrix}
b^*_1\\
\vdots\\
b^*_n
\end{pmatrix} = A \cdot \begin{pmatrix}
b_1\\
\vdots\\
b_n
\end{pmatrix}$$
Sei $ X $ die Transformationsmatrix von $ \mathcal{B}' $ nach $ (\mathcal{B}') ^* $
$$\begin{pmatrix}
b_1'^*\\
\vdots\\
b_n'^*
\end{pmatrix} = X \cdot \begin{pmatrix}
b_1^*\\
\vdots\\
b_n^*
\end{pmatrix}$$

$$\begin{pmatrix}
b_1^*\\
\vdots\\
b_n^*
\end{pmatrix} \cdot \begin{pmatrix}
b_1 & \dots b_n
\end{pmatrix} = E_n $$
$$\begin{pmatrix}
b_1'^*\\
\vdots\\
b_n'^*
\end{pmatrix} \cdot \begin{pmatrix}
b_1' & \dots b_n'
\end{pmatrix} = E_n $$
\begin{align*}
E_n &= \begin{pmatrix}
b_1'^*\\
\vdots\\
b_n'^*
\end{pmatrix} \cdot \begin{pmatrix}
b_1' & \dots b_n'
\end{pmatrix}\\
&= X \cdot \begin{pmatrix}
b_1'^*\\
\vdots\\
b_n'^*
\end{pmatrix} \bigg( A \cdot \begin{pmatrix}
b_1\\
\vdots\\
b_n
\end{pmatrix} \bigg) ^\top\\
&= X \cdot \begin{pmatrix}
b_1^*\\
\vdots\\
b_n^*
\end{pmatrix} \cdot \begin{pmatrix}
b_1 & \dots b_n
\end{pmatrix} \cdot A^\top
\end{align*}

$$En = X \cdot En \cdot A^\top = X \cdot A^\top \Rightarrow X = (A^\top)^{-1}$$

\subsection{Def:}
Sei $F: V \to W$ eine lineare Abbildung. Definiere die duale Abbildung $F^* : W^* \to V^*$, $\psi \mapsto \psi \circ F$


$$\begin{array}{ccc}
V & \buildrel F \over \longrightarrow & W \\
& \searrow & \downarrow \Psi \\
\equalto{\Psi \circ F }{F^*(\Psi)} & & K
\end{array}$$

\subsection*{Bem:}
$F^*$ ist linear. 
$$F^* (\psi_1 + \psi_2) = (\psi_1 + \psi_2) \circ F = \psi_1 \circ F + \psi_2 \circ F = F^* (\psi_1) + F^* (\psi_2)$$

$$ U \buildrel G \over \longrightarrow V \buildrel F \over \longrightarrow W$$
$$ W^* \buildrel F^* \over \longrightarrow V^* \buildrel G^* \over \longrightarrow U^*$$
von $ W^* $ nach $ U^* $:
$$(F\circ G)^* = G^* \circ F^*$$
\subsection{Bew:} $ \Psi \in W^* \quad \Psi \circ W \to K $ 
$$(F \circ G)^*(\Psi) = \ub{\Psi\circ F}_{F^*(Psi)} \circ G = G^* \circ F^*(\Psi)$$
\underline{Eigenschaften:}
\begin{itemize}
	\item[a)] $ (Id_{I_{V}})^* = Id_{I_{V^*}} $
	\item[b)] $ (F+G)^* = F^* + G^* $
	\item[c)] $ (F\circ G)^* = G^* \circ F^* $
	\item[d)] $ (\mu F)^* = \mu F^* $
\end{itemize}

$$(Id_{iv})^* (\psi) = \psi \circ Id_{iv} = \psi$$
\subsection*{Bem:} Falls $F^* =0$ dann ist $\underline{F = 0}$. Feiner, falls $v$ und $W$ endlich dimensional sind.
$$G: W^* \to V^* \tx{ lin. Abbildung}$$ dann gilt es $$F: V \to W \tx{ so dass } F^* = G$$

\Bew{}{
$$F^* = 0$$
Sei $v \in V$ fest. \\
\paragraph{Zu Zeigen:} 
$F (v) = 0$ \\[10pt]
Definiere $$W^* \to \psi \mapsto \equalto{\psi (F(v))}{F^* (\psi)(v)}$$ 
Erinnerung: $W \buildrel \varphi \over \hookrightarrow (W^*)^* , \ w \mapsto \equalto{\varphi_{wi}}{\varphi(wi)} \quad W^* \to K, \ \psi \mapsto \psi(w)$
$$\varphi_{F(v)} (\psi) = \psi (F(v) = \underbracket{F^* (\psi)}_{=0} (v) \ \underline{\underline{= 0}}$$


Aber: $ \varphi $ ist ein Monomorphismus\\
$ \Rightarrow F(v) = 0 \Rightarrow  F = 0 $ als Homomorphismus $ V\to W $\\
Die Operation:
$$ * : \tx{ Hom}(V,W) \mapsto \tx{ Hom}(W^*,V^*)$$
Falls V, W endlichdim. sind:\\
($ V \approx K^n \approx V^*  $ und $ W \approx K^m \approx W^* $)\\

\begin{align*}
\dim \tx{Hom }(V,W) &= \equalto{\dim V}{} \cdot \equalto{\dim W}{}\\
\dim \tx{Hom }(W^*,V^*) &= \dim V^* \cdot \dim W^*
\end{align*} 
Aber $*: \tx{Hom }(V,W) \to \tx{Hom }(W^*, V^*)$ ist injektiv $\Rightarrow$ Surjektiv
$$\dim V = n, \quad \dim W = m$$
}
\subsection*{Bem:}
$F: V \to W$ hat Darstellungsmatrix $A$ bezüglich $\mathcal B = \{b_i \dots b_n\}$ von $V$ und $\mathcal B' = \{b_i' \dots b_m'\}$ von $W$. Seien $\mathcal B^*$ und $(\mathcal B')^*$ die entsprechenden dualen Basen aus $V^* \tx{, und } W^*$.


Dann hat $ F^* $ Darstellungsmatrix $ A^\top $ bzgl $ (\mathcal{B}')^* $ und $ \mathcal{B}^* $.
\subsection{Bew:}
$$ F \cdot \begin{pmatrix}
x_1\\
\vdots\\
x_n
\end{pmatrix} = A \cdot \begin{pmatrix}
x_1 \\
\vdots\\
x_n
\end{pmatrix} = A \cdot \vec{x}$$
$$ F^* \begin{pmatrix}
\Psi_1\\
\vdots\\
\Psi_m
\end{pmatrix} \cdot \begin{pmatrix}
x_1\\
\vdots\\
x_n
\end{pmatrix} = \begin{pmatrix}
\Psi_1 & \dots & \Psi_m
\end{pmatrix} \circ \ub{F \cdot \begin{pmatrix}
	x_1\\
	\vdots \\
	x_n
	\end{pmatrix} }_{A \cdot \vec{x}} $$

Die Darstellungsmatrix von $ F^* $ ist:
\begin{align*}
F^* \cdot \begin{pmatrix}
\Psi_1\\
\vdots \\
\Psi_m
\end{pmatrix} &= \begin{pmatrix}
\Psi_1 & \dots & \Psi_m
\end{pmatrix} \cdot A\\
F^* \cdot \begin{pmatrix}
\Psi_1\\
\vdots \\
\Psi_m
\end{pmatrix} &= \bigg( A^\top \cdot \begin{pmatrix}
\Psi_1\\
\vdots\\
\Psi_m
\end{pmatrix} \bigg)^\top
\end{align*}
$ \rightarrow $ Die Darstellungsmatrix von $ F^* $ ist $ A^\top $
\subsection{Korrolar:}
$$ \det(F^*) = \det(F) $$

% neue vorlesung:

\paragraph{Wid:}
$V K-VR, V^* = \tx{Hom} (V,K)$ ist ein $K-VR$, $V$ endlich dimensional $\rightarrow V \simeq V^*$, $\mathcal B = \{b_1 \dots b_n\} \rightarrow \equalto{\mathcal{B}^*}{\{b_1^* \dots b_n^*\}}$ duale Basis.
$$b_i^* (b_j) = \delta_{ij} \tx{ (kronecker delta)}$$
$$\varphi: V \hookrightarrow (V^*)^* \tx{Monomorphismus},\ v \mapsto \varphi_v:\ V^* \to K,\ F \mapsto F(v)$$ 
\paragraph{Folgerung}
$$\forall v \in V \setminus \{0\} \exists F: V \to K \tx{ linear } F(v) \neq 0$$

$ * : \tx{Hom}(V,W) \to \tx{Hom}(W^*,V^*) $\\
\begin{align*}
F: V \to W \rightarrow F^* &\to V^* \\
\ub{\Psi}_{W \to K} &\mapsto \ub{ \Psi \circ F}_{ V \to K} 
\end{align*}
$ (F\circ G)^* G^* \circ F^* $\\
$ (Id_{IV})^* = Id_{IV^*} $\\
$ V,W $ endlichdim $ \rightarrow * $ \underline{\underline{Isonorphismus}} 

\subsection*{Bem}
$ \enph $ lineare Abb. mit Darstellungsmatrix $ A = (a_{ij}) $ bzgl. der Basen $ \mathcal{B} = \{b_1,\dots ,b_n\} , \mathcal{C}= \{c_1,\dots,c_m\} $\\

\begin{align*}
F^*: W^* &\to V^* \\
\mathcal{C}^* = \{ c^*_1,\dots , c^*_m\} (&\rightarrow) \mathcal{B}^* = \{b^*_1  \dots , b^*_n\} 
\end{align*} 
Darstellungsmatrix von $ F^* $ bzgl. $ \mathcal{C}^* \mathcal{B}^* $ ist $ A^\top $

\Bew{}{
	$$F^* (c_k^*) = \sum _{1\leq \varphi \leq n} \lambda_{\varphi k}b_\varphi^* \quad (1\leq k \leq m)$$
	$$F(b_i) = \sum _{1\leq j \leq m} a_{ij} c_j$$
	$$ \begin{array}{ccc}
	F^* (c_k^*) (b_i) & = &(\sum \lambda_{\varphi k} b_\varphi^*) (b_i)\\
	\rotatebox{90}{=} & &  \\
	C_k^* (F(b_i)) & = & \lambda_{ik}\\
	\rotatebox{90}{=} & & \\
	C_k^*(\sum a_{ji}c_j) & = & a_{ki}
	\end{array} \rightarrow A^\top \tx{ist die Darstellungs matrix}$$
}

\subsection{Lemma:}
$$ V = \bigoplus_{i = 1}^{n} V_i$$
$$ \Rightarrow V^* \simeq \bigoplus_{i=1}^{n} V_i^*$$
$ F: V \to K $ ist eindeutig bestimmt durch $ F \upharpoonright V_1, \dots , F \upharpoonright V_n $
\subsection{Lemma:}
$ \enph $ linear
\begin{itemize}
	\item[a)] F injektiv $ \Leftrightarrow F^* $ surjektiv
	\item[b)] F surjektiv $ \Leftrightarrow F^* $ injektiv
	\item[c)] F Isomorphismus $ \Leftrightarrow F^* $ isomorphismus
\end{itemize}

\subsection{Bew:}
a) $\boxed{\Rightarrow}\ F: V \hookrightarrow W$ injektiv.
\paragraph{Zu Zeigen:} $$F^* W^* \to V^* \ni \psi: V \to K \tx{ surjektiv ist}$$
$$\exists \Theta: W \to K,\ \equalto{F^* (\Theta)}{\Theta \circ F} = \psi$$
$$\forall v \in V,\ \Theta(F(v)) = \psi(v)$$
$$\underbrace{\tx{Im}(F)}_{\simeq V} \subset W \quad (\forall v \in \tx{Im}(F)\ \exists ! v \in V,\ w = F(v))$$
Sei $Z$ ein Komplement von Im$(F)$ in $W$ $\Rightarrow W=$Im$(F) \oplus Z$
$$\forall w \in W: \quad w= \underbrace{w'}_{\in \tx{Im}(F)} + \underbrace{\tilde w}_{\in Z} \tx{eindeutig}$$

Definiere $ \Theta: W \to K $\\
$$ W = F(v) + \tilde{w} \mapsto \Psi(v) \qquad \tilde{w} \in Z$$
Zu Zeigen: $\Theta$ ist linear
$$\Theta(\equalto{w_1}{F(v_1) + \tilde{w}_1} + \equalto{w_2}{F(v_2) + \tilde{w}_2})= \Theta (\equalto{F(v_1) + F(v_2)}{F(v_1+v_2)} + \ub{\tilde{w_1} + \tilde{w_2}}_{\in Z})$$
$$ \Theta(w_1) + \Theta(w_2) = \Psi(w_1) + \Psi(w_2) = \Psi(v_1+v_2) (= F(v_1 + v_2))$$
$\boxed{\Leftarrow}$ $ F^* : W^* \twoheadrightarrow V^* $ surjektiv\\
Zu Zeigen: $ F: V \to W $ injektiv\\
$ v \in V $ / $ F(v) = 0 $\\
$  v \neq 0 \Rightarrow  \exists \Psi:V \to K \quad \Psi(v) \neq 0 $\\
$ \buildrel F^* \tx{ surj.} \over \Rightarrow \exists \Theta : W \to K $
$$ \Theta \circ F = F^*(\Theta) = \Psi $$
$ \Rightarrow \ub{\Theta(F(v))}_{=0} = \Psi(v) $ Wid !

b) $\boxed{\Rightarrow}\ F: V \twoheadrightarrow W$ surjektiv.
\paragraph{Zu Zeigen:} $$F^* : W^* \hookrightarrow V^* \tx{ injektiv}$$
Sei $\Theta \in W^* / F^* (\Theta) = 0$ als lineare Abbildung $V \to K$
$$F^* (\Theta) = \Theta \circ F$$
\paragraph{Zu Zeigen:} $$\forall w \in W,\ \Theta(w) = 0$$
$$w \in W \to \ni v \in V / F(v) = w$$
$$F \tx{ surjektiv} \quad \Theta(w) = \Theta(F(v)) = F^*(\Theta)(v) = 0 $$
$\boxed{\Leftarrow}\ F^* : W^* \hookrightarrow V^*$ injektiv
\paragraph{Zu Zeigen:} $$ F: V \twoheadrightarrow W \tx{ surjektiv}$$
Sei $Z$ ein Komplement von Im$(F)$ in $W=$Im$(F) \oplus Z$
\paragraph{Zu Zeigen:} $$Z = \{0\} $$
Sonst sei $\mathcal B$ eine Basis von Z. 
$$G: W \mapsto K \tx{ linear der art}$$
$G \upharpoonright$Im$(F) = 0$ und $G(b) = 1,\ \forall b \in \mathcal B$\\
$ \Rightarrow G \in W^* $\\
$ F^*(G) = G \circ F : V \to K $\\
Sei $ v \in V $
$$ G \circ F(v) = G(F(v)) = 0 $$
$\Rightarrow F^*(G) $ ist die triviale Abbildung\\
$ \buildrel F^* \tx{ inj.} \over \Rightarrow G =0 $ als lineare Abbildung\\
$ \Rightarrow B = \emptyset \Rightarrow Z = 0 \Rightarrow F $ surjektiv

\section{Duale Paarung}
\subsection{Def:}
Seien $ V,W $ K-VR\\
Eine Abbildung $ \varphi: V \cdot W \mapsto K $ ist bilinear, wenn $ \varphi $ linear in jeder Koordinate ist.
\begin{itemize}
	\item[a)] $ \varphi(v, \lambda_1w_1 + \lambda_2 w_2) = \lambda_1 \varphi(v, w_1) + \lambda_2 \varphi(v,w_2) $
	\item[b)] $ \varphi(\lambda_1 v_1 + \lambda_2 v_2, w) \;= \lambda_1 \varphi(v_1,w) + \lambda_2 \varphi(v_2,w) $
\end{itemize}

\subsection*{Bem:} Falls $V, W$ endlich dimensional mit Basen $\mathcal B = \{b_1 \dots b_n\}$ von V\\
und $\mathcal C = \{c_1 \dots c_m\}$ von W. Dann ist $\varphi$ eindeutig bestimmt durch die $(n \times m)$-Matrix $A= (\varphi(b_i, c_j))$
\begin{align*}
\varphi(v, w) &= \varphi \bigg(\sum_{i \leq n} \lambda_i b_i, \sum_{j \leq m}\mu_j c_j\bigg)\\
&= \sum_{i \leq n} \lambda_i \varphi \bigg(b_i, \sum_{j \leq m} \mu_j c_j \bigg)\\
&= \sum_{i \leq n} \lambda_i \sum_{j \leq m} \mu_j \varphi(b_i, c_j)\\
&= (\lambda_1, \dots \lambda_n)A \begin{pmatrix}
\mu_1 \\ \vdots \\ \mu_m
\end{pmatrix}
\end{align*}

\subsection{Bem:}
Falls wir Basen $ \mathcal{B}' $ von $ v $ und $ \mathcal{C}' $ von $ W $ gewählt hätten, dann ist die Darstellungsmatrix von $ \varphi $
$$\mathcal{M}(\mathcal{B},\mathcal{B}')^\top \cdot A \cdot \mathcal{M}(\mathcal{C}, \mathcal{C}')$$
$$ \begin{pmatrix}
\lambda_1\\
\vdots\\
\lambda_n
\end{pmatrix} = \mathcal{M}(\mathcal{B},\mathcal{B}') \begin{pmatrix}
\lambda_1'\\
\vdots\\
\lambda_n'
\end{pmatrix}$$
$$ \begin{pmatrix}
\mu_1\\
\vdots\\
\mu_n
\end{pmatrix} = \mathcal{M}(\mathcal{C},\mathcal{C}') \begin{pmatrix}
\mu_1'\\
\vdots\\
\mu_n'
\end{pmatrix}$$
$$\begin{pmatrix}
\lambda_1 & \dots & \lambda_n
\end{pmatrix} = \begin{pmatrix}
\lambda_1' & \dots \lambda_n'
\end{pmatrix} \mathcal{M}(\mathcal{B},\mathcal{B}')^\top$$

\paragraph{Folgerung}
Der Rang hängt nicht von der Auswahl der Basen ab $\Rightarrow$ Rg$(\varphi)$ ist \underline{wohldefiniert.}

\subsection{Def:}
Ein Tripel $(V,W,\varphi)$ ist ein duales Paar, falls dim $V =$ dim $W < \infty $
$$\varphi: V \times W \to K \tx{ ist bilinear und Rg}(\varphi) = \dim (V) = \dim (W)$$


\subsection*{Bem:} Falls $ \varphi: V \times W \to K $ bilinear ist, dann ist $ \varphi': W \times V \mapsto K \quad (w,v) \mapsto \varphi(v,w)$ auch bilinear.
\subsection*{Aufgabe:}
$ (V,W,\varphi) $ duales Paar $ \Rightarrow (W,V,\varphi') $ auch!
\subsection{Lemma:} $ V,W $ fest
\begin{center}
	\begin{tabular}{cccc}
		$ \ub{\{\varphi: V\times W \to K \}}_{\tx{bilinear}} $ & $ \buildrel \tx{Bijektion} \Phi \over \longleftrightarrow$ & $ \ub{\{F: V \to W^*\}}_{\tx{linear}} $ & $  $\\
		\\
		$ \varphi: V \times W \to K $ & $ \buildrel \Phi \over \longmapsto $ & $ F_\varphi V \to W^* $ & $  $\\
		$  $ & $  $ & $ v \mapsto F_\varphi (v) : $ & $ W \to K $\\
		$  $ & $  $ & $  $ & $ w \mapsto \varphi(v,w) $\\
		$ \varphi_F : V \times W \to K $ & $ \buildrel \Phi ^{-1} \over{\reflectbox{$\longmapsto$}}$ & $ F: V \to W^* $ & $  $\\
		$ (v,w) \mapsto F(v)(w) $ & $  $ & $  $ & $  $\\
	\end{tabular}
\end{center}
Ferner, falls $ \dim V, \dim W < \infty $:\\
$$ (V;W;\varphi) \tx{ duales Paar } \quad \Leftrightarrow \quad F_\varphi : V \to W^* \tx{ Isomorphismus}$$
% its the final tafel:
$$\Phi^{-1} \circ \Phi (\varphi)(v,w)$$
$$\Phi^{-1}\bigg(F_\varphi(v):w\mapsto\varphi(v,w)\bigg) = \varphi(v,w)$$   


%%% this is the newest shit
\subsection{Wiederholung}
$$\varphi: V \times W \to K \tx{ Bilinearform}$$
$$\mathcal B = \{b_1 \dots b_n\} \tx{ von V}$$
$$\mathcal C = \{c_1 \dots c_m\} \tx{ von W}$$
$$\varphi \tx{ hat eine Darstellungsmatrix}$$
$$A_\varphi = 
\begin{pmatrix}
\varphi(b_1, c_1) & \dots & \varphi(b_1, c_m) \\
\vdots & & \vdots \\
\varphi(b_n, c_1) & \dots & \varphi(b_n, c_m)	
\end{pmatrix}$$
$$\varphi \bigg( \sum \lambda_i b_i, \sum \mu_j \epsilon_j \bigg) = (\lambda_1 \dots \lambda_n) A_\varphi \begin{pmatrix} c_1 \\ \dots \\ c_m \end{pmatrix}$$
$$\tx{Rg}(\varphi) \tx{ ist eindeutig bestimmt.}$$
$$(V, W, \varphi) \tx{ ist ein duales Paar, falls } \equalto{\tx{dim } V}{\tx{Rg}(\varphi)} = \tx{ dim } W < \infty$$
$$ \varphi: V \times W \to K \tx{ duales Paar}$$
$$\rightarrow \varphi' = W \times V \to K,\ (w,v) \mapsto \varphi(v,w) \tx{ \underline{dual}}$$
\subsection{Bew:}
$ \Phi $ ist Wohldefiniert\\
$ F_\varphi (v) \in W^* $, weil $ \varphi(v,???-) $ linear in der Koordinate ist $ F_\varphi $ ist linear weil $ \varphi $ linear in der ersten Koordinate ist (da es Wohldefiniert ist)
$$\Phi^{-1} \circ \Phi = Id_\chi$$
$$\Phi \circ \Phi^{-1} = Id_y$$
Falls $V,W$ endlich dimensional sind, mit Basen $\mathcal{B} = \{b_1 \dots b_n\}$,\\ $\mathcal{C} = \{c_1 \dots c_m\}$, $A_\varphi$ die Darstellungsmatrix von $\varphi$ bzg. dieser \underline{\underline{Basen}}
$$A_\varphi = \equalto{(a_{ij})}{\varphi(b_i, c_j)}$$
Sei $\mathcal C^* = \{c_1^* \dots c_m^*\}$ die duale Basis zu $C$ in $W^*$ 
$$c_i^*(c_j) = \delta_{ij}$$
Die Darstellungsmatrix von
$$F_\varphi \tx{ bzg. } \mathcal B, \mathcal C^*$$
$$(\lambda_{kl}) \in \mathcal M_{m \times n}(K)$$
$$F_\varphi (b_k) = \sum \lambda_{lk} c^*_l$$
$$\equalto{F_\varphi (b_k) (c_r)}{(\sum \lambda_{lk}c_l^*)(c_r)} = \varphi (b_k, c_r)$$
$$= \sum \lambda_{lk}c_l^*(c_r) = \lambda_{rk}$$
$$\begin{pmatrix}
\lambda_{11} & \lambda_{12} & \dots \\
\lambda_{21} & \ddots & \\
\vdots & & 
\end{pmatrix} = A_\varphi^\top$$
Wobei $\lambda_{12} = \varphi(b_2, c_1) = a_{21}$, und $\lambda_{21} = a_{12}$
$$\tx{Rg}(A_\varphi^\top) = \tx{Rg}(A_\varphi)$$
$$(V,W,\varphi) \tx{ ist ein duales Paar} \Leftrightarrow \dim V = \equalto{\dim W}{\dim W^*} = \equalto{\tx{Rg}(A_\varphi)}{\tx{Rg}(A_\varphi^\top)}$$ 
$$\Leftrightarrow F_\varphi \tx{ Isomorphismus.}$$
\subsection{Kor:}
Seien $V,W$ endlich dimensional, $\varphi:V \times W \to K$ Bilinearform. Dann ist $(V,W,\varphi)$ duales Paar, genau dann wenn $\varphi$ \underline{nicht-ausgeartet} ist d.h.
\begin{align*}
&a) \forall v \in V:\ \varphi(v,w) = 0\ \forall w \in W \Rightarrow v = 0 \\
&b) \forall w \in W:\ \varphi(v,w) = 0\ \forall v \in V \Rightarrow w = 0
\end{align*}
\subsection{Bew:}
$\boxed{\Rightarrow}$
\begin{itemize}
	\item[a)] Sei $  v\in V $ fest $ \varphi(v,w) = 0 \qquad \forall w \in W $\\
	$ \varphi (v,w) = F _{\varphi} (v)(w) \Rightarrow F_{\varphi} (v) = 0  \tx{ als Abbildung} \Rightarrow v = 0$\\
	$ (V,W,\varphi) \tx{ duales Paar } \Rightarrow F_{\varphi} \tx{ Isomorphismus } $
	\item[b)] $ (V,W,\varphi) $ duales Paar $ \Rightarrow (W,V,\varphi') $ auch dual $ \Rightarrow F_{\varphi'}: W \to V^* $ Isomorphismus\\
	$ v\in V , \varphi(v,w) = 0 \quad \forall v \in V $\\
	$ \varphi(v,w) = 0 \ \ = \varphi'(w,v) = F_{\varphi'} (w)(v) \ \ \Rightarrow F_{\varphi}(w) = 0 \Rightarrow w = 0 $
\end{itemize}
$ \boxed{\Leftarrow} $ Es genügt zu zeigen, dass $ F_{\varphi}: V \to W^* $ Isomorph ist: \\
Eigenschaft:
\begin{itemize}
	\item[a)] $ \Rightarrow F_{\varphi} $ Isomorph (bijektiv)\\
	$ \Rightarrow \dim V \le \dim W^ = \dim W $\\
	es genügt zu zeigen, dass $ \dim V = \dim W^* $ (da $ F_{\varphi} $ injektiv)  und, dass daraus folgt: $ (W,V,\varphi') \quad  (\varphi' = \tx{ bilinearform } ) $
	\item[b)] $ \Rightarrow F_{\varphi'}: W \to V^* $ injektiv $ \Rightarrow \dim W = \dim V^* = \dim V $\\
	$ \Rightarrow \dim V = \dim W $
\end{itemize}
\begin{flushright}
	$ \square $
\end{flushright}

\paragraph{Beispiel:}
$$(\mathbb R^2, <, > )$$
$$<x_1 \dots x_n, y_1 \dots y_n> = \sum_{i=1}^n x_i y_i \in \mathbb R.$$
$$<(x_1 \dots x_n), (1, 0 \dots 0)> = x_1 = 0$$

\subsection{Kor:}
Sei $(V,W,\varphi)$ ein duales Paar. Für jede Basis $\{c_1 \dots c_n\}$ aus $W$ gibt es eine Basis $\{b_1 \dots b_n\}$ aus $V$, welche ,,dual`` zu $\{c_1 \dots c_n\}$ ist:
$$\varphi(b_i, c_j) = \delta_{ij}$$

\Bew{}{
	Für $\{c_1 \dots c_n\}$ aus $W$ eine Basis.
	Sei $\{c_1^* \dots c_n^*\}$ die duale Basis aus $W^*$
	$$F_\varphi: V \to W^* \tx{ Iso}$$
	Sei $b_i = F_\varphi^{-1}(c_i^*),\ \{b_1 \dots b_n\}$ ist eine duale Basis von $V$
	$$\varphi(b_i, c_j) = \equalto{\underbracket{F_\varphi(b_i)}}{c_i^*}(c_j) = \delta_{ij}$$
}

\subsection{Def:}
Sei $(V,W,\varphi)$ ein duales Paar. $U \subset V$ Unterraum von $V$.\\
Definiere $$U^\perp = \{w \in W \ |\  \varphi (u,w) = 0\ \forall u \in U \}$$
\subsection*{Bem:}
$U^\perp$ ist ein Unterraum von $W$.
$$(0)^\perp = W$$
$$V^\perp = \{0\}, \tx{ weil $\varphi$ nicht-ausgeartet ist!}$$
$$U \subset V$$
$$\equalto{(U^\perp)^\perp}{U} = \{v \in V \ | \  \forall w \in U^\perp\ \varphi(v,w) = 0 \}$$
\Bew{}{
	$$U\subset (U^\perp)^\perp$$
	$$u \in U \Rightarrow  u \in (U^\perp)^\perp \Rightarrow \forall w \in U^\perp,\ \varphi(u,w) = 0$$
	Sei $v \notin U.$\\
	z.zeigen: $v \notin (U^\perp)^\perp$\\
	Es gibt $G: V \to K$ derart, $G\upharpoonright U = 0,\ G(v) = 1$
	$$G \in V^* \simeq W\ F_{\varphi'} \tx{ (Isomorphismus)}$$
	$$\Rightarrow \exists w \in W\ F_{\varphi'}(w) = G$$
	D.h. $\forall z \in V\ G (z) = \varphi' (w,z) = \varphi(z,w)$\\
	$u \in U$\\
	$ 0 = G(u) = \varphi(u,w) \rightarrow w \in U^\perp $\\
	$ 1 = G(u) = \varphi(u,w) \rightarrow u \notin (U^\perp)^\perp $
}
\subsection{Lemma:}
Se $ (v,w,\varphi) $ ein duales Paar $ U\subset V $ ein UVR\\
Dann ist $ (V/U, U^\perp, \tilde{\varphi}) $ duales Paar, wobei $ \tilde{\varphi}(v+u,w) = \varphi(u,w) $\\
Insbesondere gilt $$ \dim V = \dim U + \dim U^\perp $$
\subsection*{Bew:}
Übungsaufgabe
$$ (V/U,U^\perp,\varphi^\perp) \quad \tx{ ein duales Paar}$$
$$ \Downarrow$$
$$ \ub{\dim V/U}_{\dim V \  -\  \dim U} = \dim U^\perp $$

% neue Vorlesung 14.06.18

\subsection{wiederholung}
$$V, W \tx{ endlich, } (V,W,\varphi) \tx{ duales Paar }$$ 
$$\Leftrightarrow \dim V = \dim W = \tx{Rg}(\varphi)$$  
$$\Leftrightarrow F_\varphi: V \to W^* \tx{ Isom.}$$
$$v \mapsto F_\varphi(v),\ W \to K,\ w \mapsto \varphi(v,w)$$
$$\Leftrightarrow \varphi \tx{ nicht-ausgeartet ist: } \varphi(v,-): W \to K \tx{ die Null Abbildung} \Rightarrow v = 0$$
$$\varphi(-,w): V \to K  \tx{ die Null Abbildung} \Rightarrow w = 0$$
$$U \subset V\ \tx{UR},\ U^\perp = \{w \in W \setminus \varphi(u,w) = 0\}\ \tx{UR}$$
$$\rightarrow(U^\perp)^\perp = U \rightarrow \dim V = \dim U + \dim U^\perp$$
$$\mathcal B = \{b1 \dots b_n\}\tx{ Basis von V}$$
$$\tx{ist dual zu der Basis von W }\{c1 \dots c_n\} \tx{ falls } \varphi(b_i,c_j) = \delta_{ij}$$
\subsection{Def:}
Sei $(V,W,\varphi)$ duales Paar und $G:W \to W$ Endomorphismus.\\
Der adjungierte Endomorphismus $G^\top: V \to V$ wird definiert als $$G\top:F^{-1} \circ G^* \circ F_\varphi$$
\begin{center}
	\begin{tikzcd}
		V & \arrow{l}[swap]{F_{\varphi}^{-1}}  W^* & W \arrow{d}{G}\\
		V \arrow[mapsto]{u}{G^\top} \arrow{r}[swap]{F_{\varphi}}& W^* \arrow{u}{G^*} & W
	\end{tikzcd}
\end{center}
$ G^*(\Phi) = \Phi \circ G \qquad \Phi \circ ??? G (oder w ) \to K $
\subsection*{Bem:}
$G^*$ ist eindeutig bestimmt durch die Gleichung 
$$\varphi(G^\top(v),w) \buildrel (*) \over = \varphi(v,G(w))\qquad \ \forall v \in V,\ \forall w \in W$$
\Bew{Eindeutigkeit}{
	Falls $G:V\to V$ die selbe Eigenschaft erfüllt
	$$ \begin{array}{ccccc}
	\varphi(G_1(v) - G^* (v),w) = & \varphi(G(v),w) & - & \varphi (G^\top (v),w) \\
	& \verteq & &  \verteq \\
	& \varphi(v, G(w)) & - & \varphi(v,G(w)) & = 0
	\end{array}$$
	$\Rightarrow G_1(v) = G_1^\top \Rightarrow G_1=G^\top\ \forall v \in V,\ \forall w \in W\ \varphi$ nicht-ausgeartet 
	\paragraph{zu Zeigen:} $G^\top$ die Eigenschaft $(x)$ besitzt.\\
	Seien \underline{$v \in V,\ w \in W$} 
	$$\varphi(v,G(w)) = F_\varphi(v)(G(w))$$
	$$\varphi(v,G(w)) = \underbracket{\overbrace{F_\varphi(v)}^{\in W^*} \circ G}_{G^*(F\varphi(v))}(w)$$
	$$ \begin{array}{lc}
	G^\top(v)  = v_1 \\
	\ \ \verteq & \Leftrightarrow G^* \circ F_\varphi(v) = F_\varphi(w)\\
	F^{-1}_{\varphi}\circ G^* \circ F_\varphi(v)
	\end{array}$$
	$\rightarrow \forall w \in W:$
	$$\begin{array}{ccc}
	F_\varphi(v_1)(w) & = & \equaltoup{\varphi(v_1,w)}{\varphi(G^*(v),w)}\\
	\verteq \\
	\underbrace{G^* \circ F_\varphi(v)(w)}_{\in W^*} & = & \underline{\varphi(v,G(w))}
	\end{array}$$
}

\section{Euklidische Räume}
\subsection{Def:}
Eine Bilinearform $ \varphi: V \times V \to K$ ist symmetrisch, falls $ \varphi = \varphi' $ 
$$ \Leftrightarrow \forall u,v \in V : \varphi(u,v) = \varphi(v,u) $$
\subsection*{Bem:}
Seien $ \mathcal{B},\mathcal{C} $ Basen von V und A die Darstellungsmatrix vo $ \varphi $ bzgl. $ \mathcal{B},\mathcal{C} $.
$$ \varphi \tx{ symmetrisch } \Leftrightarrow A = A^\top = (\varphi(b_i,c_j))$$
\subsection{Bew:}
$\boxed{\Rightarrow}$ Siehe \underline{Übungsblatt}\\
$\boxed{\Leftarrow}$\\
\begin{align*}
\varphi(u,v) & = u^\top \cdot A \cdot v \\
& = (A^\top \cdot u)^\top \cdot v\\
& = v^\top \cdot \bigg( ( A^\top \cdot u)^\top \bigg)^\top\\
& = v^\top \cdot \equalto{A^\top}{A} \cdot u \qquad = \varphi(v,u) 
\end{align*}
\subsection*{Bem:}
$ \varphi $ symmetrisch\\
$ \Rightarrow $ Der Begriff der Orthogonalitt ist wohldefiniert und vor allem symmetrisch
\begin{center}
	$\begin{array}{c}
	u \perp v \Leftrightarrow \varphi(u,v) = 0\\
	\Updownarrow\\
	v \perp u \Leftrightarrow \varphi(v,u) = 0
	\end{array}$
\end{center}
\subsubsection{Beispiel}
$ \mathbb{R}^n $\\
$$\varphi(\vec{x},\vec{y}) = \sum_{i=1}^{n} x_i y_i \tx{ smmetrisch}$$ 
$$\varphi(\vec{x},\vec{x}) = \sum_{i=1}^{n} x_i^2 \ge 0$$
\subsection{Def:}
Sei $ \varphi: V \times V \to K $ symmetrische Bilinearform.\\
Die zugehörige quadratische Form:
\begin{align*}
g: V &\to K\\
v &\mapsto \varphi(v,v)
\end{align*}
\subsection*{Anmerkung:}
$ \dim V = n $\\
Sei $ \mathcal{B} = \{b_1,\dots,b_n\} $ eine Basis $ \rightarrow \varphi $ hat Darstellungsmatrix \underline{\underline{A}} bzgl. $ \mathcal{B} $.
\begin{align*}
q(v) & = \varphi(v,v) - \varphi(\sum\lambda_i b_i , \sum \lambda_j b_j)\\
& = \begin{pmatrix}
\lambda_1 & \dots & \lambda_n
\end{pmatrix} \cdot A \cdot \begin{pmatrix}
\lambda_1\\
\dots\\
\lambda_n
\end{pmatrix}\\
& = \sum a_{ij} \lambda_i \lambda_j
\end{align*}
\subsection*{Folgerung:}
(char $ K \neq 2 $)\\
Jede symmetrische Bilinearform ist durch ihre quadratische Form eindeutig \underline{bestimmt}.
\begin{align*}
q(u+v) & = \varphi(u+v,u+v)\\
& = \varphi(u,u) + \varphi(v,u) + \varphi(u,v) + \varphi(v,v)\\
& = \underbracket{\varphi(u,u)}_{q(u)} + \underbracket{\varphi(v,v)}_{q(v)} + 2 \varphi(u,v)\\
q(u-v) & = \dots = q(u) + q(v) - 2 \varphi(u,v)
\end{align*}
Insb:
$$ \varphi(u,v) = \frac{q(u+v) - q(u-v)}{4} $$


% nach pause
\subsection{Def:}
Sei $V$ ein endlich dimensionaler $\mathbb R$-VR. und $ \varphi:V\times V \to \mathbb R$ Symmetrische Bilinearform. Wir sagen,$\varphi$ ist:
\begin{enumerate}
	\item positiv semidefinit, falls $\varphi(u, u) \geq 0\ \forall u \in V$
	\item negativ semidefinit, falls $\varphi(u,u) \leq 0\ \forall u \in V$
	\item positiv definit, falls $\varphi$ pos. semidefinit ist und $\varphi(v,v) > 0\ \forall v \in V \setminus \{0\}$
	\item negativ definit, falls $\varphi$ neg. semidefinit ist und $\varphi(v,v) < 0\ \forall v \neq 0$
	\item indefinit \underline{sonst.}
\end{enumerate}
\paragraph{Beispiele:} 
\begin{enumerate}[(a)]
	\item Standard Skalarprodukt auf $\mathbb R^n$ ist positiv def.
	\item $\varphi: \mathbb R^2 \times \mathbb R^2 \to \mathbb R,\ ((x_1,y_1),(x_2,y_2)) \mapsto x_1 y_1$ ist positiv semidefinit aber nicht pos.def. $\varphi ((0,1),(0,1)) = 0$
	\item $\mathbb R^2 \times \mathbb R^2 \to \mathbb R,\ ((x_1,y_1),(x_2,y_2)) \mapsto x_1y_1  - x_2y_2$ indefinit
\end{enumerate}
\subsection*{Bem:}  
$$\varphi \tx{ ist posit (semi)-def., } -\varphi \tx{ ist neg (semi)-def.}$$
\subsection{Def:}
Ein Skalarprodukt auf einem endlichdimensionalen $ \mathbb{R} $-VR V ist eine positiv definite symmetrische Bilinearform: $ \varphi(u,v) \buildrel \tx{bijektion ???} \over \longrightarrow <u,v> $
\subsection{Def:}
$ (V,<,>) $ ist ein euklidischer Raum, wenn V ein $ \mathbb{R} $-VR endlichdimensional  und $ <,> $ ein Skalarprodukt ist.
\subsection{Def: Norm}
Eine Norm auf einem $\mathbb R$-VR $V$ ist eine Abbildung $||\cdot||: V \to \mathbb R$ derart, dass:
\begin{enumerate}
	\item $||v|| \geq 0,\ = 0 \Leftrightarrow v = 0$
	\item $||\lambda \cdot v|| = |\lambda|\ ||v||$
	\item Dreiecksungleichung $||u + v|| \leq ||u|| + ||v||$
\end{enumerate}
$(V, || \cdot ||)$ ein normierter $\mathbb R$-VR \underline{ist}
\paragraph{Beispiele:} $\mathbb R^n$
\begin{enumerate}[(a)]
	\item Euklidische Norm $||\vec{x}|| = \sqrt{x_1^2+ \dots + x_n^2}$
	\item $||\vec{x}||_{\infty \downarrow} = \sum |x_i|$
	\item $||\vec{x}||_{\infty \geq} = \underbrace{\tx{max}}_{1 \leq i \leq n}\ |x_i|$
\end{enumerate}
\subsection{Def:}
Sei $(V,<,>)$ ein euklidischer Raum und definiere $||\ ||: V \to \mathbb R,\\v \mapsto \sqrt{<v,v>}$ wohldefiniert. Wir wollen Zeigen, dass es eine Norm auf $V$ \underline{induziert}.
\subsection{Lemma}
(Cauchy-Schwarzsche-Ungleichung)
$$\forall v, w \in V$$
$$|<v,w>| \leq ||v|| \cdot ||w||$$
\Bew{}{
	Falls $w=0 \rightarrow$ ok!\\
	OBdA ist $ w \neq 0 \rightarrow ||w||>0 $\\
	Sei $  \lambda \in \mathbb{R} $ beliebig \\
	$$ 0 \le < v- \lambda w , v - \lambda w > = ||v||^2+ \lambda^2 ||w||^2 - 2 \lambda <v,w> $$\\
	Insb:
	$$ 2 \lambda <v,w> \le ||v||^2 + \lambda^2 ||w||^2 $$\\
	Falls $$ \lambda = \frac{<v,w>}{\ub{||w||^2}_{\neq 0}} \in \mathbb{R} $$
	$$ 2 \frac{<v,w>^2}{||w||^2} \le ||v||^2 + \frac{<v,w>^2}{||w||^2} \Rightarrow \frac{<v,w>^2}{||w||^2} \le ||v||^2 $$
	$$ \rightarrow <v,w>^2 \le ||v||^2 \cdot ||w||^2$$
	$$ \rightarrow |<v,w>| \le ||v|| \cdot ||w|| $$
}
\subsection{Folgerung:}
$ (V,<,>) $ euklidischer Raum\\
\begin{align*}
||\cdot ||: V & \to \mathbb{R}\\
v & \mapsto \sqrt{<v,w>}
\end{align*}
ist eine \underline{\underline{Norm}}.
\subsection{Bew:}
\begin{enumerate}[1)]
	\item klar
	\item $ ||\lambda \cdot v || = \sqrt{< \lambda v,  \lambda v>} = \sqrt{\lambda^2 <v,v>} = |\lambda| \sqrt{<v,v>} = |\lambda| \cdot ||v|| $
	\item Es genügt zu zeigen, dass $ ||u+v||^2 \le (||u|| + ||v||)^2 $\\
	\begin{align*}
	||u+v||^2 & = < u+v, u+v>\\
	& = < u,u> + <v,v> + 2 <u,v>\\
	& = ||u||^2 + ||v||^2 + 2 ||u||\, ||v||\\
	& = (||u|| + ||v||)^2
	\end{align*}
\end{enumerate}
\subsection{Def:}
$ (V,<,>) $ euklidischer Raum
$$ -1 \le \equalto{\ub{ \frac{<u,v>}{||u||\, ||v||} }}{\cos \theta} \le 1$$
$ \theta \in [0,\pi] $ eindeutig \underline{bestimmt}.\\
$ \theta $ ist der Winkel zwischen $ u $ und $ v $
\subsection*{Bem:}
$ u \perp v \Leftrightarrow <u,v> = 0 \Leftrightarrow \tx{ der Winkel } \frac{\pi}{2} \tx{\underline{ist}} $
$$ u \uparrow \rightarrow v$$

% neue Vorlesung

\subsection{Wiederholung}
$ \varphi : V \times V \to K $
symmetrisch gdw $ \varphi(u,v) = \varphi(v,u) $\\
$ \mathcal{B}= \{b_1,\dots, b_n\} $ eine Basis von V \\
$A = (\varphi(b_i,b_j)) $ Darstellungsmatrix von $ \varphi $\\
$ \varphi $ symmetrisch $ \Leftrightarrow  A = A^\top$\\
$ g(v) = g(\sum(\lambda_i b_i)) =\sum a_{ij} \lambda_i \lambda_j $\\
$ \tx{char } k \neq 2 \rightarrow q $ bestimmt $ \varphi $\\
$ \varphi(u,v) = \frac{q(u+v) - q(u-v)}{4} $\\
$ \varphi $ symmetrisch ist positiv definit, falls $ \varphi(u,u) = q(u) \ge 0 \leftarrow \tx{ Körper ist } \mathbb{R} $\\
$ q(u) = 0 \Rightarrow u = 0 $\\
Ein Skalarprodukt auf einen $ \mathbb{R} $-VR V ist eine symmetrische positiv definite Bilinearform
$ \varphi(u,v) \to <u,v> $\\
$ ||v||=\sqrt{q(v)} = \sqrt{<v,v>}$ Norm. \\
\begin{enumerate}[1)]
	\item $ ||v||\ge 0 ( = 0 \Leftrightarrow v = 0) $
	\item $ ||\lambda v || = | \lambda|\, ||v|| $
	\item $ ||v+w|| \le ||v|| + ||w|| $
\end{enumerate}
Cauchy-Schwarzsche Ungleichung $ |<v,v>| \le ||v|| \cdot ||w|| $

$$-1 < \equalto{\underbracket{\frac{<v,w>}{||v||\ ||w||}}}{ \cos \theta ,\ \theta \in [0, \pi]}<1$$
$\theta$ ist winkel zwischen $v$ und $w$.
\begin{trivlist}
	\item $v \perp w$ (orthagonal) 
	\item $\Leftrightarrow <v,w> = 0$
	\item $\Leftrightarrow$ Der Winkel ist $\frac{\pi}{2}$
\end{trivlist}

\section{Satz (Pythagoras)}
$$(V, <.>) \tx{ endlichdimensionaler Raum}$$
$v \perp w \Leftrightarrow ||v + w||^2 = ||v||^2 + ||w||^2$
%bild

\subsection{Bew:}
\begin{align*}
||v+w||^2 &=\  <v+w,v+w> \ \\ 
&= <v,v>+<w,w> + 2<v,w>\\
&= ||v||^2 + ||w||^2 + 2 <v,w> \\
&\tx{mit: } v\perp w \Leftrightarrow <v,w> = 0 \Leftrightarrow \\
&= ||v||^2 + ||w||^2
\end{align*}
\subsection{Def:}
$ \varphi:V\times V \to K $ symmetrische Bilinearform\\
Ein orthogonales System bzgl. $ \varphi $ ist eine Kollektion M von Vektoren $ 0 \notin M \ \ u,v \in M \tx{ verschieden } \varphi(u,v) = 0 $\\
Ein orthonormales System M ist eine Kollektion von Vektoren\\ $ \varphi(u,v) = \casess{0}{v\neq u}{1}{u=v} $

Dementsprechend definieren wir Othogonalbasis und \\ 
Orthonormalbasis (ONB)
\paragraph{Beispiel:}
Standardbasis $\{o_1, \dots o_k \}\ (\mathbb R^n, \underset{\scriptstyle}{<.>)}_{\tx{standard skalarprodukt}}$

\section{Satz}
(Char$(K) \neq 2$)\\
\indent Jede Symmetrische Bilinearform $\varphi$ auf einem endlich dimensionalen $K$-VR $V$ läßt sich bei einer geeigneten Basisauswahl durch eine Diagonalmatrix darstellen. Ferner ist $\varphi$ nicht-ausg. $\Leftrightarrow$ kein Eigenwert der Matrix null ist.

\subsection*{Bem:}
Es genügt zu zeigen, dass $ (V,\varphi) $ eine Basis $ \mathcal{B} = \{b_1,\dots, b_n\} $ besitzt welche aus paarweise orthogonalen Vektoren \underline{besteht}.\\
Dann ist die Darstellungsmatrix von $ \varphi $ bzgl. $ \{b_1,\dots , b_n\}$
$$\begin{pmatrix}
\varphi(b_1,b_1) & 0 & \dots & 0 \\
0 & \varphi(b_2,b_2) & \dots & 0 \\
\vdots& \vdots & \ddots & \vdots \\
0 & 0 &\dots & \varphi(b_n,b_n)
\end{pmatrix}$$
Sei $ q: V \to K \ v \mapsto \varphi(u,v) $ die zugehörige quadratische Form\\
Falls $ q(v) = 0 \ \forall v \in V $\\
$ \rightarrow \varphi(u,v) = 0 \ \forall u,v \in V $\\
$ \rightarrow $ Jede Basis von V besteht aus paarweise orthogonalen \underline{Vektoren}.

Sonst, existiert $b_1 \in V\setminus \equalto{g(b_1)}{\varphi(b_1, b_1)} \neq 0, \quad F:V \to K,\ v \mapsto \varphi(v,b_1)$
$$\custo{\Rightarrow}{\tx{Im}(F)}{\dim \ker (F) = n -1} = K \tx{ als \underline{$K$-VR}}$$
$$\ker (F) = \{v \in V \setminus \varphi(v_1,b_1=0\} = \{v \in V \setminus v \perp b_1\} = \tx{ Span}(b_1)^\perp$$
Induktion auf die Dimension von $V \Rightarrow$ Es existiert eine Basis $b_1 \dots b_n$ von $\ker(F)$ welches aus paarweise orthogonal Vektoren besteht.

Die Basis $ \{b_1, \dots , b_n\} $ ist eine orthogonale Basis von \underline{\underline{V}}.\\
\subsection*{Bem:}
Die Eigenwerte der Matrix hängen nicht von der Basis \underline{ab}\\
$ \rightarrow $ Eigenwerte sind $ \equalto{\varphi(b_1,b_1)}{\mu_1} , \dots , \equalto{\varphi(b_n,b_n)}{\mu_n} $

\boxed{\Rightarrow} Angenommen, dass 
$$\mu_j = \varphi(b_j,b_j) = 0$$
wäre
$$\left. \varphi(b_j,b_i) = \right \{ \begin{array}{ll}
0 \quad i \neq j\\
0 \quad i   =  j
\end{array}$$
$i$ \underline{beliebig}
$$\varphi(b_j,-):V \to K \tx{ ist die triviale Abbildung } \Rightarrow \tx{ Wid! } b_j \neq 0$$
\boxed{\Leftarrow} Sei $v \in V\setminus \{0\}$ beliebig.
\paragraph{Zu Zeigen:} $\varphi (v,-): V \to K$ nicht trivial ist
$$v = \summ{i=1}{n} \lambda_i b_i \rightarrow \exists i / \lambda_i \neq 0$$
$$\varphi(v,b_i) = \summ{j=1}{n} \lambda_j \varphi (b_j, b_i) = \overset{\scriptstyle\underset{\mkern4mu\rotatebox{90}{$\,\neq$}}{0}}{\lambda_i} \overset{\scriptstyle\underset{\mkern4mu\rotatebox{90}{$\,\neq$}}{0}}{\varphi(b_i,b_i)} \neq 0$$


\subsection{Kor:}
(char $ k \neq 2 $)\\
Falls in K jedes Element ein Quadrat ist, dann lässt sich jede symmetrische Bilinearform durch eine Matrix der Form
$ \begin{pmatrix}
1 \\
& \ddots \\
& & 1 \\
& & & 0 \\
& & & & \ddots \\
& & & & & 0
\end{pmatrix} $ darstellen.

\subsection{Bew:}
Es existiert eine Orthogonalbasis  $ \{b_1,\dots , b_n\} $ für $ \varphi:V \times V \to K $.
$$c_i = \casess{\frac{b_i}{\sqrt{\varphi(b_i,b_i)}}}{\varphi(b_i,b_i) \neq 0}{\quad \ b_i}{\quad \tx{sonst.}}$$
$ \{c_1,\dots , c_n \} $ ist immer noch eine Basis.\\


OBdA können wir annehmen, dass $ c_1,\dots,c_n $ ist so umgeordnet, dass
\begin{align*}
\varphi(c_i,c_i) &= 1  \qquad i \le k\\
\varphi(c_j,c_j) &= 0  \qquad j > k
\end{align*}
Darstellungsmatrix von $ \varphi $ bzgl. $ \{c_1,\dots , c_n\} $
$$ \begin{pmatrix}
1 & & & & & 0\\
& \ddots\\
& & 1\\
& & & 0\\
& & & & \ddots\\
0 & & & & & 0
\end{pmatrix} \begin{array}{l}
\\
\Bigg\} k \tx{ mal}\\
\\
\bigg\} n-k \tx{ mal}\\
\\
\end{array}$$

\section{Kor: (Sylvester)}
Jede Symm. Bilinearform $\varphi$ auf einem endlich dimensionalen $\mathbb R$-VR läßt sich bei geeigneter Basisauswahl durch eine Matrix der form \begin{center}
	\begin{tikzpicture}
	\matrix[matrix of math nodes,
	left delimiter=(,
	right delimiter=),
	nodes in empty cells] (m)
	{
		1 	&	&	&	&	&	&	&	& \\
		p =\quad & \ddots &	&	&	&	&	&	& \\
		&	& 1 &	&	&	&	&	& \\
		&	&	&	&	&	&	&	& \\
		&	&	& -1\ &	&	&	&	& \\
		&	& q = & & \ddots &	&	&	& \\
		&	&	&	&	& -1\ &	&	& \\
		&	&	&	&	&	&	&	& \\
		&	&	&	&	&	& 0\quad &	& \\
		&	&	&	&	&r =& 	& \ddots \quad & \\
		&	&	&	&	&	&	&   & 0 \ \\
	};
	\draw (m-1-1.north west) -- (m-1-1.south west) -- (m-3-3.south west) -- (m-3-3.south east);
	\draw (m-5-4.north west) -- (m-5-4.south west) -- (m-7-6.south west) -- (m-7-6.south east);
	\draw (m-9-7.north west) -- (m-9-7.south west) -- (m-11-9.south west) -- (m-11-9.south east);
	\end{tikzpicture}
\end{center}
darstellen.
Ferner hängen die Zahlen $p,g$ und $r$ nur bon $\varphi$ ab.
\Bew{}{Sei $\{b_1 \dots b_n\}$ eine Orthogonalbasis für $\varphi$
	$$ c := \left \{ 
	\begin{array}{ll}
	\frac{b_1}{\sqrt{\varphi(b_i,b_i)}} & \tx{falls } \varphi(b_i,b,i) > 0\\[15pt]
	\frac{b_i}{\sqrt{-\varphi(b_i,b_i)}} & \tx{falls } \varphi(b_i,b_i) < 0\\[15pt]
	b_i & \tx{\underline{sonst}}
	\end{array}\right.$$
	
	Nach Umordnen der Darstellungsmatrix  
	$$\begin{pmatrix}
	1\\
	& \ddots\\
	& & 1\\
	& & &1\\
	& & & & \ddots\\
	& & & & & 1\\
	& & & & & & 0\\
	& & & & & & & \ddots\\
	& & & & & & & & 0
	\end{pmatrix} \begin{array}{ll}
	\\
	\bigg\} p\\
	\\
	\\
	\bigg\} q\\
	\\
	\\
	\bigg\} r\\
	\\
	\end{array}$$
}

\subsection*{Bem:}
$ \varphi \upharpoonright \tx{Span}(c_1,\dots,c_p) \times \tx{Span}(c_1,\dots,c_p) $ ist positiv definit.

\subsection{Bew: falsch:}
$$\varphi\bigg( \sum_{i=1}^{p} \lambda_i c_i \sum_{i=1}^{p} \lambda_i c_i\bigg) = \sum_{i,j} \lambda_i \lambda_j \varphi(c_i,c_j) = \sum_{i=1}^{p} \lambda_i^2$$
Sei $ U\subset V $ der größte UR von V derart, dass $ \varphi \upharpoonright U \times U $ positiv definit ist.\\
$ \spa(c_1,\dots,c_p) \subset U $\\
zu zeigen $ \boxed{p = \dim U} \ \ \checkmark$ \\
Sonst $ U \cap \spa (c_{p+1}, \dots , c_n) \neq 0  $\\
Sei $$ 0 \neq  v \in U \cap \spa (c_{p+1},\dots , c_n) \quad v = \sum_{i=p+1}^{n} \lambda_i c_i$$
da $ v \in U \varphi\upharpoonright _{U\times U} $ positiv definit
$$ 0 \le \varphi(v,v) = \sum_{i=p+1}^{n} \lambda_i^2 \ob{\varphi(c_i,c_i)}^{\le 0} \le 0$$
$ \Rightarrow v = 0 \rightarrow $ \underline{Wid}.\\
$ g,r $ \underline{bestimmen}\\
$ \tx{Rg}(\varphi) = \underline{p} + q  \ \rightarrow $ ist eindeutig \underline{bestimmt}\\
$ r = n - \tx{Rg}(\varphi) \rightarrow $ eindeutig \underline{bestimmt}

\subsection{Def:}
Signatur $ (\varphi) = \underline{\underline{q-p}} $

% neun Vorlesung

%\begin{center}
%\begin{tabular}{|c|c|c|}
%	\hline
%	P I & & P II\\
%	\hline
%	Hohes Tempo& &Mündliche Prüfung \\
%	\hline	
%\end{tabular}
%\end{center}

\subsection{Wiederholung}
$ \varphi: \mathbb{R}^2 \times \mathbb{R}^2 \mapsto \mathbb{R} $\\
$ ((x_1,y_1), x_2,y_2) ) \mapsto 2 x_1x_2 - y_1 y_2 $\\
$ \varphi $ ist positiv definit auf $ \spa ((1,0)) \quad \varphi((\lambda,0),(lambda,0)) = 2 \lambda^2 \ge 0$\\
$ \varphi $ ist positiv definit auf $ \spa((1,1)) \quad \varphi((\lambda,\lambda),(\lambda,\lambda)) = \lambda^2 \ge 0 $\\
$ \varphi $ ist nicht positiv definit auf $ \spa((1,0)) + \spa((1,1)) = \mathbb{R}^2$\\
$\varphi((0,1),(0,1)) = -1 $

\subsection{Bew: Kor Sylvester}
Wir wollen p eindeutig bestimmen.\\
$ \varphi $ ist positiv definit auf $ \spa(c_1,\dots,c_p) $\\
Sie $$ h = \tx{max} \{\dim (U) / U \subset V \varphi\upharpoonright U \times U \tx{ pos. definit ist}\} $$
Sei $$ U \subset V \tx{ UR der Dimension h sodass } \varphi \upharpoonright U \times U \tx{ pos. definit ist.}$$
Wir zeigen: $ U \cap \spa (c_{p+1},\dots,c_n) = \{0\} $\\
$$ \rightarrow\quad  \equalto{\dim(U)}{0} + \equaltoup{n - p}{\dim \spa(c_{p+1,\dots,c_n})} = \ub{\dim \ub{U + \spa(c_{p+1},\dots,c_n)}_{\subset V}}_{\le n} $$
$ \Rightarrow h \le p  $

\chapter{Unitäre Räume}
\section{Def:} Ein unitärer Raum $V$ ist ein $\mathbb C$-VR zusammen mit einem komplexen Skalarprodukt $\langle\ ,\ \rangle : V \times V \mapsto \mathbb C$ mit folgeneden Eigenschaften
\begin{enumerate}
	\item $\langle v,w \rangle = \overline{\langle w,v \rangle},$ wobei $\overline{a+bi}=a-bi$
	\item $\langle v+ v', w \rangle = \langle v,w \rangle + \langle v',w \rangle$
	\item $\langle \lambda \cdot v, w \rangle = \lambda \langle v,w \rangle$
	\item $\langle v,v \rangle \in \mathbb R$ und ferner $\langle v,v \rangle > 0$ für $v \neq 0$ 
\end{enumerate}
\paragraph{Beispiel} $\mathbb C^n \qquad \langle \overline x, \overline y \rangle = \summ{i = 1}{n} x_i \overline{y_i}$

\subsection{Bem:}
Die Abblidung $ \langle\ ,\ \rangle $ ist nicht bilinear sondern \underline{hermitisch sesquilinear}
$$ \langle v,\lambda w + \mu w'\rangle = \overline{\lambda}\langle v,w \rangle + \overline{\mu} \langle v,w' \rangle$$
\subsection{Bew:}
\begin{align*}
\langle v,\lambda w + \mu w'\rangle &= \langle \overline{\lambda w + \mu w', v} \rangle \\
&= \overline{\lambda} \langle \overline{w,v}\rangle + \overline{mu}\langle \overline{w',v}\rangle  = \overline{\lambda}\langle v,w \rangle + \overline{mu}\langle v,w'\rangle 
\end{align*}
\subsection{Bem:}
Für einen unitären Raum V ist
$$ || \ \ ||: V \to \mathbb{R}$$
$$ \qquad \qquad \qquad \ \ \, v \mapsto \sqrt{<v,v>}$$
\begin{enumerate}[1)]
	\item $ ||v|| \ge 0 \quad = 0 \leftrightarrow v = 0 $
	\item $ || \lambda \cdot v || = | \lambda| \cdot ||v|| \quad \lambda \in \mathbb{C} $
	\item $ ||v+w|| \le ||v|| + ||w|| $
\end{enumerate}

\section*{Bem:}
$(V, \langle\ ,\ \rangle)$ unitärer Raum $v \perp w \Leftrightarrow \langle v,w \rangle = 0$
$$\left. \begin{array}{l}
\tx{Orthogonalität} \\ \tx{Orthogonales System} \\ \tx{Orthonormales System} \\ \tx{Orthogonalbasis}
\end{array} \right  \} \longrightarrow \tx{ Wir können die Begriffe hier \underline{verwenden}.}$$
\section{Def:} Sei $ ( V, \langle\ ,\ \rangle)$ ein euklidischer oder unitärer Raum. Eine Orthonormalbasis von $V$ ist eine Basis $\mathcal B = \{b_i\}\quad /\quad b_i \perp b_j \quad i \neq j \quad ||b_i|| = 1$
\section*{Bem:} Jedes orthoganles System ist linear unabhängig.
\Bew{}{$$\summ{i=1}{n}\lambda_i b_i = 0$$
	$$\equalto{\langle \sum \lambda_i b_i, b_j\rangle}{\summ{i=1}{n}\lambda_i \langle b_i, b_j \rangle = \lambda_i \custo{\neq}{\langle b_j,b_j \rangle}{0}} = 0$$}
\section{Lemma} Sei $(V, \langle\ ,\ \rangle)$ ein euklidischer oder unitärer Raum und $\{b_1, \dots b_n \}$ eine ONB. Dann
\begin{enumerate}
	\item $$\forall v \in V \qquad v = \summ{i=1}{n} \langle v,b_i \rangle b_i$$
	\item $$\begin{array}{c} v = \sum \lambda_i b_i \\ w = \sum \mu_i b_i \end{array} \Rightarrow \langle v,w \rangle = \sum \lambda_i \overline{\mu_i}$$
	\item $ F: V \to V $ Endomorphismus. Dann ist die Darstellungsmatrix $ A $ von $ F $ bzgl. $ \{b_1, \dots , b_n\} $ gegeben durch $ a_{ij} = \langle F(b_j),b_i \rangle $
\end{enumerate}

\subsection{Bew:}
\begin{enumerate}[1)]
	\item $ v = \sum\limits_{i=1}^{n} \lambda_i b_i \qquad \langle v_i,b_j \rangle = \equalto{\ub{\sum\limits_{i=1}^{n} \lambda_i \langle b_i, b_j \rangle }}{\lambda_j} $
	\item $ \langle \sum \lambda_i b_i, \sum \mu_j b_j \rangle = \sum_{i,j} \lambda_i \overline{\mu}_j \equaltoup{\overbrace{\langle b_i,b_j \rangle}}{\delta_{ij}} = \sum \lambda_i \overline{\mu}_i $
	\item  $ A = (\ub{a_{ij}}_{F(b_1) \dots F(b_n)}) $ $ a_{ij} $ ist die Koordinate von $ F(b_j) $ bzgl. $ b_i $\\
	$ \Downarrow $ 1)\\
	$ a_{ij} = \langle F(b_j),b_i \rangle $
\end{enumerate}
\subsection{Satz: (Geom-Schmidtsches\\ Orthonormalisierungsverfahren)}
Sei V ein euklidischer oder unitärer Raum. Gegeben $ \{v_1,\dots , v_n \} $ lin. unabh. dann gibt es ein orthonormalsystem $ \{e_1,\dots , e_n \} $ derart, dass
$$\spa(v_1,\dots,v_n) = \spa(e_1,\dots,e_n)$$
Insb., falls V endlichdim ist, besitzt V eine ONB.

\Bew{}{Zwei Schritte: 
\begin{itemize}
	\item Aus $v_1 \dots v_n$ konstruieren wir eine Basis welche aus orthogonalen Vektoren besteht
	\item Dann normalisieren: $e_1 \dots e_n$ werden rekursiv definiert
	$$e_1' = v_1 \buildrel{v_1 \neq 0}\over{\longrightarrow} e_1 = \frac{e_1'}{||e_1'||}$$
\end{itemize}

Angenommen $e_1 \dots e_n$ wurden konstruiert, so dass $$e_i \perp e_j \quad i \neq j \quad ||e_1'|| = 1$$ und $\spa (e_1 \dots e_n) = \spa (v_1 \dots v_n)$ Wir wollen
$$e_{k+1} = v_{k +1} - \summ{i=1}{k} \langle v_{k+1}, e_i \rangle e_i$$
	$e_{k+1}' \neq 0 \rightarrow$ sonst ist $v_{k+1} \in \spa (e_1, \dots, e_n) = \spa (v_1, \dots, v_n)$ Wid!}

Setze 
$$ e_{k+1} = \frac{e_{k+1}'}{||e'_{k+1}||}$$
zu zeigen:
$$e_{k+1} \perp e_{j_{j \le k}} :$$

\begin{align*}
< e_{k+1},e_j> &= \frac{1}{||e'_{k+1}||} < v_{k+1} - \sum_{i=1}^{k}< v_{k+1} , e_i > ,e_j>\\
&=\frac{1}{||e'_{k+1}||} \bigg( <v_{k+1},e_j> - \underbracket{\sum_{i=1}^{k} <v_{k+1},e_i> < e_i,e_j>  }_{<v_{k+1},e_j>} \bigg) = 0 \ \checkmark
\end{align*}

$$\spa (e_1 \dots e_{k + 1} = \spa (v_1 \dots v_{k + 1})$$
$$e_{k + 1} \in \spa ( v_{k + 1}, e_1 \dots e_k) = \spa(v_{k+1},v_1 \dots v_k)$$
$$v_{k+1} = ||e_{k+1}||\ e_{k+1}+\summ{j=1}{k} \langle v_{k+1}, e_j \rangle \in \spa (e_1 \dots e_{k+1})$$

\subsection{Kor:}
$ (V,\langle \ ,\ \rangle) $ euklidisch oder unitär endlichdimensional und $ D \subset V $ ein Orthonormalsystem. Dann $ \exists $ ONB $ B \supset D $ 
\subsection{Bem:}
Sei $ D = \{v_1,\dots,v_k \} $ und ergänze zu einer Basis $ \{v_1 \dots,v_n \} $ von V. $ \rightarrow $ Konstruiere eine ONB $ \{e_1,\dots,e_n\} $\\
zu zeigen: $ e_i = v_i \quad i \le k $\\
$$ e_1 = \frac{v_1}{||v_1|| =1} = v_1 $$
Annahme $ e_j \equiv v_j \qquad j \le i $\\
$$ e_{i+j} \frac{e'_{i+1}}{||e'_{i+1}||} = v_{i+1} \qquad e'_{i+1} = v _{i+1} - \sum_{j=1}^{i} \equalto{\underbracket{\langle v_{i+1},e_j \rangle }}{0} \equaltoup{e_j}{v_j}$$ 

% neue Vorlesung

%\subsection{Wiederholung}
%Unitärer Raum: $ \mathbb{C} $-VR V mit einem komplexen skalarprodukt 1) 2) 3)#
\subsection{Def:}$ (V,\langle\ ,\ \rangle) $ euklidisch und unitär\\
Zwei Teilmengen A,B von V sind Orthogonal, $ A \perp B $ falls $ v \perp w \quad \forall v \in A,\  \forall w \in B$
\subsubsection{Bem:}
$ A \perp B \Leftrightarrow \spa(A) \perp \spa(B) $\\
in Blatt 7 Aufgabe 3 schon gezeigt
\subsection{Def:}
$ A \subset B $ Teilmenge
\begin{align*}
A ^\perp &= \{ v \in V\ |\ \{v\} \perp A\}\\
&= \{ v \in V\ |\ \forall w \in A \quad v \perp W\}
\end{align*}
\subsubsection{Bem:}
$ A^\perp $ ist ein Unterraum von V
\subsubsection{Bem:}
$$ A^\perp = \spa(A)^\perp $$
\subsection{Bew:}
$ \boxed{\subset} $\\
$$ A^\perp \perp A \quad \quad \Downarrow$$
$$ A^\perp \subset \spa(A)^\perp \Leftarrow A^\perp \perp \spa(A)$$
$ \boxed{\supset} $\\
$$ v \in \spa(A)^\perp \Rightarrow \forall w \in \overbrace{\spa(A)}^{\supset A}$$
$$ v \perp w \Rightarrow \forall w \in A \quad v \perp w $$
$$ \Rightarrow v \in A^\perp$$
\subsection{Satz:}
Sei $ (V,\langle\ ,\ \rangle) $ euklidisch oder unitär und $ U \subset V $ endlichdim. UR
$$ \Rightarrow V = U \oplus U^\perp$$
\subsection{Bew:}
$$U \cap U^\perp = \{0\}$$
$ \langle\ ,\ \rangle $ nicht ausgeartet !\\
Es genügt zu zeigen, dass $ V = U + U^\perp $\\
\underline{1. Fall} $ U = \{0\} \rightarrow$ ok! \\
\underline{2. Fall} $ U \neq \{0\} $\\
$ \Downarrow U \tx{ endlichdim. Gram-Schmidt} $\\
$$ \exists \{ b_1,\dots, b_n \} \quad \tx{ ONB von } U$$
Sei $ v \in V $ beliebig.
$$ v = \ub{\sum_{i=1}^{k} \langle v, b_i \rangle b_i}_{ \in U } + \bigg( \overbrace{ v - \sum_{i=1}^{k} \langle v, b_i\rangle b_i }^{w} \bigg)$$
Wir müssen zeigen, dass $ w \in U^\perp $\\[5pt]
Es genügt, wenn wir zeigen,	dass $  w \perp b_i \quad 1 \le i \le k $
\begin{align*}
\langle w, b_i \rangle &= \langle v - \sum_{i=1}^{k} \langle v, b_i \rangle b_i , b_j \rangle\\
&= \langle v, b_i \rangle - \sum_{i=1}^{k} \overbrace{\langle v , b_i \rangle}^{ \in K } \equalto{\langle b_i,b_j \rangle}{\overbrace{\begin{array}{cc}
		0 & 1 \\
		i\neq j & i = j
		\end{array}}}\\
&= \langle v,b_i \rangle\qquad  -\qquad  \equaltoup{\langle v,b_j \rangle}{} = 0
\end{align*}
\subsection{Lemma:}
$ (V, \langle\ ,\ \rangle) $ euklidisch oder unitär\\
$ U \subset V $ endlichdim. UR
$$ (U^\perp)^\perp = U$$
\subsection{Bew:}
$ \boxed{\supset} $\\
$$ U \perp U^\perp \Rightarrow U \subset (U^\perp)^\perp $$
$ \boxed{\subset} $\\
Sei $  v \in (U^\perp)^\perp $\\
$$ \tx{ mit } V = U \oplus U^\perp \Rightarrow  v = u + w \quad u \in U \quad w \in U^\perp$$
Es genügt zu zeigen, dass $ w = 0 \Rightarrow v = u \in U $
$$ w = \ub{v}_{\in (U^\perp)^\perp} - \ub{u}_{\in U \subset (U^\perp)^\perp} \in \ub{(U^\perp)^\perp \cap U^\perp}_{= 0} \Rightarrow \tx{ ok !}$$
\subsection{Def:}
Sei $ U \subset V $ UR\\
Wir definieren die orthogonale Projektion von v auf U als den Vektor $ u \in U $ derart, dass $ v = u+\ub{w}_{\in U^\perp}  $
\subsubsection{Bem:}
Falls u existiert $ \Rightarrow $ ist er eindeutig bestimmt.
$$ \ub{u_2}_{ \in U} + \ub{w_2}_{ \in U^\perp} = v = \ub{u_1}_{\in U} + \ub{w_1}_{\in U^\perp} \ \Rightarrow \ u_1 - u_2 = w_2 - w_1 \in \ub{U \cap U^\perp}_{= \{0\}}$$
\subsubsection{Bem:}
Falls $ U \subset V $ endlichdim. $ \Rightarrow $ ist die Projektion auf U wohldefiniert\\
$ \forall v \in V $ existiert $ Pr|_{u}(v) $ derart, dass $ v = Pr|_{u}(v) + \ub{w}_{\in U^\perp} $
\subsection{Satz:}
Sei $ U \subset V $ Unterraum, $ v \in V \ u \in U $\\
sind folgende Aussagen äquivalent
\begin{enumerate}[a)]
	\item $ Pr|_{u}(v) = u $
	\item $ \forall u_1 \in U \setminus \{0\}  \quad || v - u_1 || > || v - u ||$
\end{enumerate}
\subsection{Bew:}
$\boxed{a) \Rightarrow b)} $ Es kommt.\\
$\boxed{b) \Rightarrow a)} \quad v = u + (v - u )$\\
Es genügt zu zeigen, dass $ v-u \in U^\perp $\\
Sonst, $ \exists u' \in U \ |\ \lambda = \langle v-u, u'\rangle \neq 0 \Rightarrow u' \neq 0 \Rightarrow \tx{ OBdA } ||u'|| = 1 $\\
Sei $ u + \lambda \cdot u' \in U \setminus \{u\} $\\
$$ || v-u ||^2 < || \equalto{\ub{v - (u + \lambda u')}}{} ||^2$$ 
$$ \langle v-u - \lambda u' , v-u - \lambda u' \rangle$$
$$ = ||v-u||^2 - \underbracket{\lambda \langle u',v-u\rangle}_{= \overline{\lambda}} - \underbracket{\overline{\lambda}\langle v-u,u'\rangle}_{ = \lambda} + \lambda \overline{\lambda} \underbracket{||u'||^2}_{= 1}$$
$$ = || v-u ||^2 - \underbracket{\lambda \overline{\lambda}}_{||\lambda||^2} - \lambda \overline{\lambda} + \lambda \overline{\lambda} < || v-u ||^2 \quad \tx{Wid !}$$
\section{Selbsadjungierte Endomorphismen\\ und Hauptachsentransformationen}
\subsubsection{Bem:}
Sei $ (V, \langle , \rangle) $ ein endlichsim. euklidischer Raum\\
$ \rightarrow (V,V,\langle,\rangle)$ ein duales \underline{Paar}.\\
Sei $ v\in V \quad \langle v, - \rangle : V \to \mathbb{R} $ ist nichttrivial, falls $ v \neq 0 $
$$ \langle v,v \rangle = ||v||^2 \neq 0\ , \ \tx{ falls } v \neq 0$$
Falls V endlichdim. unitärer Raum ist, dann ist 
$$ \langle v,-\rangle : V \to \mathbb{C} \quad \tx{ nichttrivial !!} $$
$$ \langle v, \lambda w + \mu w' \rangle \neq \lambda \langle v,w \rangle + \mu \langle v, w' \rangle$$
$$ \tx{aber } = \overline{\lambda} \langle v,w \rangle + \overline{\mu} \langle v, w'\rangle \qquad \quad $$
\underline{Lösung:}\\
Wir definieren eine neue Struktur auf V als $ \mathbb{C} $-VR
$$ \lambda *_{\tx{konj}} \cdot v = \overline{\lambda} \cdot v $$
Somit ist $ (V,V_{\tx{konj}}, \langle,\rangle) $ ein Duales \underline{Paar} weil 
$$ \langle v,v \rangle = ||v||^2 \neq 0 \tx{ für } v \neq 0$$
\underline{Folgerung:}\\
Sei V endlichdim. unitärer oder euklidischer Raum. Jeder Endomorphismus $ F: V \to V $ besitzt einen adjungierten Endomorphismus $ F^t: V \to V $, sodass $ \forall v,w \in V  $\\
$$ \langle v, F(w)\rangle = \langle F^t(v) , w \rangle$$
\underline{Alternative Beschreibung:}\\
Sei $ \mathcal{B}= \{b_1,\dots , b_n  \} $ eine ONB von V.\\
Seien $ v,w \in V \rightarrow w = \sum_{i=1}^{n} \equalto{\lambda _i}{\langle w, b_i \rangle} = b_i $
\begin{align*}
\langle v, F(w) \rangle &= \langle v , \sum_{i=1}^{n} \langle w, b_i \rangle b_i \rangle\\
&= \sum_{i=1}^{n} \overline{ \langle w, b_i \rangle } \langle v, F(b_i) \rangle \\
&= \sum_{i=1}^{n} \langle v, F(b_i) \rangle \langle bi , w \rangle\\
&= \langle \sum_{i=1}^{n} \langle v, F(b_i)\rangle \cdot b_i , w \rangle = \langle F^t, w \rangle\\
\end{align*}
$$ \Rightarrow F^t(v) = \sum_{i=1}^{n} \langle v, F(b_i)\rangle b_i$$
\subsection{Def:}
Sei $ K = \mathbb{R}$ oder $\mathbb{C} $\\
$$ a_{ij} = \mathcal{A} \in \mathcal{M}_{n\times n} (K) $$
Die adjungierte Matrix $ \mathcal{A}^* $ von $ \mathcal{A} $ ist die Matrix
$$ \mathcal{A}^* = ( \overline{a_{ji}})$$
\subsubsection{Bem:}
$ \mathcal{A} \in \mathcal{M}_{n \times n} (\mathbb{R}) $\\
$ \mathcal{A}^* = \mathcal{A}^t $\\
Leicht zu zeigen:\\
$ (\mathcal A^*)^* = \mathcal A $\\
$ (\lambda \cdot \mathcal{A})^* = \overline{\lambda} \cdot \mathcal{A}^* $\\
$ (\mathcal{A} + \mathcal{B})^* = \mathcal{A}^* + \mathcal{B}^* $\\
$ (\mathcal{A} \cdot \mathcal{B})^* = \mathcal{B}^* \cdot \mathcal{A}^* $

% neue vorlesung

\subsection*{Widerholung:}

$$(V, \langle\ \cdot\ \rangle) \tx{unitärer/euklidischer Raum}$$
$$A \subset V \tx{teilmenge}$$
$$A^\perp = \{v \in V\ |\ \forall w \in A\ \langle v,w \langle = 0\}$$
Unterraum
$$U \tx{ UR von } V \to U \cap U^\perp =\{0\}$$
Insbensondere falls $U$ endlich dim. $\Rightarrow V = U \oplus U^\perp$\\
Orhtogonale Projektion auf $U$ $$Pr|_U : V \to U,\ v \mapsto u\ / v = u + w \in U^\perp$$
$$\forall u \in U\ u_1 \neq Pr|_U(v) \Leftrightarrow ||v-u_1||> ||v-u||$$
$$F: \underbrace{V}_{\tx{\underline{endlichdim.}}} \to V \tx{ Endomorphismus,}$$ dann ex. 
$$F^\top : V \to V \tx{ adjungierter Endomorphismus}$$
$$\underline{\langle v, F(w) \rangle = \langle F^\top (v) ,w \rangle} \quad \forall v,w \in V$$

\subsection{Def:}
$ k= \mathbb{R} $ oder $ \mathbb{C} $\\
$ A = (a_{ij}) \in \mathcal{M}_{n\times n}(K) $\\
Die adjungierte Matrix $ A^* $ ist $ A^* = (\overline{a_{ji}}) $\\
$ K = \mathbb{R} \to A^* = A^\top $\\
$ (A \cdot B)^* = B^* \cdot A^* $\\
$ (\lambda \cdot A)^* = \overline{\lambda} \cdot A^* $\\
\subsubsection{Bem:}
Falls A regulär ist, dann ist $ \det(A^*) = \overline{\det(A)} $ weil $ \overline{a+b} = \overline{a} + \overline{b} \quad \overline{a \cdot b} = \overline{a} \cdot \overline{b} $
\subsection{Lemma:}
Sei $ \enph $ eines euklidischen bzw. unitären endlichdim. Raumes V und A die Darstellungsmatrix von F bzgl. der orthonormalen Basis $ \mathcal{B} = \basis{b}{1}{n} $ und $ \mathcal{D} = \basis{d}{1}{n} $\\
Dann hat $ F^\top: V \to V $ die Darstellungsmatrix $ A^* $ bzgl. $ \mathcal{D} $ und $ \mathcal{B} $.
\Bew{}{$$F^\top (d_i) = \summ{j=1}{n} \langle F^\top (d_i), b_j \rangle \cdot b_j$$
	$$F^\top \to C = (c \cdot j) \quad F^\top (d_j) = \sum c_{ij} b_j \quad \begin{pmatrix}
	c_{1j} \\ \vdots \\ c_{nj}\end{pmatrix}$$
	$$A= (c_{ij}) \to F(b_j) = \summ{i = 1}{n} \equalto{a_{ij}}{\langle F(b_j),d_i\rangle} d_i$$
	$$\overline{a_{ij}} = \langle \overline{F(b_j), d_i} \rangle = \langle d_i, F(b_j) \rangle \buildrel{\tx{Def. von }F^\top} \over = \underline{\langle F^\top (d_i),b_j \rangle} = c_{ji}$$
}
\subsection{Def:}
$ (V,\langle\ ,\ \rangle) $ endlichdim., euklidisch, unitär\\
$ F:V^\top $ ist \underline{normal}, falls
$$F \circ F^\top  F^\top \circ F $$ 
\subsection{Prop:}
Folgende Aussagen sind Äquivalent:
\begin{enumerate}[a)]
	\item $ F:V^\top $ ist normal
	\item $ \forall v, w \in V \quad \langle F(v) , F(w) \rangle = \langle F^\top(v) , F^\top(w) \rangle $
\end{enumerate}
\subsection{Bew:}
$ \boxed{a) \Rightarrow b)} $\\
\begin{align*}
\langle F(v) , F(w) \rangle &\buildrel \tx{Def von }F^\top \over = \langle F^\top(F(v)) , w \rangle \buildrel \tx{Normalität} \over =\\
&\quad \ \, = \quad \ \, \langle F(F^\top(v)) , w \rangle \\
&\quad \ \, = \quad \ \, \overline{\langle w , F(F^\top(v)) \rangle}\\
&\buildrel \tx{Def von } F^\top \over = \overline{\langle F^\top(w) , F^\top(v) \rangle } \\
&\quad \ \, = \quad \ \, \langle F^\top(v) , F^\top (w)\rangle
\end{align*}
$ \boxed{b) \Rightarrow a)} $\\
Zu Zeigen: $ F \circ F^\top(v) = F^\top \circ F(v)  \quad \forall v \in V $\\
Oder Äquivalent dazu, dass $$ F \circ F^\top(v) - F^\top \circ F(v) = 0 $$
\begin{align*}
\Leftrightarrow \forall w \in V \quad &\langle F \circ F^\top(v) - F^\top \circ F(v) , w \rangle = 0\\
\Leftrightarrow \qquad\qquad \ \;  &\langle F \circ F^\top (v) , w \rangle = \langle F^\top \circ F(v) , w \rangle
\end{align*}

$$\begin{array}{ccc}
\langle F^\top \circ F(v), w \rangle &=& \langle F(v) , F(w) \rangle \\ \rotatebox{90}{$\,=$}& & \rotatebox{90}{$\,=$}\\ \langle F \circ F^\top (v) , w \rangle & & \langle F^top (v) , F^\top (w) \rangle \\ \rotatebox{90}{$\,=$} & & \rotatebox{90}{$\,=$} \\
\langle \overline{ w, F \circ F^\top (v)}\rangle & = & \langle \overline{F^\top (w), F^\top (v)} \rangle
\end{array}$$

\subsubsection*{Bem:}  $F: V \to V$ 
$$\tx{ mit Darstellungsmatrix $A$ bzg. der ONB } \mathcal B =\basis{b}{1}{n} \tx{ von } \underline{V}$$
$$F^\top : V \to V \tx{ hat Darstellungsmatrix } A^*$$

$$\begin{array}{ccc}
F^\top \circ F & \to & A^* \cdot A \\ \rotatebox{90}{$\,=$}& & \rotatebox{90}{$\,=$}\\ F \circ F^\top & \to & A \cdot A^* 
\end{array}$$

\subsection{Def:}
Eine Matrix $A \in \mathcal M_{n \times n} (K)$ ist normal, falls $A^* \cdot A = A \cdot A^*$

\subsubsection{Folgerung:}
F ist normal gdw. F eine normale Darstellungsmatrix bzgl JEDER ONB \underline{besitz}.
\subsection{Lemma:}
$ \enph $ normal.\\
$ v \in V $ ist ein Eigenvektor von F bzgl $ \lambda $ gdw v ein Eigenvektor von $ F^\top $ bzgl. $ \overline{\lambda} $
\subsection{Bew:}
der Abstand von $ F(v) $ und $ \lambda \cdot v $
\begin{align*}
 || F(v) - \lambda \cdot v ||^2  &= \langle  F(v) - \lambda v, F(v) - \lambda v \rangle\\
 &= \equalto{\langle F(v), F(v) \rangle }{\langle F^\top(v) , F^\top(v)}  - \overline{\lambda} \equalto{\langle F(v) , v \rangle}{\langle F^\top(v),v \rangle} - \lambda \equalto{\langle v, F(v) \rangle}{\langle F^\top(v),v \rangle} + \lambda \overline{\lambda} \langle v, v \rangle\\
 &= \langle F^\top(v) , F^\top(v) \rangle - \overline{\lambda} \overline{\langle F^\top(v) , v \rangle} - \langle F^\top(v) , \overline{\lambda} \cdot v \rangle + \langle \overline{\lambda} v , \overline{\lambda} v \rangle\\
 &= \langle F^\top(v), F^\top(v) \rangle - \overline{\lambda} \langle v, F^\top(v) \rangle - \langle F^\top(v) , \overline{\lambda}v\rangle + \langle \overline{\lambda} v , \overline{\lambda} v \rangle\\
 &= \langle F^\top(v) - \overline{\lambda}v , F^\top(v) - \overline{\lambda} v \rangle \\
 &= || F^\top(v) - \overline{\lambda} \cdot v ||^2 
\end{align*}

\subsection{Satz:}
Sei $F: V \to >V $ derart, dass $\chi_F(T)$ in Linearfaktoren zerfällt. Folgende Aussagen sind äquivalent:
\begin{enumerate}[a)]
	\item $F$ ist normal
	\item Es existiert eine ONB $ \mathcal B =\basis{b}{1}{n}$ von $V$, welche aus Eigenvektoren von $F$ besteht.
\end{enumerate}
\subsubsection{Folg}
Jeder dnormaler Endomorphismus eines unitärem/euklidischen Raumes ist diagonalisierbar

\Bew{(satz)}{
	$\boxed{a) \Rightarrow b)}$ Induktion auf dim$(V)$ 
	$$\underline{\dim V = 1} \rightarrow F \tx{ besitzt einen Eigenvektor } o \neq v \tx{ zum Eigenwert } \lambda$$
	$$||v|| \neq 0 \rightarrow \frac{v}{||v||} \tx{ ist auch ein Eigenvektor tu }\lambda$$
	$$\bigg \{\frac{v }{||v||} \bigg \} \tx{ ist eine ONB von }V!$$
	$\boxed{\dim V \geq 2}$
	$$\chi_F (T) \tx{ zerfällt in Linearfaktoren } (T- \lambda_1) \dots (T-\lambda_n)$$
	Sei $\custo{\neq \ }{b_i}{\: 0} \in V$ Eigenvektoren zum Eigenwert $\lambda_1$.\\[10pt]
	OBdA $\underline{||b_i|| = 187}$ $$V = \spa (b_i) \oplus \equalto{U}{\spa (b_i)^\perp} \qquad \dim U = \underline{n-1}$$
}

\subsection{Beh:}
U ist $F$-invariant\\
Beweis: Nächste Woche 
$$F \upharpoonright_U : U \to U \quad \tx{ ist ein Endomorphismus}$$
\subsection{Beh:}
U ist $ F^\top $-invariant und:
$$ (F\upharpoonright_U)^\top = F^\top \upharpoonright_U$$
\subsection{Bew:}
Wenn U $ F^\top $-invariant ist wie oben:\\
$$ \forall v_1,v_2 \in U \quad \langle u_1,\equalto{F(u_2)}{F\upharpoonright_U(u_2)} \rangle = \langle \equalto{F^\top(u_1)}{F^\top\upharpoonright_U(u_1)} , u_2 \rangle $$
$$\Downarrow$$
Aus der Eindeutigkeit von adjungierten Endomorphismus  folgt\\
$ F^\top \upharpoonright_U = (F\upharpoonright_U)^\top $\\
Insb ist $ F\upharpoonright_U $ normal 
$$\Downarrow \tx{ Induktion}$$
$ \exists \basis{b}{2}{n} $ eine ONB von U welche aus Eigenvektoren von $ F\upharpoonright _U $ besteht.\\
$ \{ b_1, b_2 , \dots , b_n \} $ ist eine ONB von Eigenvektoren.\\
$\boxed{b) \Rightarrow a)}$ Sei $\mathcal B = \basis{b}{1}{n}$ eine ONB von Eigenvektoren von $F$
$$F(b_i) = \lambda_i \cdot b_i$$
Setze $G: V \to V,\ b_i \mapsto \overline{\lambda_i}\cdot b_i\rightarrow G$ ist eindeutig \underline{bestimmt}

\begin{align*}
G \circ F ( b_i) &= G(\lambda_i b_i) = \lambda_i G (b_i)\\
&= \lambda_i \overline{\lambda_i} b_i = \overline{\lambda_i} (\lambda_i b_i)\\
&= \overline{\lambda_i}(F(b-i)) = F(\overline{\lambda_i}b_i) = F(G(b_i))
\end{align*}

$$\Rightarrow G \circ F = F \circ G \tx{ \underline{auf $V$}}$$

Aber $ G = F^\top $ !! weil $ b_i $ Eigenvektor von $ F^\top $ zum Eigenwert $ \overline{\lambda_i} $\underline{ist}.
$$ \Downarrow $$
$$F^\top(b_i) = \overline{\lambda_i} b_i = G(b_i) \Rightarrow G = F^\top$$ 

% neue vorlesung

\subsubsection{Widerholung:}
$ \enph $ endlichdim. euklidisch, unitär\\ $\ska{F^\top(v)}{w} = \ska{v}{F^\top(w)} $
$ \exists F^\top: V \to V $ adjungierter Endomorphismus zu F, derart:\\
$ \mathcal{B} = \basis{b}{1}{n} $ ONB von $ V \to F^\top $ hat Darstellungsmatrix \\
$ A = a_{ij} $ Darstellungsmatrix von F bzgl. $ \mathcal{B} \qquad A^* = (\overline{a_{ji}})$ bzgl. $\mathcal{B}$\\
$F$ ist normal, falls $$F \circ F^\top = F^\top \circ F$$
$$\Leftrightarrow A \cdot A^* = A^* \cdot A$$
$$\Leftrightarrow \ska{F(v)}{F(w)} = \ska{F^\top (v)}{F^\top (w)}$$
$$\longrightarrow v \in V \setminus \{0\} \tx{ ist Eigenvektor von $F$ bzg. } \lambda $$
$$\Leftrightarrow v \tx{ Eigenvektor von } F^\top \tx{ bzg. } \overline{\lambda}$$

\subsection{Satz} $ \enph\  | \  \chi_F(T) $ zerfällt in Linearfaktoren\\
F ist normal $ \Leftrightarrow $ Es existiert eine ONB von V, welche aus Eigenvektoren von F besteht.\\\\
ende Wiederholung.

\subsection{Def:}$ F_V\to V $ ist selbstadjungiert, falls $ F = F^\top $ oder äquivalent dazu,
$$\ska{v}{F(w)} = \ska{F(v)}{w} \quad \forall v,w \in V$$
Eine Matrix $ A \in \mathcal{M}_{n\times n} (K) $ ist hermitesch, falls $ A^* = A $
\subsubsection{Wid:}
$V$ euklidischer untärer endlichedim. Raum $F: V \to V$ selbst adjungiert falls $F = F^*$\\ 
$F \tx{ selbstadj. } \Leftrightarrow$ Die darstellungsmatrix bzg.$\begin{pmatrix}
\tx{einer} \\ \tx{jeder}
\end{pmatrix}$ ONB hermitesch bzw. symmetrisch ist.\\
$A$ symmetrisch reelle $(n \times n)$-Matrix $\rightarrow$ $\chi_A$ zerfällt in Linearfaktoren über $\underline{\underline{\mathbb R}}$
$A$ ist dann diagonalisierbar $\rightarrow$ Hauptachsentransformationssatz.\\ 
$V \tx{ euklidischer endlidim. Raum}$, $\varphi: V \times V \to \mathbb R \tx{ symmetrische bilinearform}$
$\rightarrow$ Hauptachsen von $\varphi$ sind die Elemente einer ONB von $V$ sodass $\varphi$ Darstellungsmatrix in Diagonalform besitzt.
\subsection{Kor: Spektralsatz}
$F:V\to V$ selbstadjungiert $\rightarrow \exists \tx{ ONB von } V$ welches aus Eigenvektoren von $F$ besteht.
\Bew{}{$V$ unitär $\rightarrow$ ok, weil $\chi_F$ in Linearfaktoren über $\mathbb C$ zerfällt\\
	$V$ eiklidisch $\rightarrow$ Die Darstellungsmatrix von $F$ ist symmetrisch $\rightarrow$ diagonalisierbar $\rightarrow$ $\chi_F$ zerfällt in Linearfaktoren über $\mathbb R$ 
}

\subsubsection{Bem}
Falls $K = \mathbb R$ ist , dann ist hermitesch = symmetrisch
\subsubsection{Bem}
Sei $\mathcal B$ eine ONB von $V$
$$F: V \to V \tx{ ist selbstadjungiert}$$
$\Leftrightarrow$ Die Darstellungsmatrix von $F$ bzg. $\mathcal B$ hermitesch ist.

\subsection{Bem:}
Selbstadjungierte Endomorphismen sind normal.
\subsection{Lemma:}
Falls V euklidisch, unitär, endlichdim. ist $ F: V\to V $ selbstadjungiert, dann gilt:
\begin{enumerate}[a)]
	\item Alle Eigenwerte von F reell sind
	\item Eigenvektoren zu verschiedenen Eigenwerten orthogonal sind
\end{enumerate}

\Bew{}{
\begin{enumerate}[a)]
	\item Sei $\lambda \in \mathbb C$ Eigenwert von $F$ und besteht $v \in V \setminus \{0\}\ / \ F(v) = \lambda \cdot v$
	\begin{align*}
 		\equalto{\ska{\lambda \cdot v}{v}}{\lambda \cdot ||v||^2 \neq 0} = \ska{F(v)}{v}& \buildrel{\ F = F^\top} \over = \ska{v}{F(v)} = \equalto{\ska{v}{\lambda \cdot v}}{\overline{\lambda} \cdot ||v||^2}\\
 		&\Rightarrow \ska{v}{w} = 0
	\end{align*}
	\item Seien $\underbrace{\lambda \neq \mu}_{\rightarrow \in \mathbb R}$ derart, dass $F(v) = \lambda \cdot v \quad F(w) = \mu \cdot w \quad v,w \in V \setminus \{0\}$
	$$\equalto{\ska{v}{\mu \cdot w}}{\mu \ska{v}{w} \Rightarrow \ska{v}{w} = 0} = \ska{v}{F(w)} \buildrel{\ F = F^\top} \over = \ska{F(v)}{w}= \equalto{\ska{\lambda \cdot v}{w}}{\lambda \ska{v}{w}}$$
\end{enumerate}}

\subsection{Satz: Hauptachsentransformation}
Sei V ein euklidischer endlichdim. Raum und $ \varphi: V \times V \to \mathbb{R} $ eine symmetrische Bilinearform. Es existiert eine ONB  $ \basis{b}{1}{n} $ von V derart, dass $ \varphi $ durch eine Diagonalmatrix bzgl $ \basis{b}{1}{n} $ dargestellt wird.

\subsubsection*{\underline{Zuerst zwei Hilfslemmata:}}

\subsection{Lemma 1:}
$ A \in \mathcal{M}_{n\times n}(\mathbb{C}) $ hermitesch\\
$ \rightarrow $ A ist Diagonalisierbar und alle Eigenwerte sind reell
\subsection{Bew:}
$ A $ hermitesch, $ \chi_A(T) $ zerfällt in Linearfaktoren über $ \mathbb{C} $ $ \rightarrow F_A: \mathbb{C}^n \to \mathbb{C}^n$, \\ $\vec{x} \mapsto A\vec{x}$ $ F_A $ ist selbstadjungiert und normal $ \Rightarrow $ Alle Eigenwerte sind reell
\subsection{Lemma 2:}
Jede symmetrische $ n\times n $ Matrix über $ \mathbb{R} $ ist diagonalisierbar.
\subsection{Bew:}
Wir betrachten A als Matrix über $ \mathbb{C} $. A ist hermitesch $ \rightarrow $ A über $ \mathbb{C} $ diagonalisierbar. Alle Eigenwerte von A sind reell $ \Rightarrow $ A ist über $ \mathbb{R} $ diagonalisierbar

\Bew{Hauptachsentransformationssatz}{
Sei $\basis{e}{1}{n}$ ONB von $V$.\\
Definiere: $$F:V \to V, \quad e_i \mapsto \summ{j = 1}{n} \varphi (e_i, e_j) \cdot e_j$$
$$\ska{e_i}{F(e_i)}= \ska{e_i}{\summ{n=1}{n} \varphi (e_j, e_n) \cdot e_n} = \summ{k=1}{n} \varphi ( e_j, e_n) \ska{e_i}{e_k} = \varphi (e_j,e_i)$$
$v = \sum \lambda_i e_i \quad w = \sum \mu_j e_j$ beliebig.\\
Zu Zeigen: $\ska{F(v)}{w}= \ska{v}{F(w)} \Rightarrow F$ ist selbstadj.

\begin{align*}
\ska{v}{F(w)} &= \ska{\sum_{i=1}^{n} \lambda_i e_i}{\sum_{j=1}^{n} \mu_j F(e_j)}\\
&= \sum_{i=1}^{n} \sum_{j=1}^{n} \lambda_i \mu_i \ub{\ska{e_i}{F(e_j)}}_{\varphi(e_i,e_j)}\\
&= \sum_{i=1}^{n} \lambda_i \varphi(e_i, \ub{\sum_{j=1}^{n} \mu_j e_j}_{w}) = \varphi(\ub{\sum_{i=1}^{n} \lambda_i e_i , w}_{v}) = \varphi(v,w)
\end{align*}
\begin{equation*}
\equalto{\ska{v}{F(w)}}{\custo{\Rightarrow}{\ska{F^\top(v)}{w}}{\custo{\Rightarrow}{F = F^\top}{F \tx{ ist \underline{selbsadjungiert}}}}} = \equalto{\varphi(v,w)}{\varphi(w,v) \tx{ da } \varphi \tx{ symm.}} 
\end{equation*}
\begin{equation*}
\ska{F(v)}{w} = \ska{w}{F(v)}
\end{equation*}
Die Darstellungsmatrix von F bzgl. $ \basis{e}{1}{n} $ ist \underline{symmetrisch}
Aus dem lemma 2 folgt, dass $F$ diagonalisierbar ist $\rightarrow$ $\chi_F$ zerfällt in Linearfaktoren\\ 
$\Downarrow F$ selbstadj.\\
$\exists \basis{b}{1}{n}$ ONB von Eigenvektoren von $F$\\
Es genügt zu zeigen, dass die Darstellungsmatrix von $\varphi$ bzg. $\basis{b}{1}{n}$ in Diagonalform ist.
$$\varphi(b_i, b_j) = \ska{b_i}{\equalto{F(b_j)}{\lambda_j b_j}} = \lambda_j\ska{b_i}{b_j} = \casess{\lambda_i}{i = j}{0}{\underline{\tx{sonst}}}$$
}

\subsection{Korollar:}
Jede symmetrische $ (n\times n) $ Matrix über $ \mathbb{R} $ ist diagonalisierbar.\\
Jede hermitesche Matrix über $ \mathbb{C} $ ist diagonalisierbar.

\subsubsection*{\underline{Zurück zu positiv definite Matrizen}}

$$ A \in \mathcal{M}_{n \times n}(\mathbb{R}) \tx{ symmetrisch }\qquad v^\top \cdot A \cdot v \ge 0 \ \ = 0 \tx{ gdw } v = \vec{0} $$
$$\begin{pmatrix}
a & b \\
c & d
\end{pmatrix} \quad a > 0 \quad ac - b^2 > 0$$

\subsection{Satz: Sylvester}
Eine symmetrische $ (n \times n) $-Matrix A über $ \mathbb{R} $ ist genau dann positiv definit, wenn alle Eigenwerte von A echt positiv sind.

\Bew{}{\boxed{\Rightarrow} Sei $\ska{\ }{\ }$ das Standartskalarprodukt auf $\mathbb R$ und betrachte die von $A$ definierte Bilinearform $\varphi: \mathbb R^n \times \mathbb R^n \to \mathbb R, \quad (v,w) \mapsto v^\top A w$ Aus dem Hauptachsentransformationssatz folgt, dass es eine ONB $\basis{b}{1}{n}$ diagonalisierbar ist.
$$ S^{-1} \cdot A \cdot S = \Bigg(
\begin{array}{ccc}
\equaltoup{\lambda_1}{\varphi(b_1,b_1)} \\
& \ddots \\
& & \equalto{\lambda_n}{\varphi(b_n,b_n)}
\end{array} \Bigg)$$ $$\Rightarrow \lambda_1,\dots,\lambda_n \ \tx{ sind die Eigenwerte von A}$$
Insb. ist $ 0 < \varphi(b_i,b_i) = \lambda_i $ weil $ b_i \neq 0 $\\\\

\noindent $ \boxed{\Leftarrow} $
Sei $ \basis{b}{1}{n} $ eine ONB, do dass A bzgl. $ \basis{b}{1}{n} $ in Diagonalform ist:
$$S^{-1} \cdot A \cdot S = \begin{pmatrix}
\lambda_i > 0\\
& \ddots \\
& & \lambda_n >0
\end{pmatrix}$$
$$v \rightarrow \sum_{i=1}^{n} \mu_i b_i$$
$$\begin{pmatrix}
\mu_1 & \dots & \mu_n
\end{pmatrix} \begin{pmatrix}
S^{-1} A S
\end{pmatrix} \begin{pmatrix}
\mu_1\\
\vdots\\
\mu_n
\end{pmatrix}$$ $$= \sum_{i=1}^{n} \lambda_i \mu_i^2 \ge 0  \ \ = 0 \Rightarrow \mu_1 = \dots = \mu_n = 0 \Rightarrow v = 0$$
}
\subsubsection{Korrolar (Sylvester)}
Für eine symm. $(n \times n)$-$\mathcal M$atrix $\equaltoup{A}{(a_{ij})}$ über $\mathbb R$ sind folgende Aussagen äquivalent:
\begin{itemize}
	\item $A$ ist positiv definit
	\item Alle Eigenwete von $A$ sind echt positiv
	\item Alle Hauptminoren
	$$\det \begin{array}{| ccc |}
	a_{11} & \cdots & a_{1k}\\
	\vdots & & \vdots\\
	a_{k1} & \cdots & a_{kk}	
	\end{array} \ > \ 0 \qquad \forall\  1 \le v \le n$$
\end{itemize}

% neue Vorlesung


\subsubsection{Wiederholung:}
$V$ euklidischer untärer endlichedim. Raum $F: V \to V$ selbst adjungiert falls $F = F^*$\\ 
$F \tx{ selbstadj. } \Leftrightarrow$ Die darstellungsmatrix bzg.$\begin{pmatrix}
\tx{einer} \\ \tx{jeder}
\end{pmatrix}$ ONB hermitesch bzw. symmetrisch ist.\\
$A$ symmetrisch reelle $(n \times n)$-Matrix $\rightarrow$ $\chi_A$ zerfällt in Linearfaktoren über $\underline{\underline{\mathbb R}}$
$A$ ist dann diagonalisierbar $\rightarrow$ Hauptachsentransformationssatz.\\ 
$V \tx{ euklidischer endlidim. Raum}$, $\varphi: V \times V \to \mathbb R \tx{ symmetrische bilinearform}$
$\rightarrow$ Hauptachsen von $\varphi$ sind die Elemente einer ONB von $V$ sodass $\varphi$ Darstellungsmatrix in Diagonalform besitzt.
\subsection{Kor: Spektralsatz}
$F:V\to V$ selbstadjungiert $\rightarrow \exists \tx{ ONB von } V$ welches aus Eigenvektoren von $F$ besteht.
\Bew{}{$V$ unitär $\rightarrow$ ok, weil $\chi_F$ in Linearfaktoren über $\mathbb C$ zerfällt\\
	$V$ euklidisch $\rightarrow$ Die Darstellungsmatrix von $F$ ist symmetrisch $\rightarrow$ diagonalisierbar $\rightarrow$ $\chi_F$ zerfällt in Linearfaktoren über $\mathbb R$ 

$ A \in \mathcal{M}_{n \times n} (\mathbb{R}) $ symmetrisch
$$ A \tx{ positiv definit } \Leftrightarrow v^\top \cdot A \cdot v \ge 0 \, \ \ = 0 \tx{ gdw } v = 0$$
$$ A \tx{ positiv defnit } \Leftrightarrow \tx{ Alle Eigenwerte von } A \tx{ sind echt positiv}$$
}
\subsection{Bew: Satz von Sylvester}
$ \boxed{1 \Leftrightarrow 2} $ ok!\\
$ \boxed{1 \Rightarrow 3} $ Die Bilinearform $ \varphi: \mathbb{R}^n \times \mathbb{R}^n \to \mathbb{R}\ ,\ (v,w) \mapsto v^\top A w $ ist positiv definit
$$\det(A) = \prod_{i=1}^{n} \ub{\lambda_i}_{\tx{Eigenwerte}} > 0$$
$\forall\ U \subset \mathbb{R}$ UR $ \rightarrow \varphi \upharpoonright_U : U \times U \to \mathbb{R} $ auch positiv definit.\\[5pt]
Sei $ U = \spa(e_1, \dots , e_k) $
$$\varphi\upharpoonright_U: U \times U \to \mathbb{R} \ , \ (v,w)  \mapsto v^\top \cdot \begin{pmatrix}
a_{11} & \dots & a_{1k}\\
\vdots & & \vdots \\
a _{k1} & \dots & a_{kk}
\end{pmatrix} \cdot w$$
$$\Rightarrow \det\begin{vmatrix}
a_{11} & \dots & a_{1k}\\
\vdots & & \vdots \\
a _{k1} & \dots & a_{kk}
\end{vmatrix} > 0 $$
$ \boxed{3 \Rightarrow 1} $ Induktion auf n\\
$\underline{n=1}$ ,$ A = (\lambda) \to \lambda > 0 $
$$\left.\begin{array}{cc}
x^\top \lambda x = \lambda x^2 \ge 0 \phantom{\Leftrightarrow x = 0}\\
\phantom{x^\top \lambda x  = \lambda x^2 \  } = 0 \Leftrightarrow x = 0
\end{array} \right\} \rightarrow \substack{A \tx{ ist }\\\tx{positiv} \\\tx{defnit}}$$
Für $A = \spa (e_1 \dots e_{n+1}$ Die Darstellungmatrix von $\varphi_{\upharpoonright U}$ hat auch alle Hauptminoren exht positiv $\Rightarrow \varphi_{\upharpoonright U}$ \underline{positiv definit}\\
$\buildrel{\tx{Hauptachsen.}} \over \longrightarrow$ Es gibt eine Orthonormalbasis $\basis{b}{1}{n}$ von $V$ sodass $\varphi$ Diagonalform bzg. $\basis{b}{1}{n}$ hat 
$$\Bigg(
\begin{array}{ccc}
\equaltoup{\lambda_1}{\varphi(b_1,b_1)} \\
& \ddots \\
& & \equalto{\lambda_n}{\varphi(b_n,b_n)}
\end{array} \Bigg)$$
Falls $ \varphi $ nicht positiv definit wäre, gäbe es: $ \equalto{\lambda_i}{\varphi(b_i,b_i)} < 0 $
$$ \Downarrow \det(A) = \prod \lambda_i > 0$$
$$ \exists j \neq i \qquad \equalto{\lambda_i}{\varphi(b_i,b_i)} < 0$$
$\left. \begin{array}{l}
b_i = \summ{k=1}{n} \mu_k^i \epsilon_k \\ b_j = \summ{k=1}{n} \mu_k^j e_k
\end{array} \right \} \exists \alpha, \beta \in \mathbb R $ nicht beide Null,\\[10pt]
sodass $v = \alpha b_i + \beta b_j \in \equalto{\spa (e_1 \dots e_{n+1})}{U}$ 
$$\mu'_n = 0 \rightarrow \begin{array}{c}
\alpha = 1 \\ \beta = 0
\end{array}$$
$$0 = \mu^j_n \tx{ genau so sonst} \begin{array}{c}
\alpha = \mu_n^j \\ \beta = \mu_n^i
\end{array}$$
$$b_i \rightarrow (\mu_1^i,\dots \mu_n^i)$$
$$b_j \rightarrow (\mu_1^j,\dots \mu_n^j)$$
$$\alpha b_i + \beta b_j \rightarrow (\dots., \alpha \mu^i_n + \beta \mu_n^j)$$
$$ \varphi(v,v) = \ub{( \xi_1, \dots , \xi_{n-1},0) \cdot \begin{pmatrix}
	a_{11} & \dots & a_{1n-1}& 0\\
	\vdots & & \vdots & 0\\
	a_{n-1 1} & \dots & a_{n-1 n-1} & 0\\
	0 & 0 & 0 & 0
	\end{pmatrix}\cdot \begin{pmatrix}
	\xi_1\\
	\vdots\\
	\xi_n-1\\
	0
	\end{pmatrix}}_{\ge 0}$$
Weil $ \varphi \upharpoonright_U $ positiv aus der Induktion
$$ \varphi(v,v) = \varphi( \alpha b_i + \beta b_j , \alpha b_i + \beta b_j) = \ub{ \alpha^2 \lambda _i + \lambda \beta^2 }_{< 0} + 0$$
$$\Rightarrow \tx{ Widerspruch ! }$$

\section{Orthogonale Abbildungen und Drehungen}
\subsection{Def:} $V$ euklidischer bzw. unitär. $F:V \to V$ ist orthogonal, falls $\forall v,w \in V:\ \ska{v}{w} = \ska{F(v)}{F(w)}$
\subsection{Lemma:}
Folgende Aussagen sind äquivalent: 
\begin{enumerate}[1.)]
	\item $F:V \to V $ orthogonal
	\item $\forall v \in V \quad ||v|| = ||F(v)||$
	\item Für jedes Orthonormalessystem $\basis{e}{1}{k}$ ist $\{F(e_1) \dots F(e_k)\}$ auch ein Orthonormalessystem
\end{enumerate}

\subsection{Bew:}
$ \boxed{1 \Rightarrow 2} $ 
$$ ||v||^2 = \ska{v}{v} = \ska{F(v)}{F(v)} = ||F(v)||^2$$
$ \boxed{2 \Rightarrow 3} $ Es genügt zu zeigen, dass $ F(e_i) \perp (F(e_j) \quad \forall\ i \neq j $\\
$ \boxed{1^{\tx{er}}\tx{ Fall:}} $ V ist euklidisch
\begin{align*}
\ska{F(e_i + e_j)}{F(e_i + e_j)} = || F(e_i + e_j) ||^2 = ||e_i + e_j ||^2 = \ska{e_i + e_j}{e_i + e_j} = 2
\end{align*}
\begin{align*}
&\ska{F(e_i + e_j)}{F(e_i + e_j)}\\
&= \equalto{\ska{F(e_i)}{F(e_i)}}{\equalto{||F(e_i)||^2}{1}} + \equalto{\ska{F(e_j)}{F(e_j)}}{\equalto{||F(e_i)||^2}{1}} + \ska{F(e_i)}{F(e_j)} + \equalto{\ska{F(e_j)}{F(e_i)}}{\ska{F(e_i)}{F(e_j)}}\\[5pt]
&\tx{Insb: } \custo{\neq}{2}{\, 0} \ska{F(e_i)}{F(e_j)} = 0 \quad \Rightarrow \quad F(e_i) \perp F(e_j) 
\end{align*}
$\boxed{2^{\tx{er}} \tx{ Fall:}}$
$$\underline{V \tx{ unitär}}$$
$$\equalto{||e_i + \equaltoup{i}{F_1} e_j ||^2}{2\,  = \underbrace{||e_i||^2}_{1} + \underbrace{||ie_j||^2}_{1}} = \equalto{||F(e_i + ie_j)||^2}{\ska{\equalto{F(e_i) + i F(e_j)}{}}{F(e_i) + F(e_j)}}$$
$$\underbrace{||F(e_i)||^2}_{1} + \underbrace{||F(e_j)||^2}_{1} +\,  i\,  (\ska{F(e_j)}{F(e_i)}-\ska{F(e_i)}{Fe_j)}$$
$$\ska{F(e_i)}{F(e_j)} = \equalto{\ska{F(e_j)}{F(e_i)}}{}$$
Wir machen wie im ersten Teil $\quad \Leftarrow \quad \overline{\ska{F(e_i)}{F(e_j)}} \in \mathbb R$\\[10pt]
$ \boxed{3 \Rightarrow 1}  \ \ v,w \in V$\\
Zu Zeigen: $ \ska{v}{w} = \ska{F(v)}{F(w)} $\\
$ V = 0 \rightarrow $ ok!\\
Sonst, $ v \neq 0 \rightarrow 1^{\tx{er}} $ Fall $ \{v,w\} $ lin. abh. $ \buildrel v \neq 0 \over \rightarrow \lambda v \quad \lambda \in \mathbb{K} $ \\
$ v \neq 0 \rightarrow \frac{v}{||v||} $ orthonormal $ \Rightarrow F(\frac{v}{||v||}) $ auch\\
$ || F(\frac{v}{||v||})|| = 1 \Rightarrow ||v|| = ||F(v)||$
\begin{align*}
\ska{v}{w} &= \ska{v}{\lambda v} = \overline{\lambda} \ska{v}{v}\\
&= \overline{\lambda} ||v||^2 = \overline{\lambda} ||F(v)||^2\\ 
&= \overline{\lambda} \ska{F(v)}{F(v)} = \ska{F(v)}{\lambda F(v)}\\
&= \ska{F(v)}{F(w)}
\end{align*}
$\boxed{2^{\tx{er}} \tx{ Fall:}} \quad $  v,w lin. unabh.\\[5pt]
$ G - S \rightarrow \exists \ b_1, b_2 $ orthonormalsystem $ \spa(v,w) = \spa(b_1, b_2) $
\begin{align*}
v &= \lambda_1 b_1 + \lambda_2 b_2\\
w &= \mu_1 b_1 + \mu_2 b_2
\end{align*}

\begin{equation*}
\ska{v}{w} = \equalto{\ub{ \lambda_1 \overline{\mu_1} \equalto{||b_1||^2}{||F(b_1)||^2} + \lambda_2 \overline{\mu_2} \equalto{||b_2||^2}{||F(b_2)||^2}} }{\ska{F(\lambda_1 b_1 + \lambda_2 b_2)}{F(\mu_1 b_1 ) + F(\mu_2 b_2)}}
\end{equation*}
(letzter schritt weil $ F(b_1) \perp F(b_2) $) 

% neue vorlesung

\subsection{Wiederholung:}
Orthogonale Endlómorphismes $ F: V \to V \quad \forall v,w \in V$ V euklidisch oder unitär \\
$ \ska{v}{w} = \ska{F(v)}{F(w)} $\\
F orthogonal gdw $ ||v|| = ||F(v) || \  \forall v \in V $ dann bildet F Orthonormalsysterme zu Orthonormalsysteme ab \\
\underline{Frage:}\\
$ F: \mathbb{R} \to \mathbb{R}\quad x \mapsto 2 \cdot x $
$ \rightarrow $ Orthogonale Abbildungen (für euklidische Räume) erhalten den Winkel zwischen Vektoren

\subsection{Satz:}
Jede orthogonale Abbildung ist injektiv.
\subsection{Bew:} $ F:V\to V \quad v \in \ker(F)$
$$0 = F(v) \rightarrow 0 = ||F(v)|| = ||v|| \rightarrow v = 0$$
\subsection{Satz:} Sei V endlichdim. euklidischer oder unitärer Raum und $ F: V\to V $ eine Bijektive lineare Abbildung.\\
Folgende Aussagen sind dann äquivalent:
\begin{enumerate}[1)]
	\item F it orthogonal
	\item Der adjunkte Endomorphismus $ F^\top $ von $ F $ ist $ F^{-1} $
\end{enumerate}

\subsection{Bew:}
$ \boxed{1 \Rightarrow 2} $ \\
$ \forall v,w \in V: $\\
%$$\ska{\equalto{w}{\equalto{\ub{F(F^{-1}(w))}_{=}}}{}}{F(v)} = \ska{F^\top(w)}{v}$$
$ \leftarrow F \tx{ orthogonal} $\\
$$\ska{F^{-1}(w)}{v}  \buildrel \substack{\tx{Eindeutigkeit}\\\tx{von } F^\top} \over \Rightarrow  F^{-1} = F^\top $$
$ \boxed{2 \Rightarrow 1} $ \\
F \underline{orthogonal}\\
$$\ska{F(v)}{F(w)} \buildrel \tx{Definition} F^\top \over = \ska{F^\top(F(v))}{w} \buildrel F^\top = F^{-1} \over = \ska{F^{-1}(F(v))}{w} = \ska{v}{w}$$

\subsection{Kor:}
$ F: V\to V $ endlichdim. euklidisch oder unitäre orthogonale Abbildung \\
$ \Rightarrow $ F ist bijektiv und normal, mit $ F^{-1} = F^\top $

\subsection{Bew:}
F orthogonal $ \Rightarrow $ F injektiv $ \buildrel \tx{dim}(V) < \infty \over \Rightarrow $ F bijektiv ist $ \Rightarrow $ $ F^\top = F^{-1} $
$$F \circ F^\top = F \circ F^{-1} = F^{-1} \circ F = F^\top \circ F \Rightarrow \quad \tx{ normal}$$

\subsection{Satz:}
Sei V endlichdim, euklidisch oder unitär $ F:V\to V $ orthogonale Abbildung. Dann haben alle Eigenwerte von F Absolutbetrag 1. Form ist $ \det(F) = 1 $

\subsection{Bew:}
Beachte, dass 0 kein Eigenwert von D ist. (weil F injektiv ist !)\\
Sei $ \lambda \in K $ ein Eigenwert von F und $ v \in V \setminus \{0\} $ ein Eigenwert zu $ \lambda $
$$ \begin{array}{cc}
\equalto{||F(v)||}{} = ||v|| \neq 0\\
\ ||\lambda v || \ = | \lambda | \ ||v||
\end{array} \rightarrow |\lambda| = 1$$
Sei A die Darstellungsmatrix von F bzgl. der ONB
\begin{align*}
|\det(F)| &= |\det(A)| = |\overline{\det(A)}| = |\det(\overline{A})| = |\det(\overline{A^\top})| \\
&= | \det(\ub{\overline{A^\top}}_{A^*})| \buildrel F \tx{ orthogonal} \over = |\det(F^\top)| = |\det(F^{-1})| = | \det(F)|^{-1}
\end{align*}
$ F\circ F^{-1} = Id_{IV} $\\
$$|\det(F)|^2 = 1 \Rightarrow |\det(F)| = 1$$

\subsubsection{Bem:}
Sei B eine ONB von V\\
$A \ \ \;$  die Darstellungsmatrix von $F \ \ \; $ bzgl. B\\
$ A^* \ \:$ die Darstellungsmatrix von $ F^\top \ $ bzgl. B\\
$ A^{-1} $ die Darstellungsmatrix von $ F^{-1} $ bzgl. B
$$F^{-1} = F^\top \Leftrightarrow A^{-1} = A^*$$

\subsection{Def:}
Sei $A \in \mathcal M_{(n \times n)} (K) \qquad K = \mathbb{R} \tx{ oder } \mathbb{C}$ regulär 
$$A \tx{ ist Orthogonal, falls } A^{-1} = A^*$$
\subsubsection{Bem:}
$F$ ist orthogonal $\Leftrightarrow$ Die Darstellungsmatrix von $F$ bzgl. $\begin{pmatrix}
\tx{einer} \\ \tx{jeder}
\end{pmatrix}$ ONB orthogonal
\subsection{Kor:}
$ A \in \mathcal{M}_{n \times n} (K) $ regülär ist orthogonal $ \Leftrightarrow $ die  Zeilenvektoren (bzw die Spaltenvektoren) ein Orthonormalsystem in $ K^n $ bilden (und somit eine ONB)
\subsection{Bew:}
$$ A \tx{ orthogonal } \Leftrightarrow A^{-1} = A^* \Leftrightarrow A \cdot A^* = E_n = A \cdot A^{-1}$$ $$\Leftrightarrow \equalto{\ub{\sum_{k=1}^{n} a_{ik} \cdot \overline{a_{jk}}}}{\ska{\vec{a}_i}{\vec{a}_j} \in K^n} = \casess{1}{i=j}{0}{i\neq j} $$
$$\ska{\ub{\vec{a}_i}_{\tx{i-te Zeile}}}{\ub{\vec{a}_j}_{\tx{j-te Zeile}}} \in K^n \tx{ mit dem Standartskalarprodukt } $$
\subsection{Kor: zu irgendeinem Satz}
Sei $ F: V \to V $ normale Abbildung eines endlichdimensionalen unitären Raumes V derart, dass alle Eigenwerte von F Absolutbetrag 1 haben. Dann ist F orthogonal.

\subsection{Bew:}
$$ \chi_{F}(T) \in \mathbb{C} [T] \tx{ zerfällt in Linearfaktoren}$$
$$\Downarrow F \tx{ normal}$$
Es existiert eine ONB $ \basis{b}{1}{n} $ von Eigenvektoren von F $ F(b_i) = \lambda \cdot b_i $\\[5pt]
F hat Darstellungsmatrix:
$\begin{pmatrix}
\lambda_1\\
& \ddots \\
& & \lambda_n
\end{pmatrix} $
bzgl $ B = \basis{b}{1}{n} $\\[5pt]
$$\Rightarrow 0 \neq |\det(F)| = | \prod  \lambda_i | = \prod | \lambda_i| = 1$$
F hat Darstellungsmatrix:
$\begin{pmatrix}
\overline{\lambda_1}\\
& \ddots \\
& & \overline{\lambda_n}
\end{pmatrix}$
bzgl $ B ) \basis{b}{1}{n} $\\[5pt]
$$\lambda_i \cdot \overline{\lambda_i} = |\lambda_i|^2 = 1 \Rightarrow \overline{\lambda_i} = \lambda_i^{-1}$$
$ F^\top $ $ (= F^{-1}) $ hat Darstellungsmatrix: $ \begin{pmatrix}
\frac{1}{\lambda_1} \\
& \ddots \\
& & \frac{1}{\lambda_n}
\end{pmatrix} $ bzgl $ B = \basis{b}{1}{n} $ 
$$\Rightarrow F^{-1} = F^\top \rightarrow F \tx{ ist orthogonal}$$

\subsection{Def:}
Eine Matrix $ A \in \mathcal{M}_{n \times n}(K) $ ist orthogonal diagonalisierbar, falls es eine orthogonale (reguläre) Matrix $ S $ gibt, sodass $ S^{-1} A S  \ (= S^* A S)$ in Diagonalform ist.

\subsection{Prop:}
Jede normale Matrix $ A \in \mathcal{M}_{n \times n}(K) $ deren charakteristisches Polynom in Linearfaktoren zerfällt, ist orthogonal diagonalisierbar.

\subsection{Bew:}
Sei $ F_A: K^n \to K^n $ die zugehörige lineare Abbildung $ \Rightarrow  F_A $ ist normal $ \Rightarrow \chi_A(T) $ zerfällt in Linearfaktoren.\\
Es gibt eine ONB $ \basis{b}{1}{n} $ von Eigenvektoren von F(d.h. von A)\\
Sei $ S = (b_1| \dots | b_n) \qquad b_i$ als Spaltenvektoren
$$S^{-1} A S = \dmat{\lambda}{1}{n} \quad \tx{in Diagonalform}$$
Es genügt zu zeigen, dass S eine orthogonale Matrix ist.\\
Aber die Spaltenvektoren von S bilden ein orthonormales System.\\
$$\Rightarrow S \tx{ ist orthogonal} $$

\subsection{Kor:}
Jede symmetrische $ n \times n $ Matrix über $ \mathbb{R} $ bzw. jede hermitesche Matrix über $ \mathbb{C} $ ist orthogonal diagonalisierbar.

\subsection{Bew:}
A ist normal, $ \chi_A(T) $ zerfällt in Linearfaktoren.

\subsection{Def: Drehung}
Sei V ein endlichdim euklidischer Raum.\\
$ F: V \to V $ orthogonale Abbildung ist eine \underline{\underline{Drehung}}, falls $ \det(F) = 1 $.

\subsubsection{Bem:}
Die Kollektion aller Drehungen bilden eine \underline{\underline{Gruppe}}.

\subsection{Satz:}
Sei $ \mathbb{R}^2 $ mit dem Standartskalarprodukt als euklidischer Raum.\\
$F: \mathbb{R}^2 \to \mathbb{R}^2$ ist eine Drehung, gdw F Darstellungsmatrix bzgl $ \{e_1,e_2\} $ der Form:
$$\begin{pmatrix}
\cos \alpha & - \sin \alpha \\
\sin \alpha & \cos \alpha
\end{pmatrix}$$
hat, für ein $ \alpha \in [0, 2\pi] $. Wobei $ \alpha $ der Winkel der Drehung ist.
\underline{Bsp:}

$$ \alpha = \frac{\pi}{2} :\qquad \begin{tikzpicture}
\draw[->] (0,0) to [out=90,in=0] (-.5,.5);
\end{tikzpicture} = \begin{pmatrix}
0 & -1\\
1 & 0
\end{pmatrix}\qquad 
\begin{array}{ccccc}
\begin{pmatrix}
1\\0
\end{pmatrix}& (\rightarrow) & \buildrel \begin{tikzpicture}
\draw[->] (0,0) to [out=90,in=0] (-.25,.25);
\end{tikzpicture} \over \longrightarrow & \begin{pmatrix}
0\\1
\end{pmatrix}& (\uparrow)\\[10pt]
\begin{pmatrix}
0\\1
\end{pmatrix}& (\uparrow) & \buildrel \begin{tikzpicture}
\draw[->] (0,0) to [out=90,in=0] (-.25,.25);
\end{tikzpicture} \over \longrightarrow & \begin{pmatrix}
-1\\0
\end{pmatrix}& (\leftarrow)
\end{array}$$

% neue Vorlesung
% wud auf bild evtl nachtragen

\subsection{Wid:}
V  euklidischer Raum
$$\tx{Dreh}(V) = \{F: V \to V \tx{ Drehung}\}$$
Drehung $ \leftrightarrow $ orthogonale Abbildung mit $ \det(F) = 1 $\\
eins \underline{Gruppe}

\Bew{Satz}{
$\boxed{ \Leftarrow }$
$$\det \begin{pmatrix}
\cos \alpha & - \sin \alpha\\
\sin \alpha & \cos \alpha
\end{pmatrix} = 1$$
$\boxed{\Rightarrow}$ Sei $ \{e_1,e_2\} $ die kanonische Basis $ A = (a_{ij})_{1 \le i,j \le 2} $ und die Darstellungsmatrix von F:
$$\phantom{A = } F(e_1) \qquad \ \  F(e_2)$$
$$A = \begin{pmatrix}
\ska{F(e_1)}{e_1} & \ska{F(e_2)}{e_1}\\
\ska{F(e_1)}{e_2} & \ska{F(e_2)}{2_2}
\end{pmatrix}$$
Die Spaltenvektoren bilden ein orthonormales System.\pagebreak
$$||\equalto{F(e_1)}{}||^2 = 1$$\\[-30pt]
$$\ska{F(e_1)}{e_1}^2 + \ska{F(e_1)}{e_2}^2$$
$ \Rightarrow \exists \alpha \in [0,2 \pi] \quad \cos \alpha = \ska{F(e_1)}{e_1} \quad \sin \alpha = \ska{F(e_1)}{e_2} $
$$\left. \begin{array}{cc}
\det(A) = 1\\
0 = \ska{F(e_1)}{e_2}
\end{array}\right\} \begin{array}{cc}
\rightarrow & 1 = \cos \alpha \ska{F(e_2)}{e_2} - \sin \alpha \ska{F(e_2)}{e_1}\\
\rightarrow & 0 = \cos \alpha \ska{F(e_2)}{e_1} + \sin \alpha \ska{F(e_2)}{e_2}
\end{array}$$
$$\begin{pmatrix}
\cos \alpha & \sin \alpha\\
- \sin \alpha & \cos \alpha
\end{pmatrix} \begin{pmatrix}
\ska{F(e_2)}{e_1}\\
\ska{F(e_2)}{e_2}
\end{pmatrix} = \begin{pmatrix}
0\\1
\end{pmatrix}$$
$$\begin{pmatrix}
\ska{F(e_2)}{e_1}\\
\ska{F(e_2)}{e_2}
\end{pmatrix} = \begin{pmatrix}
\cos \alpha & \sin \alpha\\
- \sin \alpha & \cos \alpha
\end{pmatrix} \begin{pmatrix}
0\\1
\end{pmatrix} = \begin{pmatrix}
- \sin \alpha\\
\cos \alpha
\end{pmatrix}$$
}

\subsection{Satz:}
$ F : \mathbb{R}^3 \to \mathbb{R}^3 $ ist genau dann eine Drehung, wenn F Darstellungsmatrix bzgl. einer geeigneten ONB $ \{b_1,b_2,b_3\} $ der Form
$$\begin{pmatrix}
1 & 0 & 0 \\
0 & \cos \alpha & - \sin \alpha\\
0 & \sin \alpha & \cos \alpha
\end{pmatrix} \quad \tx{ hat.} \qquad \qquad \alpha \in [0,2 \pi]$$ 

\Bew{}{
$ \boxed{\Leftarrow}\  \checkmark $\\
$ \boxed{\Rightarrow} $ Sei A die Darstellungsmatrix von F bzgl. der kanonischen Basis $ \{a_1,a_2,a_3\} $
$$ \Rightarrow \tx{ A ist orthogonal}$$
$ \Rightarrow $ Als Matrix über $ \mathbb{C} $ haben alle Eigenwerte $ \lambda_1,\lambda_2,\lambda_3 $ von A Absolutbetrag 1.
$$|\lambda_1| = |\lambda_2| = |\lambda_3| = 1$$
$ \chi_F(T) = \chi_A(T) $ ist ein \underline{normiertes} Polynom Grades 3 über \underline{\underline{$ \mathbb{R} $}}\\
$ \rightarrow $ Es muss eine Nullstelle in $ \mathbb{R} $ haben $ \rightarrow $ OBdA $ \lambda_1 \in \mathbb{R} $\\[5pt]
$ |\lambda_1| = 1 \Rightarrow \lambda_1 = \pm 1 $\\
$  1 = \det(A) = \lambda_1 \lambda_2 \lambda_3 $\\
\textbf{1er Fall:} $ \lambda_2, \lambda_3 \in \mathbb{R} \rightarrow $ Für ein i muss $\lambda_i = 1 \ \Rightarrow $ OBdA $ \lambda_1 = 1 $\\
\textbf{2er Fall:} Sonst. $ \lambda_2 \in \mathbb{C/R} $
$$\chi_F(T) = (T-\lambda_1) (T-\lambda_2) (T-\lambda_3)$$
hat Koeffizienten in $ \mathbb{R} $\\[5pt]
Insb. $ \overline{\lambda_2} = \left\{ \begin{array}{cc}
\cancel{\bcancel{\lambda_1}}\\ \lambda_3
\end{array} \right. $
nicht $ \lambda_1 $ da es Reell ist.\\
$ |\lambda_2| = |\lambda_3| = 1 $\\
$ \lambda_2 = e^{i\alpha} = \cos \alpha + i \sin \alpha \quad \lambda_3 = \overline{\lambda_2} = \cos \alpha - i \sin \alpha = e^{-i\alpha} = \frac{1}{\lambda_2} $\\
$ \det(F) = 1 = \lambda_1 \cdot \ub{ \lambda_2 \cdot (\lambda_2)^{-1}}_{= 1} $\\
$ \Rightarrow \lambda_1 = 1 $\\[5pt]
Sei $ b_1 \in \mathbb{R}^3 $ ein Eigenvektor zum Wert \underline{$ \lambda_1 = 1 $}\\
Sei $ U = \spa (b_1)^\perp \rightarrow\dim(U) = 2 $\\[10pt]
\textbf{Frage:} $ F(U) \perp \spa(b_1) $ ?\\
$$\Leftrightarrow F(U) \subset U ?$$
$$ u \in U \qquad \equalto{\ska{F(u)}{\equalto{b_1}{F(b_1)}}}{} \Rightarrow F(u) \in U $$
gleich da F orthogonal
$$ \ska{u}{b_1} = 0 $$
Aus dem Gram-Schmidt'schen Verfahren wähle eine ONB $ \{b_2,b_3\} $ von U. (aus dem Eulersatz) $ \Rightarrow \{b_1,b_2,b_3\} $ ist eine ONB von $ \mathbb{R}^3 $.\\
$ F\upharpoonright_U $ hat auch eine Drehung ! $ \begin{pmatrix}
1 & 0 & 0\\
0 & * & *\\
0 & * & *
\end{pmatrix} $\\
$ \rightarrow $ F hat Darstellungsmatrix bzgl. $ \{b_1,b_2,b_3\} $ ist $ \begin{pmatrix}
1 & 0 & 0\\
0 & \cos \alpha & - \sin \alpha\\
0 & \sin \alpha & \cos \alpha 
\end{pmatrix} $
}

\section{Multilineare Algebra}
Sei K ein beliebiger Körper
\subsection{Def:}
Das Tensorprodukt von zwei K-VR U und V ist ein K-VR T zusammen mit einer (universellen) bilinearen Abbildung: $ \otimes: U \times U \to T $ derart, dass jede bilineare Abbildung $ g: U \times V \to W $ sich schreiben läßt als eine Komposition.
\begin{center}
	\begin{tikzpicture}[baseline= (a).base]
	\node[scale=1.3] (a) at (0,0){
		\begin{tikzcd}
		\arrow{r}{\otimes} \arrow{d}[swap]{f\, \circ\, \otimes\, =\, g} U \times V &   T \arrow{dl} { \exists ! f} \\
		W
		\end{tikzcd}
	};
	\node at (0.1,0.2) {$ \blacksquare $};
	\end{tikzpicture}
\end{center}
für eine eindeutig bestimmte lineare Abbildung $ f: T \to W  $. Formel ist $ (T;\otimes) $ bis auf Isomorphi eindeutig bestimmt.
\subsection{Satz:}
Je zwei K-VR U und V besitzen ein Tensor-Produkt $ (T;\otimes) $, dass bis auf Isomorphi eindeutig bestimmt ist. Schreibe $ U \otimes V $.
\Bew{}{
Seien $ \{a_I\}_{i \in I} $ Basis von U. $ \{b_j\}_{j\in J} $ Basis von V.\\
Wähle Elemente $ (c_{i,j})_{(i,j) \in I \times J} $ K-linear unabh.\\[5pt]
Setze 
\begin{align*}
T &= \spa(\{c_{i,j}\}_{(i,j) \in I \times J})\\
&= \bigg\{ \sum_{\tx{endliche}} \lambda_{i',j'} c_{i,j} \bigg\}
\end{align*}
\begin{align*}
\otimes : U \times V &\to T\\
(a_i,b_i) &\mapsto c_{i,j}
\end{align*}
Erweitern wegen Bilinearität:
$$\otimes (\sum \lambda_i a_i, \sum \mu_j b_j) = \sum_i \sum_j \lambda_i \mu_j a_{ij}$$
Sei nun $ g: U \times V \to W \ \ , \ \ (a_i,b_j) \mapsto g(a_i,b_j) $ bilinear

\begin{align*}
f: T & \to W\\c_{i,j} &\mapsto g(a_i,b_j)
\end{align*}
erweitere f aus \underline{linearität} $ \rightarrow $ f ist eindeutig \underline{bestimmt}
$$f \circ \otimes (a_i,b_j) = g(a_i,b_j)$$
$$ \rightarrow f \circ \otimes = g$$
$$ a_i \otimes b_j := c_{i,j}$$
$$\cancel{\otimes (a_i,b_j)} $$
}

% neue Vorlesung

\subsection{Bew:} 
Sei $ \{a_i\}_{i\in I} $ Basis von U, $ \{b_j\}_{j \in J} $ Basis von V
$$\forall i \in I , j \in J \rightarrow T_{ij}$$
$$T = \Big\{ \sum_{\tx{endliche}} \equalto{\lambda_j}{\in K} T_{ij} \Big\}$$
\begin{align*}
\otimes : \  & U \times V \to T \quad \rightarrow \tx{ Erweitere sie um Bilinearität}\\
& (a_i,b_j) \to T_{ij}
\end{align*}

% tikz bild 1 in bildern

$ F(T_{ij}) = g(a_i,b_j) $ ist eindeutig bestimmt!\\[10pt]
\textbf{Eindeutigkeit:}\\
Sei  $ T' $ auch ein Tensorprodukt von U,V $\quad \otimes ' : U \times V \to T' $

% tikz bild 2 in bildern

\subsection{Kor:}
Für jede Basis $ \{a_i\}_{i\in I} $ von U und jede Basis $ \{b_j\}_{j\in J} $ von V ist: $ \{a_i \otimes b_j\}_{(i,j) \in I \otimes J} $ eine Basis von $ U \otimes V $.\\
Isbesondere falls $ \dim U , \dim V =  \infty \quad $ ist $ \quad \dim (U  \otimes  V) = \dim U  \cdot  \dim V$

\subsection{Kor:}
Jedes Element $ w $ von $ U \otimes V $ lässt sich schreiben als $ \sum_{k=1}^{n} a_{i_k} \otimes v_k $ für $ v_k \in V $ eindeutig bestimmt.

\Bew{}{
$$w \in U \otimes V$$
$$\{\equalto{ a_i \otimes b_j }{\otimes (a_i,b_j)}\} \quad \tx{Basis}$$
\begin{align*}
w &= \sum \lambda_{ij} a_i \otimes b_j\\
&= \sum a_i \otimes \lambda_{ij} b_j\\
&= \sum_i \sum_j a_i \otimes \lambda_{ij} b_j\\
&= \sum_i a_i \otimes \underbrace{\sum_j \lambda_{ij} b_j}_{v_i \in V}
\end{align*}
\textbf{Frage:} Warum sind die $ v_i' $ s eindeutig bestimmt?
$$w = \sum a_i \otimes v_i' \quad \longrightarrow \quad v_i' = \sum \mu_{ij} b_j$$
$$ w = \sum a_i \otimes \Big( \sum \mu_{ij} b_j \Big) = \sum_i \sum_j \mu_{ij} a_i \otimes b_j$$
$$\Rightarrow \lambda_{ij} = \mu_{ij} \Rightarrow v_i = v_i'$$
$ \{ a_i \otimes b_j \} $ eine Basis von $ U \otimes V $
}\\[5pt]
\textbf{\underline{Achtung!}}\\
Nicht jedes Element von $ U \otimes V $ lässt sich als ein rein Tensor $ u \otimes v $ schreiben!

\Bew{}{
$ U = V = \mathbb{R}^2 $ mit der Standartbasis $ \{ e_1, e_2\} $
}

\subsection{Beh:}
$$w = e_1 \otimes e_2 + e_2 \otimes e_1 \in \mathbb{R}^2 \otimes \mathbb{R}^2$$
angenommen:
$$w = u \otimes v \qquad u,v \in \mathbb{R}^2$$
$$u = (\lambda_1 e_1, \lambda_2 e_2) \qquad v = (\mu_1 e_1, \mu_2 e_2)$$
$$u \otimes v = \lambda_1 \mu_1 e_1 \otimes e_1 + \lambda_1 \mu_2 e_1 \otimes e_2 + \lambda_2 \mu_1 e_2 \otimes e_1 + \lambda_2 \mu_2 e_2 \otimes e_2$$
$$ \left.\begin{array}{c}
\lambda_1 \mu_1 = 0\\
\lambda_2 \mu_2 = 0\\
\lambda_1 \mu_2 = 1\\
\lambda_2 \mu_1 = 1
\end{array} \right\} \begin{array}{cc}
\rightarrow \mu_1 = 0\\
\uparrow & \rightarrow \mu_2 = 0\\
\rightarrow \lambda_1 \neq 0 & \uparrow\\
& \rightarrow \lambda_2 \neq 0\\
\end{array}$$
$$\rightarrow v = 0 \Rightarrow u \otimes v = 0 \neq e_1 \otimes e_2 + e_2 \otimes e_1$$

\subsection{Lemma:}
V K-VR
$$ K \otimes V \simeq V$$

\Bew{}{
Wir wollen zuerst eine Abbildung von $ K \otimes V \to V $ konstruieren

% tikz bild 3

\begin{align*}
G: &V \to K \otimes V \quad \tx{ ist linear}\\
&v \ \mapsto \ 1 \otimes v
\end{align*}
$ G $ ist surjektiv 
$$\equalto{\lambda \cdot (1 \otimes v)}{\equalto{1 \otimes \lambda \cdot v}{G(\lambda v)}}$$
Weil 1 eine Basis von K über K ist!\\
Zu zeigen: $ F : K \otimes V \to V $ Isomorphismus
\begin{align*}
G \circ F(\lambda \otimes v) &= G(\lambda v) \\
&= 1 \otimes \lambda \cdot v\\
&= \lambda(1 \otimes v)\\
&= \lambda \otimes v
\end{align*}
$$F \circ G(v) = F(1 \otimes v) = 1 \cdot v = v$$
F und G sind Inverse voneinander
}

\subsection{Lemma:}
\begin{enumerate}[a)]
	\item $ F: U \to U'  \quad G: V \to V' $ lineare Abbildungen\\
	$ \Rightarrow \exists \ F \otimes G: U \otimes V \to U' \otimes V' \quad u \otimes v \mapsto F(u) \otimes G(v) $ lineare Abbildung
	\item $ Id_{Iu} \otimes Id_{Iv} = Id_{u \otimes v} $
	\item $ (F_1 + F_2) \otimes G = F_1 \otimes G + F_2 \otimes G $
	\item $ (\lambda F) \otimes G = \lambda (F \otimes G) $
	\item $ (F_2 \circ F_1) \otimes (G_2 \circ G_1) = (F_2 \otimes G_2) \circ (F_1 \otimes G_1) $
\end{enumerate}

% tikz bild 4

\Bew{}{
\begin{enumerate}[a)]
	\item Sei $$U \times V \to U' \otimes V' \quad (u,v) \mapsto F(u) \otimes G(v) \quad \tx{bilinear}$$
	
	% tikz bild 5
	
	\item trivial
	\item einfach:
	\item einfach:
	\begin{align*}
	(F_1+F_2) \otimes G(u\otimes v) &= (F_1 + F_2) (u) \otimes G(v)\\
	&=(F_1(u) + F_2(v)) \otimes G(v)\\
	&=\equalto{ F_1(u) \otimes G(v) }{(F_1 \otimes G) (u \otimes v)} + \equalto{ F_2(u) \otimes G(v) }{(F_2 \otimes G) (u \otimes v)}
	\end{align*}
	\item \begin{align*}
	&\phantom{=} (F_2 \circ F_1) \otimes G_2 \circ G_1 (v)\\
	&= F_2 \circ F_1 (u) \otimes G_2 \circ G_1 (v)\\
	&= F_2 (F_1(u)) \otimes G_2(G_1(v))\\
	&= F_2 \otimes G_2 \equalto{(F_1(u) \otimes G_1(v))}{(F_1 \otimes G_1) (u \otimes v)} \quad \tx{ ok !}
	\end{align*}
\end{enumerate}

}

	 
\end{document}
